\chapter{El infinito}

Según dicen Rey Pastor y Babini en~\cite{BabiniReyPastor2}, las consideraciones
de índole infinitesimal son casi tan antiguas como la matemática, ya que
podremos encontrar rastros del uso de los métodos infinitesimales en todas las
etapas de la evolución matemática. A lo largo de este capítulo, entonces,
intentaremos mostrar algunas de las distintas manifestaciones que hace el
infinito en el pensamiento matemático.

Ya vimos que los griegos no estaban del todo cómodos con la idea del infinito y
que vieron necesario evitarlo y lograron establecer ciertas bases que les
permitieran tratar el infinito de forma más o menos rigurosa. Vimos que el
descubrimiento de los irracionales generó graves problemas en el pensamiento
matemático de aquella época. 

\index{Paradoja!de Zenón}
Por Aristóteles conocemos las paradojas de Zenón, aproximadamente del año 450
a.  C. Estas paradojas son argumentos contra algunas ideas pitagóricas, ya que
muestran que la concepción de los cuerpos como suma de puntos, o del tiempo
como suma de instantes, lleva a contradicciones. 

Una de las paradojas, la de Aquiles y la tortuga, nos muestra que el veloz
corredor nunca alcanzará a la lenta tortuga, sin importar qué tan escasa sea la
distancia que los separe, ya que cuando el corredor haya recorrido aquella
distancia y llegue donde estaba la tortuga, esta habrá ya avanzado un poco. Y
cuando el corredor llegue allí, la tortuga habrá avanzado incluso un poco más,
y cuando el corredor\dots y así sucesivamente. 

Otra de las paradojas de Zenón, la paradoja de la dicotomía, nos muestra que
para que alguien pueda caminar desde un lugar a otro, deberá recorrer primero
la mitad de la distancia entre estos dos lugares, pero antes de recorrer la
mitad de esta distancia deberá recorrer un cuarto de esta distancia, pero antes
deberá recorrer una octava parte de esta distancia\dots y así sucesivamente. 

La tercera de las paradojas, conocida como la paradoja de la flecha, se nos
presenta una flecha en vuelo y se nos hace observar que en cada instante de
tiempo no se produce movimiento alguno. Si todo está inmóvil en cada instante y
el tiempo está compuesto de instantes, entonces el movimiento es imposible. 

Sin entrar en detalles vemos claramente que el manejo poco cuidadoso del
infinito es algo muy peligroso. Quizá estos argumentos de Zenón hicieron
entonces que algunos de los matemáticos griegos posteriores intentaran evitar
toda manipulación más o menos inocente que involucrara el infinito.

En 1906 se encontró un tratado de Arquímedes donde podemos ver cómo es que
Arquímedes logró descubrir muchos de sus teoremas. El método muestra que muchos
de los profundos resultados de Arquímedes fueron encontrados gracias al uso de
argumentos dudosamente rigurosos que involucraban al infinito; las
demostraciones rigurosas de aquellos resultados venían después. El método de
Arquímedes es más o menos similar a esa idea que hoy conocemos como el princpio
de Cavallieri.

Eudoxo fue uno de los matemáticos griegos que trabajó para que pudiera
comprenderse y utilizarse correctamente la idea del infinito. Su teoría de las
proporciones, aproximadamente del año 350 a.C. y explicada en el libro V de los
Elementos de Euclides, permite tratar, gracias al uso de números racionales,
como números a cantidades geométricas tales como longitudes de segmentos. Como
vimos, los griegos no aceptaban números irracionales pero sí aceptaban
cantidades geométricas irracionales. Sorprendentemente, la idea de Eudoxo puede
usarse para definir números irracionales pero la matemática tardó unos dos mil
años en desarrollar rigurosamente a los números reales. Por ejemplo, $\sqrt{2}$
quedará determinado por dos conjuntos de números racionales positivos,
$\{r\in\Q_{>0}:r^2<2\}$ y $\{r\in\Q_{>0}:r^2>2\}$. Dedekind definió entonces en
1872 al número $\sqrt{2}$ como este par de conjuntos. Como nos dice Stillwell,
la idea de las cortaduras de Dedekind es una ``vuelta de tuerca'' de aquella
idea de Eudoxo. 

Más tarde Eudoxo concibió una generalización de su teoría de las proporciones,
hoy conocida como el método exhaustivo. En el libro XII de los Elementos de
Euclides nos encontramos con aproximaciones por polígonos para el círculo y
aproximaciones para una pirámide. 

Arquímedes llevó el método de exhaustivo de Eudoxo hacia el máximo nivel de
madurez y logró calcular volumen y área de la esfera y el área debajo de un
segmento parabólico. Para el área debajo de la parábola la idea de Arquímedes 
se basa en la construcción de una sucesión infinita de triángulos cuya suma 
de áreas es igual a 
\[
	1+\frac14+\frac{1}{16}+\frac{1}{64}+\cdots=\frac43.
\]
Este es además el primer ejemplo del cálculo de la suma de una serie infinita.
%es la siguiente.  Primero se divide la región cuya área se quiere calcular en
%triángulos $\Delta_1$, $\Delta_2$, $\Delta_3$\dots y luego se observa que valen
%las siguientes fórmulas
%\begin{align*}
%	&\Delta_3=\frac18\Delta_1,\\
%	&\Delta_2=\Delta_3,\\
%	&\Delta_2+\Delta_3=\frac14\Delta_1,\\
%	&\Delta_4+\Delta_5+\Delta_6+\Delta_7=\frac{1}{16}\Delta_1
%\end{align*}
%Luego el área es igual a 
%\[
%	\Delta\left(1+\frac14+\frac{1}{4^2}+\cdots\right)=\frac43\Delta_1
%\]
Arquímedes, sin embargo, no usó series infinitas, aunque las series aparezcan
de alguna forma en el libro IX de Euclides en su teorema sobre números
perfectos. 

El método de Arquímedes no funciona siempre. Por ejemplo, no puede usarse para
calcular el área debajo de la curva $\frac{1}{x}$. Hoy sabemos que esto se debe
a que la función $\log(x)$ no puede expresarse mediante funciones elementales.

Si bien las series infinitas aparecieron en Grecia, hay que mencionar que en
aquellos tiempos se pretendía trabajar con sumas finitas arbitrarias 
\[
	a_1+a_2+a_3+\cdots+a_n
\]
y no con sumas infinitas como 
\[
	a_1+a_2+a_3+\cdots
\]
En la matemática griega aparecen algunas sumas infinitas tales como
\[
	\sum_{n=1}^\infty\frac{1}{2^n}=1
\]
en los trabajos de Zenón y  
\[
	\sum_{n=0}^\infty\frac{1}{4^n}=\frac43
\]
en el cálculo de Arquímedes del área bajo un segmento parabólico. Ambas
expresiones son casos particulares de la fórmula para la serie geométrica, 
\[
	\sum_{n=0}^\infty ar^n=\frac{a}{1-r},
\]
válida para todo $r$ tal que $|r|<1$. 

En 1350 aparece por primera vez una serie infinita no geométrica, donde Suiseth
considera la fórmula
\[
	\sum_{n=1}^\infty\frac{n}{2^n}=2,
\]
Esta fórmula fue encontrada independientemente y demostrada geométricamente por
Oresme, también en 1350. Ese mismo año Oresme demostró que la serie 
\[
	\sum_{n=1}^\infty\frac1n
\]
diverge. La demostración de Oresme es precisamente esa bonita demostración que
hoy vemos en nuestros cursos de cálculo:
\begin{align*}
	1&+\frac12+\left(\frac13+\frac14\right)+\left(\frac15+\frac16+\frac17+\frac18\right)+\cdots\\
	&>1+\frac12+\left(\frac14+\frac14\right)+\left(\frac18+\frac18+\frac18+\frac18\right)+\cdots\\
	&=1+\frac12+\frac12+\frac12+\cdots
\end{align*}

En el siglo XVI el matemático portugués Tomas calculó con exactitud la suma de
algunas series convergentes tales como 
\begin{align*}
	1+\frac74\frac12+\frac{11}{8}\frac14+\frac{19}{16}\frac18+\cdots = \frac52, && 
	1+\frac42\frac12+\frac76\frac14+\frac{13}{12}\frac18+\cdots=\frac{20}{9}.
\end{align*}
Aproximó además otras series convergentes. Mostró, por ejemplo, que la suma
\[
	1+\frac21\frac12+\frac32\frac14+\frac43\frac18+\cdots
\]
es un número acotado entre 2 y 4. El valor real de esta suma puede calcularse y
es igual a $2+\log 2=2,693\dots$ 

Unos años más tarde comenzaron a considerarse series de potencias. Mercator
encontró en 1668 la fórmula
\[
	\log(1+x)=x-\frac{x^2}{2}+\frac{x^3}{3}-\frac{x^4}{4}+\cdots
\]
Obtuvo esa fórmula al integrar término a término 
la expresión
\[
	\frac{1}{1+x}=1-x+x^2-x^3+\cdots
\]


El cálculo\dots

Para Newton, el cálculo es el álgebra de los polinomios infinitos. Para Leibniz, el álgebra de los infinitesimales. De hecho,
la derivada 
\[
	\dfrac{dy}{dx}
\]
es el cociente entre el infinitesimal $dy$ y el infinitesimal $dx$ y la integral 
\[
	\int f(x)dx
\]
es la suma de los infinitesimales $f(x)dx$. 

El cálculo fue extremadamente exitoso ya que permitió reemplazar complicados
cálculos que involucran el método exhaustivo por cálculos de rutina más o menos
sencillos. 

La idea de integración aparece al intentar aproximar el área debajo de las curvas $y=x^k$ 
por rectángulos. Los árabes, cerca del año 1000, lograron calcular la suma
\[
	1^k+2^k+3^k+\cdots+n^k
\]
para $k\in\{1,2,3,4\}$ y con esto pudieron calcular el volumen de un cierto sólido de revolución. 
Cavalieri extendió estos resultados hasta $k=9$ en el año 1635 y con esto logró calcular
\[
	\int_0^a x^kdx=\frac{a^{k+1}}{k+1}
\]
para todo $k\in\{1,\dots,9\}$, resultado que lo indujo a conjeturar que la
fórmula era válida para todo entero positivo $k$. Fermat, Descartes y Rooberval
demostraron aquella conjetura de Cavalieri alrededor de 1630. Fermat incluso
observó que aquella fórmula era válida incluso si $k$ era un número racional.
Cavalieri introdujo lo que hoy llamamos el \emph{principio de Cavalieri}, que
suele verse en nuestros cursos de cálculo. La idea de este principio es similar
a la idea de Arquímedes encontrada recién en 1906. Curiosamente, Torricelli,
contemporáneo de Cavalieri, especulaba que la idea detrás del principio de
Cavalieri, era ya conocida por los griegos. Con esta idea, en 1644, calculó el
área de una parábola esencialmente de la misma forma en que lo había hecho
Arquímedes. Demostró en 1643 que la superficie de revolución obtenida al rotar
$y=1/x$ alrededor del eje $x$, con $x>1$, tiene volumen finito y superficie
infinita.\framebox{dibujito}

La diferenciación vino después, aunque hoy en día, en los cursos de cálculo
suele verse antes, ya que se considera más fácil de entender que la
integración.  Arquímedes calculó la recta tangete de la espiral que vemos en la
figura\dots\framebox{espiral} pero este es el único ejemplo de un cálculo de
esos que hoy expresaríamos como
\[
	\lim_{\Delta x\to 0}\frac{f(x+\Delta x)-f(x)}{\Delta x},
\]
cálculo introducido por Fermat en 1629, y publicados muchoh tiempo después, en
1679, en el caso en que $f(x)$ es un polinomio y utilizado para encontrar
extremos y tangentes de las curvas definidas por polinomios. Stillwell nos
muestra con un ejemplo cómo hacía Fermat para calcular la pendiente de la recta
tangente a la curva $y=x^2$ es igual a $2x$. Primero debe considerarse la
cuerda que une los puntos $(x,x^2)$ y $(x+E,(x+E)^2)$, donde $E$ es una
cantidad no determinada, un ``elemento infinitesimal'' que cuando sea necesario
será igual a cero. 
Se calcula entonces
\[
	\frac{(x+E)^2-x^2}{E}=\frac{2xE+E^2}{E}=2x+E.
\]
En aquellos tiempos aquel argumento generó mucha controversia, ya que se asumía que
$2x+E=2x$ y simultáneamente $E\ne 0$. Hoy sabemos que lo que pasa es que 
\[
	\lim_{E\to 0}(2x+E)=2x,
\]
algo que en aquellos tiempos no se sabía. El método funcionaba perfectamente
para polinomios de una variable y pudo adaptarse para curvas dadas por la
ecuación $p(x,y)=0$, donde $p(x,y)$ es un polinomio de dos variables. 

Nos toca hablar del cálculo de Newton y Leibniz. Inevitablemente entonces nos encontraremos 
con series infinitas. 

En 1593 Vieta publicó la primera expresión convergente de un producto
infinito que involucra al número $\pi$:
\begin{align}
	\label{eq:Vieta}
	\frac{2}{\pi}&=\cos(\pi/4)\cos(\pi/8)\cos(\pi/16)\cdots
\end{align}
Vieta parte de la expresión
del perímetro $p$ de un polígono regular de $2^{n-1}$ lados.  Gracias
a la fórmula 
\begin{equation}
	\label{eq:sin(2x)}
	\sin x=2\sin(x/2)\cos(x/2), 
\end{equation}
observa que
\[
	\frac{4}{p}=\cos(\pi/4)\cos(\pi/8)\cos(\pi/16)\cdots\cos(\pi/2^{n-1}).
\]
Al pasar del polígono a la circunferencia, se obtiene la fórmula que buscamos.
Euler generalizó esta expresión al observar que la fórmula~\eqref{eq:sin(2x)}
implica que 
\[
	\dfrac{\sin x}{2^n\sin\left(\dfrac{x}{2^n}\right)}
	=\cos\left(\dfrac{x}{2}\right)\cos\left(\dfrac{x}{4}\right)\cdots\cos\left(\dfrac{x}{2^n}\right).
\]
algo que a su vez implica que 
\[
	\frac{\sin x}{x}
	=\cos\left(\dfrac{x}{2}\right)\cos\left(\dfrac{x}{4}\right) \cos\left(\dfrac{x}{8}\right)\cdots
\]
Al evaluar esta última fórmula en $x=\pi/4$ obtenemos la fórmula~\eqref{eq:Vieta}. 

\begin{exercise}
	Demuestre que la fórmula~\eqref{eq:Vieta} puede reescribirse como aquella
	fórmula que vimos en la página~\pageref{Vieta}:
	\[
		\frac{2}{\pi}
		=\dfrac{\sqrt{2}}{2}\dfrac{\sqrt{2+\sqrt{2}}}{2}\dfrac{\sqrt{2+\sqrt{2+\sqrt{2}}}}{2}\cdots
		%=\left(\dfrac12\right)^{1/2}\left(\dfrac12\left(1+\sqrt{\dfrac12}\right)\right)^{1/2}\left(\frac12\left(1+\sqrt{\frac12\left(1+\frac12\right)}\right)\right)^{1/2}\cdots
	\]
\end{exercise}

Wallis escribió en 1655 un libro de texto donde intentó sistematizar el cálculo
de áreas y volúmenes. Demostró, por ejemplo, que 
\[
	\int_0^1 x^pdx=\frac{1}{p+1}
\]
para $p$ un entero positivo, al observar 
que 
\[
	\lim_{n\to\infty}\frac{0^p+1^p+\cdots+n^p}{n^p+n^p+\cdots+n^p}=\frac{1}{p+1}.
\]
Calculó además
\[
	\int_0^1 x^{m/n}dx
\]
con nuevas técnicas. Su forma de trabajar con los infinitesimales era algo
ambigua y este no era el único problema que nuestro rigor matemático
encontraría en el trabajo de Wallis. Su uso de la inducción era algo
deficiente, ya que Wallis se contentaba con mostrar que una cierta propiedad
era válida en los primeros casos. Otra de las fórmulas demostradas por Wallis
es la siguiente:
\begin{equation}
	\label{eq:Wallis}
	\frac{\pi}{4}=\frac23\frac43\frac45\frac65\cdots
\end{equation}
En el libro de Wallis aparece una consecuencia de la fórmula~\eqref{eq:Wallis}
encontrada por Brouncker:
\begin{equation}
	\label{eq:Brouncker}
	\frac{4}{\pi}=1+\dfrac{2}{2+\dfrac{3^2}{2+\dfrac{5^2}{2+\dfrac{7^2}{2+\cdots}}}}
\end{equation}
Euler observó que a partir de la fórmula 
\begin{equation}
	\label{eq:Leibniz}
	\frac{\pi}{4}=1-\frac13+\frac15-\frac17+\cdots,
\end{equation}
que vimos en la página~\pageref{LeibnizGregory} podemos obtener 
la fracción continua~\eqref{eq:Brouncker}. Primero observemos que vale la siguiente fórmula
\[
	\dfrac{1}{a_1}-\dfrac{1}{a_2}=\dfrac{1}{a_1+\dfrac{a_1^2}{a_2-a_1}}.
\]
Esta misma fórmula implicará entonces que vale la siguiente fórmula 
\[
	\dfrac{1}{a_1}-\dfrac{1}{a_2}+\dfrac{1}{a_3}=\dfrac{1}{a_1+\dfrac{a_1^2}{a_2-a_1+\dfrac{a_2^2}{a_3-a_2}}}
\]
Al continuar con este procedimiento obtendremos una expresión que nos permitirá
reemplazar a nuestra suma alternada por una fracción continua. 

\begin{theorem}[Leibniz]
	\index{Teorema!de Leibniz}
	\[
	\frac{\pi}{4}=1-\frac13+\frac15-\frac17+\cdots
	\]
\end{theorem}

\begin{proof}
	Partimos de la fórmula 
	\[
		\frac{1-t^n}{1-t}=1+t+\cdots+t^{n-1},
	\]
	que puede escribirse como
	\[
		\frac{1}{1-t}=1+t+\cdots+t^{n-1}+\frac{t^n}{1-t}.
	\]
	Esta fórmula en el caso $t=-x^2$ nos dice que 
	\[
		\frac{1}{1+x^2}=1-x^2+x^4-x^6+\cdots+(-1)^{n-1}x^{2n-2}+(-1)^n\frac{x^{2n}}{1+x^2}.
	\]
	Integramos esta igualdad entre $0$ y $1$ y obtenemos
	\[
		\frac{\pi}{4}=\int_0^1\frac{dx}{1+x^2}=1-\frac13+\frac15-\frac17+\frac19-\cdots+(-1)^{n-1}\frac{1}{2n-1}+T_n,
	\]
	donde 
	\[
		T_n=(-1)^n\int_0^1\frac{x^{2n}}{1+x^2}dx.
	\]

	Para terminar la demostración debemos ver que 
	$\displaystyle{\lim_{n\to\infty}|T_n|=0}$. Para eso primero
	observamos que, como $0\leq x\leq 1$, entonces 
	\[
		\frac{x^{2n}}{1+x^2}\leq x^{2n}.
	\]
	Al integrar esta última desigualdad, tenemos
	\[
		|T_n|=\int_0^1\frac{x^{2n}}{1+x^2}dx\leq\int_0^1x^{2n}dx=\frac{1}{2n+1},
	\]
	y entonces $|T_n|\to 0$ si $n\to\infty$. 
\end{proof}

El cálculo de Newton dependía principalmente de la manipulación de series
infinitas. Si entendemos al cálculo como el álgebra de las series infinitas,
entonces Newton es sin duda uno de sus fundadores.  Mercator demostró en 1668
que 
\[
	\int_0^x\dfrac{dt}{1+t}=x-\frac{x^2}{2}+\frac{x^3}{3}-\frac{x^4}{4}+\cdots
\]
Newton descubrió esa misma fórmula independientemente en 1665 junto con
expresiones similares para $\arctan x$, $\sin x$ y $\cos x$.  Para realizar
estos cálculos primero expresaba la función como una serie para luego integrar
esta serie término a término. Veamos un ejemplo:

%\begin{example}
%	\begin{align*}
%	&\int_0^x \dfrac{dt}{1+t}=\int_0^x(1-t+t^2-t^3+\cdots)dt=x-\frac{x^2}{2}+\frac{x^3}{3}-\frac{x^4}{4}+\cdots
%\end{align*}
%\end{example}
%\begin{example}
\begin{align*}
	\arctan x&=\int_0^x \dfrac{dt}{1+t^2}\\
	&=\int_0^x(1-t^2+t^4-t^6+\cdots)dt\\
	&=x-\frac{x^3}{3}+\frac{x^5}{5}+\cdots
\end{align*}
%\end{example}

Gracias a ser capaz de invertir series infinitas, Newton también logró encontrar la expresión
\[
	\log(1+x)=\int_0^x\dfrac{dt}{1+t}=x-\frac{x^2}{2}+\frac{x^3}{3}-\cdots
\]
Este procedimiento fue correctamente justificado por De Moivre en 1698.  
En 1669 Newton encontró una fórmula aún más sorprendente:
\dots

Leibniz publicó el primer trabajo sobre el cálculo en 1684. Los trabajos de Newton fueron inicialmente rechazados para su publicación\dots
Disputa entre Newton y Leibniz. No queremos meternos en este asunto y referiremos a \dots para el lector interesado 
en conocer más sobre aquella pelea. 

Según Stillwell no hay duda: ambos descubrieron el cálculo en forma
independiente. La notación utilizada por Leibniz era amejor que aquella de
Newton y eso hizo que\dots Leibniz introdujo la notación que hoy en día
utilizamos para derivadas y para integrales y encontró las reglas para calcular
la derivada de una suma, un producto y un cociente de funciones. Demostró
además lo que hoy conocemos como el teorema fundamental del cálculo, que
Leibniz escribía como
\[
	\frac{d}{dx}\int f(x)dx=f(x).
\]
Newton ya conocía este resultado. Más aún, al menos de forma geométrica, este resultado era también
conocido por Barrow, uno de los profesores de Newton. 

Leibniz prefería expresiones cerradas para las series infinitas y para evaluar la integral 
de una función $f(x)$ necesitaba una función cuya derivada fuera la función $f(x)$. 

Johann Bernoulli hizo aportes significativos al cálculo de Newton y Leibniz. En 1696 publicó el primer libro de texto sobre el tema, aunque 
el libro fue publicado bajo el nombre de su estudiante L'Hopital, se cree que 
para agradecerle la ayuda económica que L'Hopital le había brindado. Bernoulli y Leibniz conocían la diferenciación parcial pero
mantuvieron ese resultado en secreto por unos veinte años, ya que lo consideraban un arma secreta que les permitía atacar problemas sobre curvas. 
Encontró la expresión
\[
	\int_0^1 x^xdx=1-\frac{1}{2^2}+\frac{1}{3^3}-\frac{1}{4^4}+\cdots
\]

Los métodos poco rigurosos que involucran la manipulación de series hoy pertenecen a la teoría de funciones generatrices. Este concepto 
fue introducido por De Moivre en 1730 (aunque al parecer había sido descubierto
en 1728 por algún Bernoulli)
y fue utilizado para encontrar una fórmula cerrada para la sucesión de Fibonacci. Recordemos que la sucesión de Fibonacci
se define recursivamente mediante 
\[
	F_0=0,\quad
	F_1=1,\quad
	F_{n+2}=F_{n+1}+F_n\quad\text{para $n\geq 0$}.
\]
Sin preocuparnos por la convergencia, 
consideremos la serie infinita
\begin{align*}
	f(x)&=F_0+F_1x+F_2x^2+\cdots
\end{align*}
Como $xf(x)=F_0x+F_1x^2+\cdots$ y además $x^2f(x)=F_0x^2+F_1x^3+\cdots$, podemos escribir
\[
	f(x)(1-x-x^2)=F_0+F_1x-F_0x+(F_2-F_1-F_0)x^2+\cdots
\]
Pero como los coeficientes de la serie que vemos en el miembro de la derecha son todos cero salvo el primero, nos queda
que
\[
	f(x)=\frac{x}{1-x-x^2}.
\]
Al observar que
\[
	1-x-x^2=(-1)\left(x-\frac{1+\sqrt{5}}{2}\right)\left(x-\frac{1-\sqrt{5}}{2}\right)
\]
podemos obtener una fórmula cerrada para en $n$-ésimo término de la sucesión de Fibonacci:
\[
	F_n=\frac{1}{\sqrt{5}}\left( \left(\frac{1+\sqrt{5}}{2}\right)^n-\left(\frac{1-\sqrt{5}}{2}\right)^n\right).
\]

\begin{exercise}
	Demuestre que 
	\[
		\lim_{n\to\infty}\frac{F_{n+1}}{F_n}=\frac{1+\sqrt{5}}{2}.
	\]
\end{exercise}

\begin{exercise}
	Utilice la fórmula
	\[
		\dfrac{1}{1+\dfrac{F_n}{F_{n+1}}}=\dfrac{F_{n+1}}{F_{n+2}}
	\]
	para encontrar una fracción continua para el número $\dfrac{1+\sqrt{5}}{2}$. 
\end{exercise}


