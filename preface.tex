\preface

Estas notas pertenecen a un curso dictado el segundo cuatrimestre de 2019, en
la Facultad de Ciencias Exactas y Naturales de la Universidad de Buenos Aires.
Me propuse que ni el curso ni estos apuntes fueran una colección
de anécdotas sobre las personas que desarrollaron la matemática. No tengo nada
en contra de tales anécdotas: la matemática y su historia es  
más interesante que esos simpáticos cuentos acerca de las hazañas o particularidades
de las personas involucradas.
Los interesados en la vida de algunos personajes de la matemática deberían
consultar el libro de Bell~\cite{MR3728304}, el libro de Osen~\cite{MR0497806},
o quizá mejor aún las biografías 
del \href{http://www-history.mcs.st-andrews.ac.uk/}{MacTutor History of Mathematics
Archive}.
%disponibles en:
%\begin{center}
%\verb+http://www-history.mcs.st-andrews.ac.uk/+
%\end{center}

Este curso no pretende ser un tratado enciclopédico sobre la historia de la
matemática. El objetivo es, simplemente, el de desarrollar, más o menos 
en detalle, algunos tópicos  
caprichosamente seleccionados, e intentar explicarlos desde un contexto
histórico. Por eso, siempre que sea posible, intentaré utilizar algunos de los 
temas estudiados como excusa para discutir  
algunos temas más o menos actuales de la investigación matemática. 

Muchas páginas de estas notas estarán dedicadas a la matemática
griega. Sin embargo, dado que el curso no está destinado al especialista sino 
al estudiante con ganas de conocer más sobre el desarrollo de la matemática, 
aquel que quiera profundizar en la matemática griega 
deberá consultar, por ejemplo, alguno de los textos de Heath~\cite{MR654679,
MR654680,MR0156760}. 

El curso tiene una estructura similar al libro de
Stillwell~\cite{MR2667826}.  Como ya mencioné, hay una selección personal 
más o menos arbitraria
de los temas a desarrollar. Algunos de estos tópicos son clásicos y
prácticamente imposibles de evitar: el teorema de Pitágoras, los trece libros
de Euclides, el análisis infinitesimal. Otros
temas son un poco más actuales y creo que --quizá por por gusto, lo admito-- 
también son difíciles de omitir; la titánica 
clasificación de los grupos simples finitos
es un claro ejemplo.

Cada uno de los temas que componen el curso será desarrollado en un capítulo. 
La unidad de la matemática me permitirá hacer digresiones
sobre la matemática en las civilizaciones antiguas, sobre cómo fueron creándose
o formalizándose determinadas ideas, sobre cómo la matemática pura se conecta
con otras ciencias, etc. 

Para ilustrar la forma en la que este curso está organizado hay que observar
que las notas comienzan con el teorema de Pitágoras. Después de haber enunciado
el teorema y haber hecho algunos comentarios más o menos conocidos sobre las
demostraciones, quedo casi obligado a mencionar la aparición de este famoso
resultado en civilizaciones antiguas y puedo además intentar explicarlo dentro
del contexto en el que fue descubierto, el ambiente donde un teorema
es descubierto siempre es de gran importancia. Después del
teorema de Pitágoras quedo naturalmente habilitado para hablar de los
pitagóricos y la introducción de los números irracionales. Luego --ya sin culpa-- 
hago un salto caprichoso de unos dos mil años y demuestro que $e$ y $\pi$ son
números irracionales.

Veamos otro ejemplo. En la historia de la matemática la resolución de
ecuaciones polinomiales en una variable ocupa un papel fundamental. Todos
sabemos cómo resolver la ecuación cuadrática. Muchas personas saben que también pueden
resolverse las ecuaciones de grado tres y cuatro, aunque pocas personas conocen explícitamente 
estas fórmulas y seguramente casi nadie las recuerda. Un capítulo de estas notas 
desarrolla algunas de estas fórmulas y nos
lleva naturalmente a la imposibilidad de resolver la ecuación general de grado
cinco con fórmulas que involucren raíces de expresiones racionales de los
coeficientes del polinomio. Estamos entonces frente a una de las motivaciones
de la teoría de grupos, que nos permitirá conectarnos con muchos tópicos
actuales de gran importancia. Dentro de estos tópicos quiero destacar la
clasificación de los grupos simples finitos, resultado conocido como el teorema
de las diez mil --o quizá quince mil-- páginas, o los sistemas formales de 
verificación computacional de demostraciones. 

Los temas elegidos son la excusa perfecta para contar un poco de historia y
para aprender un poco más de matemática. 

\index{OESIS}
\index{Enciclopedia!de sucesiones de enteros}
\subsubsection*{Algunas sucesiones}
Frecuentemente nos encontraremos con sucesiones de números. Para reconocer 
una sucesión particular y ver cómo se conecta con otros tópicos de la matemática
identificaremos nuestra sucesión con una de las sucesiones de la
\emph{enciclopedia de sucesiones de enteros Sloane}, OEIS. En 1965 un
estudiante de doctorado inglés de apellido Sloane comenzó a recopilar
sucesiones ya que suponía que iban a serle de gran utilidad en sus
investigaciones sobre combinatoria. La colección de sucesiones rápidamente tuvo
éxito y por esa razón, en 1973, Sloane publicó un libro con algunas de las
sucesiones de su colección; esta primera versión contenía 2372 sucesiones. En
1995 apareció una nueva versión del libro, esta vez con 5488 sucesiones. El
éxito de estas ediciones hizo que muchos matemáticos se comunicaran con Sloane
y le facilitaran nuevas sucesiones, cosa que hizo que la colección se tornara
gigante e inmanejable. Cuando la enciclopedia alcanzó las 16000 sucesiones,
Sloane comprendió que solamente existía una única forma de continuar con aquel
proyecto: la enciclopedia tenía que transformarse y pasar a ser una base de datos
disponible en Internet. Por casi cuarenta años Sloane fue el encargado del
mantenimiento de la base de datos, hasta que, en 2002 una comisión editorial y
muchos voluntarios se hicieron cargo de continuar con el proyecto. La
enciclopedia está disponible \href{https://oeis.org/}{acá} y contiene
actualmente\footnote{junio de 2019} 320000 sucesiones. Cada una de las
sucesiones de la enciclopedia está representada por un código. La sucesión de
los números primos
\[
	2,3,5,7,11,13,17,19\dots
\]
es la sucesión \href{https://oeis.org/A000040}{A000040} y en esa página 
%y en la dirección \verb+https://oeis.org/A000040+
nos encontraremos no solo a los primeros números primos sino mucha otra
información relevante, desde referencias bibliográficas relacionadas con esta
sucesión y conexiones con otras sucesiones de la enciclopedia, hasta fragmentos
de código en distintos lenguajes de programación que nos permitirán repetir los
cálculos que dan los números que la enciclopedia nos muestra. 


\subsubsection*{Sobre la estructura del curso}

El curso estará organizado de la siguiente forma.  El primer capítulo está
dedicado al teorema de Pitágoras. Mencionaremos algunas demostraciones y cómo
este teorema aparece en distintas épocas y comunidades matemáticas. Hablaremos
además de ternas pitagóricas y nos encontraremos con la oportunidad de
mencionar algunos hechos notables sobre la matemática en babilonia.  El segundo
capítulo es sobre números irracionales. Demostraremos que $\sqrt{2}$ es un
número irracional de varias formas distintas y veremos cómo era tratada la
irracionalidad de ciertos números en la matemática griega. Al final del
capítulo nos encontraremos con la irracionalidad y la historia de dos de las
constantes matemáticas más famosas: los números $e$ y $\pi$. 

En general cada capítulo contiene varios ejercicios de distinto nivel de
dificultad, ya sea para que los estudiantes puedan apreciar las distintas
técnicas utilizadas en la matemática o para que se pueda profundizar en temas
específicos. Algunos de estos ejercicios pueden utilizarse como los temas
necesarios para aprobar la materia. 

Si alguien tuviera comentarios o sugerencias, automáticamente se ganará el
derecho a ser mencionado en la siguiente sección.

\subsubsection*{Agradecimientos}

Quiero agradecer a toda la gente que me prestó libros y que leyó y corrigió
estas notas: Sofía Amnini, Fernando Cukierman, Ricardo Durán, Guillermo Henry, Martín Mereb. También le agradezco a Juan Pablo
Pinasco por compartir conmigo el material que usó para su propio curso sobre la
historia de la matemática. 

\medskip
Versión compilada el \today~a las~\currenttime.

\bigskip
\begin{flushright}
Leandro Vendramin\\Buenos Aires, Argentina\par
\end{flushright}
