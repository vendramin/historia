\chapter{El número $\pi$}

Bien sabemos que el número $\pi$ es la razón entre la circunsferencia y el
díametro de cualquier círculo, algo que en nuestra notación escribiríamos como
\[
	\pi=\frac{C}{D},
\]
donde $C$ es la circunsferencia y $D$ es el diámetro. 

Repasaremos un poco la historia del número $\pi$. Basaremos el capítulo principalmente en el libro~\cite{MR0449960}. 

De los documentos que conocemos puede verse que tanto los babilónicos como los
egipcios conocían la existencia y algunas propiedades del número $\pi$. Los babilónicos lo aproximaban como
\[
	3\frac18=\frac{25}{8}=3,125
\]
y los egipcios como 
\[
	4\left(\frac89\right)^2=\frac{256}{81}\sim 3,16049\dots
\]

No sabemos con exactitud cómo fue que ambas civilazaciones llegaron a encontrar
estas aproximaciones pero no es difícil hacer algunas conjeturas. La idea más
inocente será dibujar algún círculo, medir su circunsferencia, su diámetro y
calcular el cociente de estos números. Sin embargo, no debemos olvidarnos que
en egipto esto se hizo sin tener instrumentos precisos de medición, sin el
algoritmo de división, sin el sistema de numeración decimal, sin regla, sin
compás, sin lápiz ni papel. 

Euler fue el primero en utilizar la letra griega $\pi$ para denotar a esta
famosa constante. 



En 1761 Lambert demostró que $\pi$ es irracional.  La demostración que dio
Lambert sobre la irracionalidad de $\pi$ es bastante difícil. En 1941 Niven
encontró una demostración mucho más sencilla que bien podría darse en cualquier
curso de cálculo. Exponemos a continuación la demostración de Niven:

\begin{theorem}
	El número $\pi$ es irracional.	
\end{theorem}

\begin{proof}
		
\end{proof}

En 1760 Lambert y Riccati, en forma independiente, introdujeron las funciones
hiperbólicas y estudió algunas de sus propiedades. 

\begin{theorem}
	\[
	\sum_{n=1}^{\infty}\frac{1}{n^2}=\frac{\pi^2}{6}.
	\]
\end{theorem}

\begin{proof}
	Vamos a calcular la integral 
	\[
		I=\int_0^1\int_0^1 \frac{1}{1-xy}dxdy
	\]
	de dos formas distintas. Primero escribimos el integrando como una serie y
	hacemos algunos cálculos sencillos:
	\begin{align*}
		I&=\int_0^1\int_0^1 \sum_{n\geq0}(xy)^ndxdy
		=\sum_{n\geq0}\int_0^1\int_0^1 (xy)^ndxdy\\
		&=\sum_{n\geq0}\int_0^1x^ndx\int_0^1y^ndy
		=\sum_{n\geq0}\frac{1}{n+1}\frac{1}{n+1}
		=\sum_{n\geq1}\frac{1}{n^2}=\zeta(2).
	\end{align*}
	Por otro lado, podemos hacer el cambio de variables $u=\frac{x+y}{2}$ y $v=\frac{-x+y}{2}$, tenemos que 
	\begin{align*}
		I&=4\int_0^{1/2}\left(\int_0^u\frac{dv}{1-u^2+v^2}\right)du+4\int_{1/2}^1\left(\int_0^{1-u}\frac{dv}{1-u^2+v^2}\right)du\\
		&=\dots
	\end{align*}
\end{proof}
