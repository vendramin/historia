\chapter{Números complejos}


Para repasar algunos aspectos históricos sobre los números complejos nos
basaremos en el artículo~\cite{MR490654}. 
En un libro escrito por Herón alrededor del año 75 nos encontramos con un
cálculo que involucra el número $\sqrt{81-144}$. Para ``resolver'' este cálculo
Herón hace lo siguiente:
\[
	\sqrt{144-81}=\sqrt{63}\sim 7\frac{15}{16}.
\]
Observemos que la aproximación racional para $\sqrt{63}$ es bastante buena. No
sabemos, sin embargo, si el problema de signo que vemos en la raíz es culpa de
Herón o de la persona a cargo de copiar el libro. En cualquier caso, esta
parece ser la primera aparición del problema que presenta intentar calcular la
raíz cuadrada de un número negativo.

En los libros de Diofanto podemos encontrarnos con el siguiente problema. Se
quiere construir un triángulo rectángulo de área igual a 7 y perímetro igual a
12. Se quiere resolver entonces 
\[
	\frac12 xy=7,\quad
	x^2+y^2=(12-x-y)^2.
\]
Diofanto no resuelve el problema pero observa que tendrá solución solamente si
$(172/2)^2\geq 24\times 336$.  Un cálculo sencillo nos muestra que
\[
	x=\frac{43\pm\sqrt{-167}}{12},
\]
por lo que queda más o menos claro que Diofanto vio al menos una parte del
problema. 

Muchos años después, alrededor del año 1100, nos encontramos con menciones
explícitas del problema o la imposibilidad de calcular raíces cuadradas de
números negativos. 

En un libro publicado en 1494 el matemático italiano Luca Pacioli observó que
la ecuación $x^2+c=bx$ puede resolverse solamente si $\frac14b^2\geq c$, algo
que deja en claro la imposibilidad de calcular raíces cuadradas de números
negativos. Más o menos por la misma época, en 1484, el matemático francés
Chuquet también observó esta imposibilidad en un manuscrito que nunca fue
publicado.

En su famoso \emph{Ars Magna} de 1545 Cardano utilizó por primera vez cálculos
con raíces cuadradas de números negativos. Cardano quería resolver el problema
\[
	x+y=10,\quad
	xy=40
\]
y para ese propósito hizo manipulaciones algebraicas formales y observó que
\[
	x=5+\sqrt{-15},\quad
	y=5-\sqrt{-15}
\]
dan una solución al problema. Intentó entonces verificar que el par
$x=5+\sqrt{15}$, $y=5-\sqrt{15}$ es también una solución y se encontró con que
$xy\ne 40$. Como bien remarca Green en su artículo, es conveniente enfatizar
que Cardano hizo esto con una notación algebraica mucho más precaria que la
nuestra, por lo que muchos de sus cálculos involucraban complicados argumentos
geométricos. 

Curiosamente, los números complejos aparecieron con más fuerza al intentar
resolver ecuaciones cúbicas, un problema de gran importancia para los
matemáticos en épocas de Cardano. De hecho, Cardano encontró un método para
resolver 
\[
	x^3+ax=b
\]
donde $a$ y $b$ son números positivos, que ilustraremos a continuación.

\begin{example}
	Nos interesa resolver $x^3+9x=24$. Para eso, escribimos $x=u-v$, donde
	$uv=9/3=3$. Al reemplazar obtenemos $u^3-v^3=24$ y luego 
	\[
		u^6-24u^3-27=0. 
	\]
	De aquí obtenemos $u^3=12+\sqrt{171}$, $v^3=12-\sqrt{171}$. En
	consecuencia, 
	\[
		x=u-v=\sqrt[3]{12+\sqrt{171}}-\sqrt[3]{12-\sqrt{171}}.
	\]
\end{example}

\begin{exercise}
	Resuelva $x^3+6x=20$.
\end{exercise}

Es interesante mencionar además que Cardano, ya que igual que sus
contemporáneos, no aceptaba números negativos, consideraba que las ecuaciones
$x^3=ax+b$ y $x^3+ax=b$ eran esencialmente distintas, y entonces, para
resolverlas, se requerían métodos distintos.

\begin{example}
	Nos interesa resolver $x^3=15x+4$. Para eso, escribimos $x=u+v$, donde
	$uv=15/3=5$. Tal como hicimos en el ejemplo anterior podremos calcular
	\[
		x=\sqrt[3]{2+\sqrt{-121}}+\sqrt[3]{2-\sqrt{-121}}.
	\]

\end{example}

El ejemplo anterior nos muestra un fenómeno particularmente interesante:
Incluso si intentamos resolver una ecuación cúbica con tres raíces reales, nos
aparecen números complejos. Aparentemente Cardano sabía que $x=4$ era solución
de la ecuación $x^3=15x+4$ y no entendía cómo las soluciones complejas dadas
por su método podían transformarse en esa solución real tan evidente. Hoy en
día entendemos perfectamente el problema y podrá resolverse solamente si
podemos operar con números complejos. 



%Los números complejos en sus orígenes eran utilizados solamente dentro del
%álgebra y como si fueran una especie de notación útil que permitía resolver la
%ecuación cúbica. Se los conocía como \emph{números imposibles}. 
%
%Ahora todos sabemos que la unidad imaginaria $i$ es solución de la ecuación
%$x^2+1=0$, pero en tiempos pretéritos nadie consideraba que todas las
%cuadráticas deberían tener sí o sí una solución. 

%En la resolución de la ecuación cúbica $y^3=py+q$ 
%sabemos que 
%\[
%	y=\dots
%\]
%Observemos que aparecerán los números complejos si 
%$(q/2)^2-(p/3)^3<0$ y entonces siempre hay al menos una solución. Hay un hecho bastante curioso que debemos mencionar. La ecuación\dots
%siempre tendrá al menos una solución real pues
%\dots

%Es natural entonces que nos preguntemos cómo conseguimos una solución real 
%de la forma
%\[
%	\sqrt[3]{a+b\sqrt{-1}}+\sqrt[3]{a-b\sqrt{-1}}.
%\]

Bombelli fue el primer matemático en considerar seriamente a los números
complejos.  Esto le permitió ordenar y presentar el trabajo de Cardano de forma
clara y precisa.  La ecuación
\[
x^3=15x+4
\]
tiene una solución real $x=4$ que podemos obtener por inspección. Las fórmulas
de Cardano nos dicen que
\[
	x=\sqrt[3]{2+11\sqrt{-1}}+\sqrt[3]{2-11\sqrt{-1}}.
\]
Mediante el álgebra de los números complejos podemos demostrar que
\[
	\sqrt[3]{2+11\sqrt{-1}}=2+\sqrt{-1},\quad
	\sqrt[3]{2-11\sqrt{-1}}=2-\sqrt{-1},
\]
lo que implica que $x=4$ tal como esperábamos. 

Bombelli fue precisamente el primero en observar lo mencionado en el párrafo
anterior y fue además el primer matemático capaz de observar que al intentar
resolver ecuaciones cúbicas mediante el método de Cardano involucraría números
de la forma $a+\sqrt{b}$ y $a-\sqrt{b}$. Hoy sabemos que este par de números se
conocen como números complejos \emph{conjugados}, pero esa terminología
apareció en 1821 en los trabajos de Cauchy. Las observaciones sobre el método
de Cardano y la utlización de números complejos, hechas posiblemente entre 1557
y 1560, aparecieron en el libro \emph{Algebra}, publicado en 1572.

\begin{exercise}
	Demuestre que
	\[
		\sqrt[3]{18+26\sqrt{-1}}+\sqrt[3]{18-26\sqrt{-1}}=6.
	\]
\end{exercise}

En el libro de Bombelli nos encontramos además con el siguiente ejemplo, que
dejamos como ejercicio:

\begin{exercise}
	Resuelva $x^3=7x+6$. 
\end{exercise}

Ya mencionamos que Cardano no disponía de nuestra notación algebraica. Para dar
un ejemplo, no disponía del uso de los paréntesis para fijar los órdenes en los
que pueden hacerse operaciones algebraicas, algo que sí tenía Bombelli. El
símbolo $\sqrt{-1}$ fue introducido mucho tiempo después por el matemático
holandés Girard en 1629. Green remarca que otra de delicadas las diferencias de
la notación era el uso de potencias $n$-ésimas, ya que para Cardano $x^2$ y
$x^3$ representaban cuadrados y cubos de lado $x$, respectivamente.
Evidentemente aquella manera de interpretar potencias resulta mucho más difícil
para manipular que la interpretación abstracta que utilizamos en estos días. 

Decartes fue el primero en considerar las partes real e imaginaria de los
números complejos. Aquí conviene mencionar que el nombre de los números
complejos fue concebido por Gauss en 1832. 

En 1673 Wallis fue el primero en intentar encontrar una interpretación
geométrica para los complejos. No logró su objetivo pero curiosamente estuvo
muy cerca de lograr encontrar la interpretación que hoy todos conocemos. ¡No
nos olvidemos que en aquella época incluso hasta los números negativos eran a
veces un dolor de cabeza!

Leibniz también contribuyó a la teoría de los números complejos. En 1676
sorprendió a la comunidad matemática al demostrar que
\[
	\sqrt{1+\sqrt{-3}}+\sqrt{1-\sqrt{-3}}=\sqrt{6}
\]
y factorízó linealmente al polinopmio $x^4+a^4$.  Observó que la suma de dos
complejos conjugados siempre da como resultado un número real y logró demostrar
la validez de las fórmulas de Cardano para encontrar raíces de una ecuación
cúbica.  Sin embargo, Leibniz no fue capaz de encontrar una interpretación
geométrica para los números complejos. 

En 1714 el matemático inglés Cotes demostró que
\[
	\log(\cos\phi+i\sin\phi)=i\phi,
\]
fórmula que implica 
\[
	\cos\phi+i\sin\phi=e^{i\phi},
\]
famosa fórmula que se le atribuye a Euler, así como también la conocida fórmula
incorrectamente atribuida a de Moivre:
\begin{equation}
	\label{eq:deMoivre}
	(\cos\phi+i\sin\phi)^n=\cos(n\phi)+i\sin(n\phi).
\end{equation}

Ya vimos que muchos descubrimientos matemáticos llevan el nombre incorrecto. La
fórmula~\ref{eq:deMoivre} no es una excepción, ya que este matemático nunca
escribió aquella fórmula sino que encontró una fórmula bastante similar:
alrededor de 1730 dio una fórmula para calcular $(\cos\phi+i\sin\phi)^{1/n}$.
De hecho, 
Euler en 1748 fue el primero en enunciar y demostrar la
fórmula~\eqref{eq:deMoivre}. Euler también encontró las fórmulas
\[
	\sin\phi=\frac{e^{i\phi}-e^{-i\phi}}{2i},\quad
	\cos\phi=\frac{e^{i\phi}+e^{-i\phi}}{2},
\]
y fue el primero en utilizar el símbolo $i$ para denotar a $\sqrt{-1}$ en 1777,
aunque apareció impreso por primera vez en 1794. El uso de este símbolo fue
popularizado por Gauss.

\begin{exercise}
	En 1770 Euler ``demostró'' que $\sqrt{-2}\times\sqrt{-3}=\sqrt{6}$. ¿Cuál
	es el problema con esa fórmula de Euler?
\end{exercise}

En 1702 Johnan Bernoulli escribió un trabajo sobre integración y utilizó los
números complejos para escribir
\[
	\frac{1}{1+z^2}=\frac{1/2}{1+zi}+\frac{1/2}{1-zi}
\]
y calculó $\arctan z$. 

\begin{exercise}
	Demuestre que $i^i\in\R$. 
\end{exercise}

El resultado del ejercicio anterior fue encontrado por Euler 
en 1746.

La interpretación geométrica que hoy tenemos para los números complejos fue
descubierta en 1797 por un matemático amateur noruego de apellido Wessel. Esto
también fue hecho en forma independiente por el matemático suizo Argand en
1806. Se cree que Gauss también descubrió en forma independiente esta
interpretación para los números complejos alrededor del año 1800.

\index{Teorema!fundamental del álgebra}
\index{Teorema!de Bolzano}
Hoy en día nadie duda de la importancia de los números complejos. De hecho, el
\emph{teorema fundamental del álgebra} es precisamente ese resultado que afirma
que todo polinomio posee al menos una raíz compleja. El teorema puede
enunciarse en forma equivalente de la siguiente manera: todo polinomio de
$\C[X]$ puede factorizarse linealmente en $\C[X]$. Por ejemplo
\[
	X^4+1=(X-1)(X+1)(X+i)(X-i).
\]

Con la intención de convencernos de que él era el primer matemático que
presentaba una demostración rigurosa del teorema fundamental del álgebra, Gauss
afirmo que d'Alambert había intentado en vano demostrar el teorema 1747. Hoy
sabemos que aquella supuesta demostración rigurosa de Gauss también tenía
algunos defectos, pero el rigor en matemática siempre tiene sentido dentro de
un determinado contexto, y esto no debe olvidarse. Curiosamente, hoy nos
resultaría más o menos fácil tapar los agujeros que tiene la demostración de
d'Alambert, y esto que debe hacerse utilizando teoremas y técnicas básicas del
análisis matemático, algo que no sucede con la demostración dada por Gauss, que
también contiene algunos problemas que no son tan fácilmente reparables.  Ambas
pruebas se basan en propiedades geométricas de los números complejos y
requieren el uso de argumentos de continuidad. Ya vimos que interpretar a un
número complejo $x+iy$ como el punto $(x,y)$ del plano complejo es una idea que
tardó mucho tiempo en materializarse. Esta es una de las razones por la que la
prueba de d'Alambert no resulta del todo clara y para comenzar a repararla es
necesario hacer uso del plano de Argand. Los argumentos de continuidad
necesarios en ambas demostraciones tampoco estaban del todo claros ni para
d'Alambert ni para Gauss, aunque Gauss en 1799 fue capaz de reconocer ciertas
dificultades en relación a la continuidad. Por esa razón, dio en 1816 una
demostración alternativa, bastante algebraica, del teorema fundamental del
álgebra donde se minimiza el rol de la continuidad. Básicamente, lo que esa
segunda demostración encontrada por Gauss utiliza del análisis matemático es un
caso particular del teorema que hoy se le atribuye a Bolzano: una función
polinomial $p(x)$ toma todos los valores entre $p(a)$ y $p(b)$ si $x$ se mueve
entre $a$ y $b$. Bolzano en 1817 hizo explícita la necesidad del concepto de
continuidad para demostrar el teorema fundamental del álgebra y dio una
demostración, no del todo satisfactoria, de su famoso teorema. La demostración
de Bolzano estaba bien encaminada, pero el problema que tenía era medio
inevitable ya que en aquel momento no existía una buena definición de los
números reales. En 1874 Weiestrass estableció y demostró las propiedades
básicas de las funciones continuas y con esas ideas claramente establecidas fue
capaz de completar aquella segunda prueba de Gauss para el teorema fundamental
del álgebra y la demostración original de d'Alambert.

%La demostración de d'Alambert es más fácil que las demostraciones de Gauss ya
%que se basa solamente en propiedades geométricas de los números complejos y de
%propiedades de las funciones continuas. 

\section*{Cuaterniones y otros sistemas numéricos}

Veamos brevemente cómo podemos demostrar la identidad 
	\begin{equation*}
		\begin{aligned}
			(a_1^2+a_2^2+a_3^2+a_4^2)(b_1^2&+b_2^2+b_3^2+b_4^2)\\
			=(a_1 b_1 &- a_2 b_2 - a_3 b_3 - a_4 b_4)^2\\
			&+(a_1 b_2 + a_2 b_1 + a_3 b_4 - a_4 b_3)^2\\
			&+(a_1 b_3 - a_2 b_4 + a_3 b_1 + a_4 b_2)^2\\
			&+(a_1 b_4 + a_2 b_3 - a_3 b_2 + a_4 b_1)^2.
		\end{aligned}
	\end{equation*}
que vimos en el capítulo sobre números poligonales, mediante el uso de los
cuaterniones.  Recordemos que usamos esa fórmula en la demostración del teorema de Lagrange, 
que afirma que todo número es suma de cuatro cuadrados.

Un \textbf{cuaternión}
es una matriz de la forma
\begin{equation}
	\begin{pmatrix}
		\alpha & \beta\\
		-\overline{\beta} & \overline{\alpha}
	\end{pmatrix}
\end{equation}
donde $\alpha$ y $\beta$ son números complejos. 

%Denotaremos por $\mathbb{H}$ al
%conjunto de cuaterniones.
%
%\begin{exercise}
%	Demuestre que si $x,y\in\mathbb{H}$, entonces $x+y\in\mathbb{H}$ y $xy\in\mathbb{H}$.
%\end{exercise}

Si para $j\in\{1,2\}$ consideramos los cuaterniones 
\[
	q_j=\begin{pmatrix}
		\alpha_j & \beta_j\\
		-\overline{\beta_j} & \overline{\alpha_j}
	\end{pmatrix},
\]
y calculamos el determinante $\det(q_1q_2)=\det(q_1)\det(q_2)$, entonces
obtememos 
\begin{equation}
	\label{eq:Gauss}
	(|\alpha_1|^2+|\beta_1|^2)(|\alpha_2|^2+|\beta_2|^2)=|\alpha_1\alpha_2-\beta_1\overline{\beta_2}|^2+|\alpha_1\beta_2+\beta_1\overline{\alpha_2}|^2.
\end{equation}
Se sabe que esta identidad era conocida por Gauss y que posiblemente fue
descubierta en 1820, mucho antes del descubrimiento de los cuaterniones. 

\begin{exercise}
	Demuestre que con 
	\begin{align*}
		\alpha_1=a_1+d_1\sqrt{-1},
		&&
		\alpha_2=a_2+d_2\sqrt{-1},
		&&
		\beta_1=b_1+c_1\sqrt{-1}, 
		&&
		\beta_2=b_2+c_2\sqrt{-1},
	\end{align*}
	la identidad~\eqref{eq:Gauss} se traduce en la identidad de Euler. 
\end{exercise}

En 1835 Hamilton definió un número complejo como un par ordenado $(a,b)$ de
números reales y dio reglas para sumar y multiplicar números complejos.
Hamilton reconoció la importancia de tener bien definidas una suma y una
multiplicación, aunque es evidente que encontrar una buena multiplicación es
realmente la parte difícil del problema, la suma se hará coordenada a
coordenada. 

Hamilton, Peacock, De Morgan y Graves intentaron extender el concepto de
número. Todos tenían en mente las inclusiones
\[
	\N\subseteq\Q\subseteq\R\subseteq\C
\]
y ciertas condiciones que los sistemas numéricos debían verificar:
asociatividad, conmutatividad, propiedades distributivas, etcétera. Estas
condiciones se materializaron con la definición de cuerpo dada por Dedekind en
1871, noción que había sido dada independientemente por Galois en 1830.
Hamilton, en cambio, tenía en mente una propiedad extra: la existencia de una
norma multiplicativa, es decir una función $x\mapsto |x|$ tal que $|x|\geq0$ si
$x\ne 0$ y $|xy|=|x||y|$ para todo $x,y$.  Desde 1830 hasta 1843 Hamilton intentó en
vano encontrar un buena forma de multiplicar ternas, algo que sabemos no puede existir en
virtud del teorema de Legendre sobre la suma de tres cuadrados: no existe 
una identidad de la forma
\[
	(a^2+b^2+c^2)(d^2+e^2+f^2)=x^2+y^2+z^2
\]
pues, por ejemplo, como el número $15$ no es suma de tres cuadrados, basta
considerar las ternas $(a,b,c)=(1,1,1)$ y $(d,e,f)=(0,1,2)$. Hamilton nunca
estudió teoría de números y no conocía ni el teorema de Legendre ni la fórmula
de Euler que se utiliza para demostrar que todo número es suma de cuatro
cuadrados, no sabemos qué hubiera pasado si hubiera tenido acceso a estos
resultados.  El caso es que aquellos experimentos llevaron a Hamilton a
encontrar, en octubre de 1843, la regla fundamental que define a los
cuaterniones. Un cuaternión es un número de la forma 
\[
	a+bi+cj+dk,
\]
donde $a,b,c,d\in\R$ y la multiplicación de cuaterniones se basa en las
fórmulas 
\[
	i^2=j^2=k^2=ijk=-1.
\]
La clave detrás de los cuaterniones está en observar (y aceptar) que el producto de
cuaterniones no es conmutativo. 

La representación matricial para los cuaterniones que dimos al principio de la
sección fue encontrada por Cayley en 1858. 

En 1843 Graves le escribió a Hamilton y le comunicó que había descubierto a los
octoniones, un sistema numérico que involucraba $8$-uplas y cuyo producto no
solo no era conmutativo sino tampoco asociativo. En 1845 Cayley también
descubrió independientemente a los octoniones. En 1914 Dickson describió a los
octoniones como pares ordenados $(a,b)$ de cuaterniones con el producto dado por
\[
	(a_1,b_1)(a_2,b_2)=(a_1a_2-\overline{b_2}b_1,b_1a_1+b_1\overline{a_2}),
\]
donde 
\[
	\overline{a+bi+cj+dk}=a-bi-cj-dk
\]
denota el conjugado del número
$a+bi+cj+dk$. Es interesante observar que la misma fórmula permite obtener a
los complejos a partir de los reales y a los cuaterniones a partir de los
números complejos. 

\index{Teorema!de Frobenius}
\index{Teorema!de Weiestrass}
\index{Teorema!de Hurwitz}
Naturalmente estos descubrimientos tentaron a muchos matemáticos a buscar
$n$-uplas de números reales que junto con la suma usual de $\R^n$, una
multiplicación distributiva y una norma multiplicativa, dieran lugar a nuevos
sistemas numéricos. En 1884 Weiestrass demostró que los números complejos
resultan ser el único sistema numérico con multiplicación conmutativa.  En 1878
Frobenius demostró que los únicos sistemas numéricos cuya multplicación es
asociativa son los números reales, los complejos o los cuaterniones.  En 1898
Hurwitz demostró que si se tiene una identidad de la forma
\[
	(a_1^2+a_2^2+\cdots+a_n^2)(b_1^2+b_2^2+\cdots+b_n^2)
	=c_1^2+c_2^2+\cdots+c_n^2,
\]
donde las $c_i$ son funciones bilineales en las $a_j$ y las $b_k$, entonces
$n\in\{1,2,4,8\}$. 
El teorema de Hurwitz tiene una linda aplicación: 

\begin{quote}
Sea $V$ un espacio vectorial real con producto interno con $\dim V\geq3$.
Supongamos que existe una función $V\times V\to\R$, $(v,w)\mapsto v\times w$,
bilineal y tal que $v\times w$ es ortogonal al espacio generado por $v$ y $w$,
y además 
\[
	\|v\times w\|^2=\|v\|^2\|w\|^2-\langle v,w\rangle^2,
\]
donde $\|v\|^2=\langle v,\rangle$. Entonces $n\in\{3,7\}$. 
\end{quote}

La demostración de este resultado es bastante sencilla si disponemos del
teorema de Hurwitz, y por eso la exponemos a continuación.  Consideremos el espacio vectorial $W=V\oplus\R$
con el producto escalar dado por 
\[
	\langle (v_1,r_1),(v_2,r_2)\rangle = \langle v_1,v_2\rangle+r_1r_2.
\]
Primero observemos que
\begin{align*}
\langle v_1\times &v_2+r_1v_2+r_2v_1,v_1\times v_2+r_1v_2+r_2v_1\rangle\\
&=\|v_1\times v_2\|^2+r_1^2\|v_2\|^2+2r_1r_2\langle v_1,v_2\rangle+r_2^2\|v_1\|^2.
\end{align*}
Luego 
\begin{align*}
(\|v_1\|^2+r_1^2)&(\|v_2\|^2+r_2)\\
&= \|v_1\|^2\|v_2\|^2+r_2^2\|v_1\|^2+r_1^2\|v_2\|^2+r_1^2r_2^2\\
&=\|v_1\times v_2+r_1v_1+r_2v_2\|^2-2r_1r_2\langle v_1,v_2\rangle+\langle v_1,v_2\rangle^2+r_1^2r_2^2\\
&=\|v_1\times v_2+r_1v_1+r_2v_2\|^2+(\langle v_1,v_2\rangle-r_1r_2)^2\\
&=z_1^2+\cdots+z_{n+1}^2,
\end{align*}
donde las $z_k$ son funciones bilineales en $(v_1,r_1)$ y $(v_2,r_2)$. El
teorema de Hurwitz implica entonces que $n+1\in\{4,8\}$ y luego $n\in\{3,7\}$.

En caso en que el espacio vectorial $V$ sea de dimensión tres, el resultado nos
da el producto vectorial usual. Si, en cambio, $\dim V=7$, consideramos el
espacio vectorial 
\[
        W=\{(v,k,w):v,w\in V,k\in\R\}
\]
con el producto interno dado por
\[
        \langle (v_1,k_1,w_1),(v_2,k_2,w_2)\rangle = \langle v_1,v_2\rangle+k_1k_2+\langle w_1,w_2\rangle.
\]
Queda como ejercicio demostrar que entonces la operación 
\begin{multline*}
        (v_1,k_1,w_1)\times (v_2,k_2,w_2)\\
        =(k_1w_2-k_2w_1+v_1\times v_2-w_1\times w_2,
        \\-\langle v_1,w_2\rangle+\langle v_2,w_1\rangle, 
        k_2v_1-k_1v_2-v_1\times w_2-w_1\times v_2)
\end{multline*}
cumple las propiedades del resultado que enunciamos como aplicación del teorema
de Hurwitz. 


Para más información acerca de la historia detrás de expresiones similares a la
fórmula de Euler referimos a~\cite{MR1502549,MR1502566}.  Existen identidades
que involucran sumar 16 cuadrados pero, como sabemos por el teorema de Hurwitz,
debe abandonarse la condición de bilinealidad; una de estas identidades es la
identidad de Pfister. 


