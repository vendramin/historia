\chapter{Números y aritmética}

\index{Número!poligonal}
Para los griegos los números tenían propiedades místicas. De hecho, le dieron
mucha importancia --quizá demasiada-- a los números poligonales.  El ejemplo
más famoso de número poligonal es el de los cuadrados. El $25$ es un cuadrado
pues se representa con el siguiente objeto geométrico:
\[
\begin{array}{ccccc}
	\bullet & \bullet & \bullet & \bullet & \bullet\\
	\textcolor{orange}{\bullet} & \textcolor{orange}{\bullet} & \textcolor{orange}{\bullet} & \textcolor{orange}{\bullet} & \bullet\\
	\textcolor{green}{\bullet} & \textcolor{green}{\bullet} & \textcolor{green}{\bullet} & \textcolor{orange}{\bullet} & \bullet\\
	\textcolor{blue}\bullet & \textcolor{blue}{\bullet} & \textcolor{green}{\bullet}     & \textcolor{orange}{\bullet} & \bullet\\
	\textcolor{red}{\bullet} & \textcolor{blue}{\bullet} & \textcolor{green}{\bullet}    & \textcolor{orange}{\bullet} & \bullet
\end{array}
\]
Al observar los colores que usamos en los puntos que forman nuestro cuadrado,
vemos además que podemos escribir al $25$ como $25=1+3+5+7+9$. Esta observación
es un hecho general: la suma de los primeros impares consecutivos siempre será
un cuadrado. Nuestra notación nos permite traducir esta observación en la
identidad 
\[
	1+3+5+\cdots+(2n+1)=(n+1)^2,
\]
que fácilmente puede demostrarse por inducción. 

Similarmente podemos considerar números triangulares: 1, 3, 6, 10\dots
\begin{align*}
	\bullet
	&&
	\begin{array}{cc}
		\bullet\\
		\bullet & \bullet 
	\end{array}
	&&
	\begin{array}{ccc}
		\bullet\\
		\bullet & \bullet \\
		\bullet & \bullet & \bullet
	\end{array}
	&&
	\begin{array}{cccc}
		\bullet\\
		\bullet & \bullet \\
		\bullet & \bullet & \bullet\\
		\bullet & \bullet & \bullet & \bullet
	\end{array}
\end{align*}
Observermos que para calcular el $n$-ésimo núnero triangular $T_n$ lo que
hacemos es sumar los primeros $n$ números: 
\[
	T_n=1+2+3+\cdots+n.
\]
El siguiente gráfico nos permite deducir otra expresión para la fórmula general
que tendrá un número triangular:
\[
	\begin{array}{ccccc}
		\bullet & \textcolor{blue}{\bullet} & \textcolor{blue}{\bullet} & \textcolor{blue}{\bullet} & \textcolor{blue}{\bullet}\\
		\bullet & \bullet & \textcolor{blue}{\bullet} & \textcolor{blue}{\bullet} & \textcolor{blue}{\bullet}\\
		\bullet & \bullet & \bullet & \textcolor{blue}{\bullet} & \textcolor{blue}{\bullet} \\
		\bullet & \bullet & \bullet & \bullet & \textcolor{blue}{\bullet}
	\end{array}
\]
Como el número triangular $T_4$ verifica la ecuación $2T_4=4\times 5$, entonces
$T_4=\frac{4\times 5}{2}$. En general, este mismo argumento nos permite
demostrar que
\[
	T_n=\frac{n(n+1)}{2}
\]

\begin{exercise}
	Calcule $T_{100}$. 
\end{exercise}

Uno de los resultados más interesantes sobre números poligonales es el teorema
de Lagrange. Este teorema tiene sus orígenes en los trabajos del matemástico
griego Diofanto. Esto motivó a que en 1621 el francés Bachet mientras trabajaba
en una traducción de uno de los libros de Diofanto, conjeturara (o afirmara)
que todo número puede escribirse como suma de a lo sumo cuatro cuadrados. La
primera demostración de este teorema fue dada por Lagrange en 1770.

\begin{theorem}[Lagrange]
	\index{Teorema!de Lagrange}
	\index{Teorema!de los cuatro cuadrados}
	Todo número entero positivo es suma de (a lo sumo) cuatro cuadrados.	
\end{theorem}

\begin{proof}
	Como todo entero es producto de números primos, 
	la identidad 
	\begin{equation}
		\begin{aligned}
			\label{eq:Euler}
			(a_1^2+a_2^2+a_3^2+a_4^2)(b_1^2&+b_2^2+b_3^2+b_4^2)\\
			=(a_1 b_1 &- a_2 b_2 - a_3 b_3 - a_4 b_4)^2\\
			&+(a_1 b_2 + a_2 b_1 + a_3 b_4 - a_4 b_3)^2\\
			&+(a_1 b_3 - a_2 b_4 + a_3 b_1 + a_4 b_2)^2\\
			&+(a_1 b_4 + a_2 b_3 - a_3 b_2 + a_4 b_1)^2.
		\end{aligned}
	\end{equation}
	implica que solamente necesitamos demostrar el teorema para los números
	primos. El caso $p=2$ es trivial pues $2=1^2+1^2+0^2+0^2$. Tenemos que
	demostrar entonces el teorema para todo primo impar $p$. 

	\begin{claim}
		Existen $a,b\in\{0,\dots,\frac{p-1}{2}\}$ tales que
		$a^2+b^2+1\equiv0\bmod{p}$. 
	\end{claim}

	Para demostrar la afirmación consideramos los conjuntos 
	\begin{align*}
		A&=\{a^2:a=0,1,\dots,(p-1)/2\},\\
		B&=\{-b^2-1:b=0,1,\dots,(p-1)/2\}.
	\end{align*}
	Como los elementos de $A$ son todos distintos módulo $p$ y los elementos de
	$B$ son también todos distintos módulo $p$, tenemos que $|A|=|B|=(p+1)/2$.
	En consecuencia, $A\cap B=\emptyset$. Existen entonces $a\in A$ y $b\in B$
	tales que $0\leq a,b\leq (p-1)/2$ y $a^2=-1-b^2\bmod p$.

	\medskip
	Gracias a la afirmación anterior, sabemos que existe un entero $n\geq1$ tal
	que $a^2+b^2+1=np$. Entonces
	\[
		p\leq np=a^2+b^2+1=a^2+b^2+1^2+0^2.
	\]
	es suma de cuatro cuadrados. Sea $m$ el menor entero positivo tal que $mp$
	es suma de cuatro cuadrados, digamos
	\[
		mp=x_1^2+x_2^2+x_3^2+x_4^2.
	\]
	Entonces $1\leq m\leq n<p$. En efecto, si $n\geq p$, como $a$ y $b$ son
	$\leq(p-1)/2$, entonces $(p+1)^2\leq 2$, una contradicción. 

	Queremos demostrar que $m=1$. Supongamos
	entonces que $1<m<p$. Si $m$ es par, el número 
	\[
		\frac{m}{2}p=\left(\frac{x_1+x_2}{2}\right)^2
		+\left(\frac{x_1-x_2}{2}\right)^2
		+\left(\frac{x_3+x_4}{2}\right)^2
		+\left(\frac{x_3-x_4}{2}\right)^2
	\]
	es suma de cuatro cuadrados, una contradicción a la minimalidad de $m$. 
	
	Sean ahora $y_1,\dots,y_4$ tales que $x_i\equiv y_i\bmod m$
	y $-m/2<y_i\leq m/2$ para todo $i\in\{1,\dots,4\}$. Como
	\[
		y_1^2+\cdots+y_4^2\equiv x_1^2+\cdots x_4^2= mp\equiv 0\bmod m,
	\]
	existe un entero $r\geq1$ tal que $mr=y_1^2+\cdots+y_4^2$. 
	
	\begin{claim}
		$r\in\{1,\dots,m-1\}$. 
	\end{claim}

	Si $r=0$, entonces $y_1=y_2=y_3=y_4=0$. Luego $m$ divide a cada $x_i$ y
	entonces $m^2$ divide a cada $x_i^2$. En consecuencia, 
	\[
		m^2\mid x_1^2+\cdots+x_4^2=mp,
	\]
	una contradicción pues $m\nmid p$. Como entonces $r\geq 1$, 
	\[
		mr=y_1^2+\cdots+y_4^2\leq 4\left(\frac{m}{2}\right)^2=m^2
	\]
	y luego $r\leq m$. Para ver que $r\ne m$ basta observar que si $r=m$, 
	entonces 
	\[
		m^2=y_1^2+\cdots+y_4^2\leq (m/2)^2+(m/2)^2+(m/2)^2+(m/2)^2=m^2,
	\]
	y luego cada $y_j$ tomará el máximo valor posible, es decir
	$y_1=y_2=y_3=y_4=m/2$. En particular, $m$ es un número par, una contradicción. 

	\medskip
	La identidad de Euler nos permite escribir
	\[
		m^2rp=(mp)(mr)=(x_1^2+\cdots+x_4^2)(y_1^2+\cdots+y_4^2)=z_1^2+\cdots+z_4^2.
	\]
	Como $x_i\equiv y_i\bmod m$ para todo $i\in\{1,\dots,4\}$, entonces 
	$z_i\equiv 0\bmod m$ para todo $i\in\{1,\dots,4\}$. Para cada $i\in\{1,\dots,4\}$ sea
	$w_i=z_i/m\in\Z$. Entonces 
	\[
		rp=w_1^2+w_2^2+w_3^2+w_4^2,
	\]
	una contradicción a la minimalidad de $m$.
\end{proof}

La identidad~\eqref{eq:Euler} fue descubierta por Euler; figura en una carta
que le escribió a Goldbach el 4 de mayo de 1748. Puede demostrarse la
identidad~\eqref{eq:Euler} con a un cálculo directo. La identidad fue
redescubierta por Hamilton en 1843 en sus trabajos sobre cuaterniones. Una
manera sencilla y elegante de demostrar esta identidad se basa en que el
determinante es una multiplicación multiplicativa y que vale además la
siguiente fórmula:
\[
	(x_1^2+x_2^2+x_3^2+x_4^2)^2=\det\begin{pmatrix}
		x_1 & x_2 & x_3 & x_4\\
		-x_2 & x_1 & -x_4 & x_3\\
		-x_3 & x_4 & x_1 & -x_2\\
		-x_4 & -x_3 & x_2 & x_1
	\end{pmatrix}.
\]
Alrededor de 1840 Graves y Cayley descubieron independientemente una identidad
similar que involucra suma de ocho cuadrados. 

\begin{exercise}
	Demuestre que todo número $n>169$ puede escribirse como suma de exactamente
	cinco cuadrados.
\end{exercise}

\index{Fórmula!de Jacobi}
En 1828 Jacobi dio una fórmula sorprendente que permite calcular con exactitud
la cantidad $r_{4}(n)$ de representaciones del entero $n$ como suma de cuatro
cuadrados:
\[
	r_{4}(n)=8(2+(-1)^n)\sum_{\substack{d\mid n\\2\nmid d}}d.
\]

\index{Suma!de tres cuadrados}
\index{Teorema!de Legendre}
No todo número puede representarse como suma de tres cuadrados.  Los primeros
numeros que pueden representarse como suma de exactamente cuatro cuadrados son 
\[
	7, 15, 23, 28, 31, 39, 47, 55, 60, 63, 71\dots
\]
La sucesión de tales números lleva el código \verb+A004215+.  En 1798 Legendre
demostró que un entero positivo $n$ puede representarse como suma de tres
cuadrados si y sólo si no es de la forma $n=4^a(8b+7)$. Este resultado de
Legendre puede utilizarse para demostrar el teorema de Lagrange: Si $n$ es un
entero positivo, escribimos $n=4^{\alpha}m$, donde $m$ es un entero no
divisible por $4$. Si $m\equiv k\bmod 8$ para algún $k\in\{1,2,3,5,6\}$,
entonces $m$ es suma de tres cuadrados y el resultado queda demostrado. En caso
contrario, $m-1$ es también suma de tres cuadrados y luego $m$ es suma de
cuatro cuadrados. 

% otra demo con geometría de números?


% Escribimos $n-169=x_1^2+\cdots+x_4^2$ como suma de cuatro cuadrados, donde
% podemos suponer sin perder generalidad que $x_1\geq x_2\geq x_3\geq x_4\geq
% 0$. Si todos los $x_j$ son positivos, no hay nada para demostrar pues
% $169=13^2$. Si $x_4=0$, entonces escribimos $169=5^2+12^2$ y el resultado
% queda también demostrado. Si $x_3=x_4=0$, entonces usamos que
% $169=12^2+4^2+3^2$. Por último, si $x_2=x_3=x_4=0$, usamos que
% 169=10^2+8^2+2^2+1^2$. 

El teorema de Legendre también permite además demostrar fácilmente el siguiente
teorema de Gauss:

\begin{theorem}[Gauss]
	\index{Teorema!de Gauss}
	Todo entero positivo es suma de tres números triangulares.
\end{theorem}

\begin{proof}
	Por el teorema de Legendre, todo entero de la forma $8n+3$ puede expresarse
	como suma de tres cuadrados, digamos $8n+3=a_1^2+a_2^2+a_3^2$. Como los
	cuadrados módulo ocho son 0, 1 y 4, cada $a_i$ es un número impar. Como 
	\begin{align*}
		8n+3&=(2x+1)^2+(2y+1)^2+(2z+1)^2
		=4(x(x+1)+y(y+1)+z(z+1))+3,
	\end{align*}
	entonces tenemos que 
	\[
		n=\frac{x(x+1)}{2}+\frac{y(y+1)}{2}+\frac{z(z+1)}{2}.\qedhere
	\]
\end{proof}

Este teorema aparece en los diarios de Gauss con fecha del 10 de julio de 1796
y su demostración fue publicada en \emph{Disquisitiones Arithmeticae}.  Este
teorema de Gauss y el teorema de Lagrange sobre suma de cuatro cuadrados son
casos particulares de un resultado más general, conjeturado por Fermat en 1638
y demostrado por Cauchy en 1813.  Puede demostrarse que los números
$(m+2)$-poligonales se definen mediante la fórmula
\[
	p_m(k)=\frac{mk(k-1)}{2}+k.
\]

En el caso $m=1$ la fórmula da números triangulares pues
\[
	p_1(k)=\frac{k(k+1)}{2}.
\]
En cambio, si 
$m=2$, la fórmula da los números cuadrados pues
\[
p_2(k)=k(k-1)+k=k^2
\]

\begin{exercise}
	Demuestre que los números pentagonales son de la forma
	$\displaystyle{\frac{3k^2-k}{3}}$. 
\end{exercise}

Cauchy demostró en 1813 que para todo $m\geq3$ todo entero positivo puede
expresarse como una suma de $m+2$ números $(m+2)$-poligonales. Una demostración
breve y sencilla del teorema de Cauchy fue descubierta por Nathanson en
1987~\cite{MR866422}.

Otro resultado interesante sobre números poligonales es teorema de Euler sobre
números pentagonales. En 1750 Euler demostró que
\[
	\prod_{n=1}^\infty(1-x^n)=1+\sum_{k=1}^\infty (-1)^k\left(x^{(3k^2-k)/2}+x^{(3k^2+k)/2}\right).
\]
El teorema de Lagrange sobre cuatro cuadrados y esta fórmula de Euler son ahora
parte de la teoría de funciones theta que desarrolló Jacobi en 1830. 

%\begin{theorem}[Cauchy]
%	\index{Teorema!de Cauchy}
%	Todo entero positivo puede escribirse como suma de $n$ números
%	$n$-poligonales.
%\end{theorem}
%

\section*{El problema de Waring}

El teorema de Lagrange admite varias direcciones generalizaciones posibles. En
1770 Waring afirmó en sus \emph{Meditationes Algebraic\ae} que todo número es
suma de $\leq 4$ cuadrados, $\leq 9$ cubos, $\leq 19$ potencias cuartas, y, en
general, $\leq g(k)$ potencias $k$-ésimas, donde $g(k)$ depende únicamente de
$k$ y no del número que se está representando. Waring no tenía una demostración
sino que hizo esas afirmaciones únicamente basado en limitada evidencia
numérica. 

Si existe $s\in\N$ tal que todo número es suma de $\leq s$ potencias
$k$-ésimas, entonces también valdrá que todo número es suma de $\leq t$
potencias $k$-ésimas para todo $t\geq s$. Tiene sentido entonces definir $g(k)$
como el menor $s\in\N$ tal que 
\[
	x_1^k+\cdots+x_s^k=n
\]
admite solución en los enteros positivos para todo $n\in\N$. El siguiente
resultado fue demostrado en 1772 por Johannes Euler, hijo del famoso
matemático: 

\begin{theorem}[Euler]
	\index{Teorema!de Euler}
	$g(k)\geq 2^k+\lfloor (3/2)^k\rfloor-2$. 
\end{theorem}

\begin{proof}
	Primero observemos que para escribir al entero $n=2^k\lfloor
	(3/2)^k\rfloor-1$ como suma de potencias $k$-ésimas, necesitamos al menos
	$2^k+\lfloor (3/2)^k\rfloor-2$ sumandos. En efecto, si $q=\lfloor
	(3/2)^k\rfloor$, entonces
	\[
		n=2^kq-1\leq 2^k(3/2)^k-1<3^k.
	\]
	Si $n=x_1^k+\cdots+x_{g(k)}^k$, entonces $x_j\in\{0,1,2\}$ para todo
	$j\in\{1,\dots,g(k)\}$ y además 
	\[
		n=2^kq-1=\underbrace{(2^k+\cdots+2^k)}_{q\text{-veces}}-1=\underbrace{(2^k+\cdots+2^k)}_{(q-1)\text{-veces}}+2^k-1.
	\]
	Como además 
	\[
		2^k-1=\underbrace{1^k+\cdots+1^k}_{(2^k-1)\text{-veces}},
	\]
	se tiene la cota que buscamos.
\end{proof}

La constante 
\[
	I(k)=2^k+\lfloor (3/2)^k\rfloor-2 
\]
tiene interés en el estudio del problema de Waring. En 1853 Bretschneider
conjeturó que $g(k)=I(k)$ para todo $k\geq2$.

\begin{exercise}
	\index{Teorema!de Lagrange}
	Demuestre que $g(2)=4$. 
\end{exercise}

Entre 1909 y 1912 Wieferich y Kempner demostraron que $g(3)=I(3)=9$.  La
primera demostración de este resultado fue de Wieferich, pero contenía algunos
errores que posteriormente fueron reparados por Kempner. En 1940 Pillai
demostró que $g(6)=I(6)=73$ y en 1964 Jing--Jung Chen demostró que
$g(5)=I(5)=37$.  El caso de los bicuadrados es bastante más difícil: entre 1986
y 1992 Balasubramanian, Deshouillers y Dress demostraron que $g(4)=I(4)=19$.
Entre 1936 y 1990 varios matemáticos contribuyeron a calcular el valor exacto
de $g(k)$ para $7\leq k\leq 471000000$. En todos estos casos, $g(k)=I(k)$.
Veamos algunos de estos valores:
\begin{align*}
	&g(7)=I(7)=143,
	&&g(8)=I(8)=279,
	&&g(9)=I(9)=548.
\end{align*}
Otros valores de $g(k)$ pueden encontrarse en la sucesión \verb+A002804+.
%
%
%\begin{center}
%\begin{table}
%	\begin{tabular}{|c|ccccccccc|}
%		\hline
%		$k$ & 2 & 3 & 4 & 5 & 6 & 7 & 8 & 9 & 10\tabularnewline
%		\hline
%		$g(k)$ & 4 & 9 & 19 & 37 & 73 & 143 & 279 & 548 & 1079\tabularnewline
%		\hline
%	\end{tabular}
%	\caption{Algunos valores para $g(k)$.}
%\end{table}
%\end{center}

En general es muy difícil calcular con exactitud el valor de $g(k)$. Nos
contentaremos con mostrar algunas cotas en ciertos casos particulares, que en
realidad es lo único que se necesita si se quiere demostrar algún caso
particular del problema de Waring.

\begin{theorem}[Liouville]
	\index{Teorema!de Liouville}
	$g(4)\leq 53$.
\end{theorem}

\begin{proof}
	Es fácil demostrar que
	\begin{equation}
		\label{eq:Lucas}
		6(x_1^2+x_2^2+x_3^2+x_4^2)^2=\sum_{1\leq i<j\leq 4}(x_i+x_j)^4+\sum_{1\leq i<j\leq 4}(x_i-x_j)^4
	\end{equation}
	Dado $n\in\N$ escribimos $n=6q+r$ con $0\leq r\leq 5$. Por el teorema de Lagrange podemos
	escribir a $q$ como suma de cuatro cuadrados:
	\[
		n=6(x_1^2+x_2^2+x_3^2+x_4^2)+r=6x_1^2+6x_2^2+6x_3^2+6x_4^2+r.
	\]
	La identidad~\eqref{eq:Lucas} nos dice que todo número de la forma $6y^2$
	es suma de doce potencias cuartas, y entonces $n-r$ es suma de 48 potencias
	cuartas. Como $0\leq r\leq 5$, vemos que $r$ es suma de $\leq 5$ potencias
	cuartas, se concluye que para escribir al entero $n$ vamos a necesitamos
	$\leq 48+5=53$ potencias cuartas.
\end{proof}

\index{Identidad!de Lucas}
\index{Identidad!de Liouville}
La identidad~\eqref{eq:Lucas} fue descubierta por Lucas en 1876, la identidad
usada por Liouville para demostrar que $g(4)\leq53$ es un poco más complicada y
es la identidad que se obtiene de~\eqref{eq:Lucas} con $x_1=y_1+y_2$,
$x_2=y_1-y_2$, $x_3=y_3+y_4$, $x_4=y_3-y_4$.

No es difícil convencerse de que las identidades como la que vimos
en~\eqref{eq:Lucas} son importantes para atacar el problema de Waring. Se hace
necesario tener una mejor notación. Escribiremos entonces 
\[
	(x_1\pm x_2\pm\cdots\pm x_m)^k=\sum(x_1+\epsilon_2x_2+\cdots+\epsilon_mx_m)^k
\]
donde la suma se toma sobre todas las posibles elecciones de  
$\epsilon_j\in\{-1,1\}$ para $2\leq j\leq m$. Por ejemplo:
\[
	(x_1\pm x_2)^4=(x_1+x_2)^4+(x_1-x_2)^4.
\]
La identidad~\eqref{eq:Lucas} que usamos para demostrar que $g(4)\leq 53$ puede
escribirse entonces abreviadamente como
\[
	6(x_1^2+x_2^2+x_3^2+x_4^2)^2=\sum_{i<j}(x_i\pm x_j)^4,
\]
donde además resultará de utilidad recordar que la suma involucrada tiene
además ocho términos.

El problema de Waring comenzó a acercarse a una solución definitiva cuando 
en 1908 Hurwitz observó que la existencia de una identidad polinomial
de la forma
\[
	p(x_1^2+\cdots+x_4^2)^k=\sum_{i=1}^N p_i(a_{1i}x_1+a_{2i}x_2+a_{3i}x_3+a_{4i}x_4)^{2k}
\]
donde $p,p_1,p_2,p_3,p_4\in\N$ y $(a_{ij})\in\Z^{N\times 4}$, permite
demostrar que 
\[
	g(2k)\leq g(k)(p_1+\cdots+p_N)+p-1,
\]
de donde evidentemente se obtiene que si $g(k)$ es finito, también lo será
$g(2k)$. 

\begin{theorem}[Hurwitz]
	\index{Teorema!de Hurwitz}
	$g(8)\leq 44793$. 
\end{theorem}

\begin{proof}
	Vamos a utilizar la identidad polinomial
	\begin{equation}
	\label{eq:Hurwitz}
	\begin{aligned}
		5040(x_1^2+x_2^2+x_3^2+x_4^2)^4= \sum_{1\leq i\leq 4}6x_i^8
		&+\sum_{1\leq i<j\leq 4}60(x_i\pm x_j)^8\\
		&+ \sum_{1\leq i<j<k\leq 4}(2x_i\pm x_j\pm x_k)^{8}\\
		&+\sum_{1\leq i<j<k<l\leq 4}6(x_i\pm x_j\pm x_k\pm x_l)^{8}.
	\end{aligned}
	\end{equation}
	Observemos que la primera suma del miembro derecho tiene cuatro términos,
	la segunda tiene doce términos, la tercera tiene 48 términos y la cuarta
	tiene ocho términos.  La identidad~\eqref{eq:Hurwitz} nos dice entonces que
	todo número de la forma $5040q^4$ es suma de 
	\[
		6\times 4+12\times 60+48+8\times 6=840
	\]
	potencias octavas. Como además demostramos que todo número es suma de
	$g(4)\leq 53$ potencias cuartas, tenemos que todo $n\geq 5040$ puede
	escribirse como suma de $840\times g(4)\leq 840\times 53=44520$ potencias
	cuartas. Por otro lado, si $n<5040$, entonces siempre podremos escribir a
	$n$ como suma de $\leq 273$ potencias octavas (pues $3^8=6561>5039)$. Luego 
	$g(8)\leq 840g(4)+273\leq 44793$.
\end{proof}

\index{Teorema!de Hilbert--Waring}
En 1909 Hilbert demostró que aquellas identidades propuestas por Hurwitz
siempre existen. Con eso logró entonces obtener una demostración completa al
problema de Waring.  Muchos matemáticos simplificaron la complicada demostración
original de Hilbert: Haursdoff, Stridsberg, Hurwitz, Remak y Frobenius. En la
demostración de Hilbert no aparecen cotas explícitas para $g(k)$. En 1953
Rieger usó las ideas de Hilbert y logró demostrar que 
\[
	g(k)\leq (2k+1)^{260(k+3)^{3k+8}}.
\]

Hoy en día la demostración que dio Hilbert en 1909 no es sino apenas una
curiosidad. Es cierto que después de hacer algunas modificaciones a los
argumentos de Hilbert tales como las que hizo Rieger uno puede obtener cotas
para $g(k)$, pero mucho mejores cotas pueden obtenerse gracias a métodos
analíticos. Un método analítico muy poderoso con el que puede resolverse
el problema de Waring fue divisado por Hardy y Littlewood en 1920 y se conoce
como el \emph{método del círculo} de Hardy--Litlewood. 

\section*{Números perfectos}

\index{Número!perfecto}
Otros números que tuvieron cierto interés en épocas pasadas fueron los números
perfectos. Un entero positivo se dice \textbf{perfecto} si es igual a la suma
de sus divisores propios positivos. Veamos algunos ejemplos:
\begin{align*}
	6 &= 1+2+3,\\
	28 &= 1+2+4+7+14,\\
	496 &= 1 + 2 + 4 + 8 + 16 + 31 + 62 + 124 + 248.
\end{align*}
Otros ejemplos de números perfectos aparecen en la sucesión \verb+A000396+.  

En el libro IX Euclides demostró que si $p$ es un número primo tal que $2^p-1$
es primo, entonces $n=2^{p-1}(2^p-1)$ es un número perfecto. 

\begin{exercise}
	Demuestre el teorema de Euclides sobre números perfectos.
\end{exercise}

En el siglo XVIII
Euler demostró el recíproco del resultado de Euclides: cada número perfecto par
$n$ es de la forma $n=2^{p-1}(2^p-1)$ donde $p$ y $2^{p}-1$ son números primos. 

\begin{exercise}
	Demuestre el teorema de Euler sobre números perfectos.
\end{exercise}

\begin{exercise}
	Demuestre que un número perfecto par es siempre un número triangular. 
\end{exercise}

\begin{exercise}
	Demuestre que un número perfecto par es siempre un número binomial.
\end{exercise}

Hasta el siglo XV se conocían muy pocos números perfectos.  En 1603 el
matemático italiano Pietro Cataldi encontró los siguientes números perfectos: 
\[
	2^{16}(2^{17} - 1) = 8 589 869 056,\quad
	2^{18}(2^{19} - 1)= 137 438691 328. 
\] 

Al momento se conocen únicamente 51 números perfectos. El mayor de los
perfectos conocidos hasta el momento\footnote{Septiembre de 2019} es un número
de 49724095 dígitos:
\[
	2^{82 589 932} (2^{82 589 933} - 1)
\]
y fue encontrado en diciembre de 2018.

No se sabe si existen finitos números perfectos y no se conocen números
perfectos impares.  En 2012, Ochem y Rao demostraron que no existen números
perfectos impares de orden $<10^{1500}$. 

\section*{Números primos}

Como bien señala Weil en~\cite{MR734177}, el amor del hombre por los números es
quizá aún más antiguo que la teoría de números.  

\index{Número!primo}
Los griegos consideraban a los números primos como números que no admiten
representaciones rectangulares.  Los primeros números primos son
\[
    2, 3, 5, 7, 11, 13, 17, 19, 23, 29, 31, 37, 41, 43, 47, 53, 59, 61, 67, 71, 73, 79\dots
\]
La sucesión de números primos es \verb+A000040+. 

El teorema fundamental de la aritmética afirma que todo número puede escribirse
como producto de números primos y esa escritura es esencialmente única. Este
resultado no aparece explícitamente en los libros de Euclides, aunque hay
muchos resultados del libro VII que son casi equivalentes. Tampoco aparece en
\emph{Essai sur la Théorie des Nombres}, famoso tratado escrito por Legendre de
1798 sobre teoría de números. La primera aparición precisa junto con una
demostración rigurosa figura en el famoso \emph{Disquisitiones arithmeticae} de
Gauss de 1801. 

Euclides demostró en la proposición 20 del libro IX que existen infinitos
números primos. 

\begin{quote}
	Hay más números primos que cualquier cantidad propuesta de números primos.
\end{quote}
La demostración es bien conocida y no la reproduciremos aquí. Sí mencionaremos,
en cambio, que la demostración de Euclides revela que la existencia de una
sucesión de enteros positivos coprimos dos a dos, implicará la infinitud de los
números primos. En efecto, si $a_1,a_2,\dots$ es una sucesión de enteros
coprimos dos a dos y para cada $j$ se toma un primo $p_j$ que divide al
elemento $a_j$, entonces los $p_j$ serán todos distintos.

Recordemos que los números de Fermat son los enteros de la forma
\[
	F_n=s^{2^n}+1.
\]
En un una carta que Goldbach le escribió a Euler en julio de 1730, se demuestra
que los números de Fermat son coprimos dos a dos. Este resultado da entonces
una demostración alternativa de la infinitud de los números primos.

\begin{theorem}(Goldbach)
	\index{Teorema!de Goldbach}
	Los numeros de Fermat son coprimos dos a dos.
\end{theorem}

\begin{proof}
	Se demuestra fácilmente por inducción que $F_n-2=F_0F_1\cdots F_{n-1}$ vale
	para todo $n\in\N$. Si $m<n$, entonces $F_m$ divide a $F_n-2$. Si $p$ es un
	número primo tal que $p\mid F_n$ y $p\mid F_m$, entonces $p\mid F_n-2$ y
	luego $p\mid 2$, una contradicción pues $F_m$ es un número impar. 
\end{proof}

Veamos otra demostración muy breve y elegante, aparentemente encontrada por Hermite:

\begin{quote}
Para cada $n\geq0$ sea $q_n$ el menor número primo que divide a $n!+1$. Como
$q_n>n$, se concluye que el conjunto $\{q_n:n\geq1\}$ contiene infinitos elementos.
\end{quote}
Esta demostración es particularmente interesante. Calculemos algunos
términos de la sucesión $(q_n)$:
\[
	2,2,3,7,5,11,7,71,61,19\dots
\]
Es la sucesión \verb+A051301+. 

\begin{exercise}
	Demuestre que todo número primo pertenece a la sucesión $(q_n)$.
\end{exercise}

%Es interesante mencionar que al día de hoy no se sabe la sucesión $n!+1$
%contiene infinitos números primos. Los primos de esta forma son 
%\[
% 1, 2, 3, 4, 6, 7, 11, 12, 14\dots
%\]
%y 
%aparecen en la sucesión \verb+A088054+.

Existen muchas otras demostraciones de la infinitud de los números primos. 
Euler dio varias demostraciones distintas de este hecho. Demostró la infinitud
de los primos al observar, por ejemplo, que el producto
\[
	\prod_{p}\frac{p}{p-1}
\]
es infinito. La ``demostración'' de Euler es más o menos así: sea
\[
	x=1+\frac12+\frac13+\frac14+\cdots
\]
Calculamos 
\[
	\frac{x}{2}=\frac12+\frac14+\frac16+\cdots
\]
y restamos para obtener
\[
	\frac{x}{2}=1+\frac13+\frac15+\frac17+\frac19+\cdots
\]
Hacemos ahora algo similar: dividimos por tres y restamos par obtener:
\[
	\frac12\frac23x=1+\frac15+\frac17+\cdots=\sum_{\substack{n\geq1\\(n:6)=1}}\frac1n,
\]
donde la suma se toma sobre todos los enteros positivos coprimos con seis. Al
continuar con este procedimiento, Euler obtiene la expresión
\[
	\prod_p\frac{p}{p-1}
	=\frac{(1\cdot 2\cdot 4\cdot 6\cdot 10\cdots)}{(2\cdot 3\cdot 5\cdot 7\cdot 11\cdots)}x=1
\]
y concluye entonces que, como $x$ es infinito (pues la serie armónica diverge),
el producto $\prod p/(p-1)$ también debe ser infinito. En conclusión, existen
infinitos números primos. La demostración de Euler carece del rigor necesario,
aunque hay que mencionar que los problemas que presenta pueden arreglarse. De
hecho, entre 1875 y 1876 Kronecker arregló esta demostración de Euler y la
expuso en sus clases sobre la teoría de los números primos.

En 1737 Euler dio una demostración alternativa de la infinitud de los números
primos. Lo hizo al demostrar que la serie
\[
	\frac12+\frac13+\frac15+\frac17+\frac{1}{11}+\frac{1}{13}+\frac{1}{17}+\cdots=\sum_{p}\frac{1}{p},
\]
donde la suma se toma sobre todos los números primos, diverge. La falta de
rigor en la demostración original de Euler puede arreglarse fácilmente. Daremos
la demostración de este resultado encontrada por Clarkson en 1966:

\begin{theorem}[Euler]
	\index{Teorema!de Euler}
	La serie $\sum_{p}\frac{1}{p}$ diverge. 
\end{theorem}

\begin{proof}
	Supongamos que la serie fuera convergente. Existe entonces un entero positivo $k$ tal que 
	\[
		\sum_{m=k+1}^{\infty}\frac{1}{p_m}<\frac12.
	\]
	Sea $Q=p_1\cdots p_k$. Para cada $n\geq1$ consideramos el número $1+nQ$.
	Como ningún primo $p_j$ con $j\in\{1,\dots,k\}$ divide a $1+nQ$, los
	factores primos de $1+nQ$ están todos en el conjunto $\{p_{k+1},p_{k+2}\dots\}$. Para cada $N\geq1$, 
	\[
		\sum_{n=1}^N\frac{1}{1+nQ}\leq \sum_{t=1}^\infty\left(\sum_{m=k+1}^{\infty}\frac{1}{p_m}\right)^t
	\]
	pues la suma del miembro derecho contiene a todos los términos que aparecen
	en la suma del miembro izquierdo. Como además 
	\[
		\sum_{n=1}^N\frac{1}{1+nQ}\leq \sum_{t=1}^\infty\left(\sum_{m=k+1}^{\infty}\frac{1}{p_m}\right)^t<\sum_{t=1}^\infty\left(\frac12\right)^t<\infty,
	\]
	obtenemos la convergencia de la serie 
	\[
		\sum_{n=1}^{\infty}\frac{1}{1+nQ}\geq\sum_{n=1}^\infty\frac{1}{n(1+Q)}=\frac{1}{1+Q}\sum_{n=1}^\infty\frac{1}{n},
	\]
	una contradicción, pues sabemos que la serie $\sum_{n=1}^\infty\frac{1}{n}$ diverge.
\end{proof}

Hasta el momento, los siete mayores números primos conocidos son primos de Mersenne. 
Desde 1997 todos los primos de Mersenne descubiertos se hicieron dentro del
proyecto GIMPS. Este proyecto de cálculo distribuido fue creado por George
Woltman en 1996 y tiene como objetivo buscar números primos de Mersenne. GIMPS
es uno de los primeros proyectos de cálculo distribuido y es notablemente
exitoso: de los 50 primos de Mersenne conocidos, los últimos 17 encontrados
fueron dentro de este proyecto.  El mayor primo de Mersenne conocido hasta el
momento es 
\[
	2^{82589933}-1
\]
y tiene 24862048 dígitos; fue descubierto en diciembre de 2018. Este número es
además el mayor número primo que se conoce. 

Hoy en día los números primos son importantes no solo en matemática pura sino
en la criptografía. Muchos de los algoritmos de encriptación utilizados se
basan en el uso de números primos. 

Desde mucho tiempo atrás existe un profundo interés en conocer cómo se
distribuyen los números primos. Basados en una tabla de primos $\leq10^6$
similar a la que vemos en la tabla~\ref{tab:pi}.

\begin{table}[h!]
  \begin{center}
    \caption{Algunos valores para $\pi(x)$.}
    \label{tab:pi}
    \begin{tabular}{r|r|r|r} 
	  \hline
	  $x$ & $\pi(x)$ & $x/\log x$ & $\pi(x)/(x/\log x)$\\
      \hline
	  $10$ & 4 & 4,3 & 0,93\\
	  $10^2$ & 25 & 21,5 & 1,15\\
	  $10^3$ & 168 & 144,9 & 1,16\\
	  $10^4$ & 1229 & 1086 & 1,11\\
	  $10^5$ & 9592 & 8686 & 1,10\\
	  $10^6$ & 78498& 72464 & 1,08\\
	  \hline
    \end{tabular}
  \end{center}
\end{table}

Gauss y Legendre conjeturaron independientemente el teorema del número primo.
La fórmula de Legendre para aproximar a la cantidad $\pi(x)$ de primos $\leq x$
es la siguiente:
\[
	\pi(x)\sim \frac{x}{\log x},
\]
que quiere decir que 
\[
	\lim_{n\to\infty} \frac{\pi(x)}{(x/\log x)}=1.
\]
La fórmula de Gauss es equivalente, aunque bastante mejor que la de Legendre: 
\[
	\pi(x)\sim\operatorname{Li}(x)=\int_2^x\frac{dt}{\log t}.
\]
Por ejemplo:
\[
	\pi(10^7)=664579,\quad
	\pi(10^7)-\frac{10^7}{\log(10^7)}=44158,\quad
	\pi(10^7)-\operatorname{Li}(10^7)=339.
\]
Veamos otro ejemplo, aún más sorprendente: 
\begin{align*}
	&\pi(10^{10})=455052511,\\
	&\pi(10^{10})-\frac{10^{10}}{\log (10^{10})}=20758029,\\
	&\pi(10^{10})-\operatorname{Li}(10^{10})=3104.
\end{align*}


Durante muchos años los matemáticos intentaron en vano demostrar estos
resultados sobre la distribución de los números primos.  En 1859, Riemann atacó
el problema con métodos analíticos y encontró una sorprendente conexión entre
los números primos y la función de variable compleja 
\[
\zeta(s)=\sum_{n\geq1}1/n^{s}. 
\]
Esta función hoy se conoce como la \textbf{función zeta de Riemann} y es de
fundamental importancia en la matemática moderna. De hecho, uno de los
problemas más importantes de la matemática, que se conoce como la hipótesis de
Riemann, es una conjetura sobre la distribución de los ceros de la función
$\zeta(s)$. 

El teorema del número primo fue demostrado independiente por de la Valle
Poussin y por Hadamard en 1896; en ambos casos, la demostración se basa en el
estudio de propiedades analíticias de la función de variable compleja
$\zeta(s)=\sum_{n\geq1}1/n^{s}$. En 1949, Selberg e Erd\"os encontraron,
independientemente, una demostración elemental del teorema del número primo.
Esta demostración elemental generó mucho revuelo en la comunidad matemática y
muchas discusiones entre Erd\"os y Selberg, ya que ambos, de alguna forma,
consideraban merecer el crédito de haber encontrado una demostración elemental
del teorema del número primo.  Cabe aclarar que cuando nos referimos a una
demostración elemental, nos referimos a una demostración que solamente utiliza
técnicas básicas, aunque en general sean demostraciones muy difíciles de
entender. Es importante recordar lo siguiente: demostración elemental no quiere
significa sencilla y fácil de entender. 

Uno de los resultados más importantes de la teoría de números durante el siglo
XIX es el siguiente teorema, demostrado por Dirichet entre 1837 y 1839: Si $k$
y $l$ son enteros coprimos, entonces existen infinitos números primos
congruentes a $l$ módulo $k$. 
Este resultado fue enunciado en forma explícita por Euler en 1785 en el caso
$l=1$ y por Legendre en 1798 en el caso general. La segunda edición del libro
de Legendre sobre teoría de números, publicada en 1808, incluye una
demostración incorrecta de este resultado sobre números primos. El error está
en creerse que un cierto lema es fácil de demostrar, tal como afirmaba
Legendre.  Dirichlet fue uno de los primeros en observar que aquel lema no era,
en realidad, fácil de demostrar. En 1858 Dupré demostró que aquel lema de
Legendre era en realidad falso.

Existe además una versión del teorema del número primo para progresiones
aritméticas: la cantidad de primos $p\leq x$ congruentes a $l$ módulo $k$ es
asintóticamente igual a 
\[
	\frac{1}{\phi(k)}\pi(x).
\]

En la teoría de los números primos podemos encontranos con muchos resultados
que son consecuencia de propiedades matemáticas muy profundas, y
simultáneamente podemos también encontrarnos que resultados que difícilmente
puedan interpretarse como algo más que una curiosidad.  En la correspondencia
entre Euler y Goldbach podemos encontrarnos con el siguiente hecho notable: El
polinomio 
\[
	x^2-x+41
\]
da un número primo para todo $x\in\{0,\dots,40\}$. Similarmente, el polinomio
\[
	x^2-79x+1601
\]
da un número primo para todo $x\in\{0,\dots,79\}$. En 1752 Goldbach demostró,
aunque con algunas deficiencias fácilmente reparables, el siguiente resultado:

\begin{theorem}[Goldbach]
	\index{Teorema!de Goldbach}
	No existe $f\in\Z[X]$ de grado $n\geq1$ tal que $f(n)$ es primo para todo
	$n\in\N$.
\end{theorem}

\begin{proof}
	Sea $f\in\Z[X]$ no constante tal que $f(n)$ es un número primo para todo
	$n\in\N$. Fijemos $x_0\in\N$. Como $f(x_0)\not\in\{-1,0,1\}$, existe
	entonces un primo $p$ tal que $p$ divide a $f(x_0)$. Como
	\[
		f(x_0+mp)\equiv f(x_0)\equiv 0\bmod p
	\]
	para todo $m\in\Z$, se tiene entonces que $f(x_0+mp)\in\{-p,p\}$ para todo
	$m\in\Z$, una contradicción.
\end{proof}

Entre las curiosidades matemáticas que involucran números primos tenemos, por
supuesto, las fórmulas explícitas. 

\begin{exercise}
	\index{Teorema!de Wilson}
	Demuestre que un entero $p$ es primo si y sólo si
	$(p-1)!\equiv -1\mod p$. 
\end{exercise}

El resultado del ejercicio anterior se conoce como el teorema de Wilson.
Aparentemente, el resultado era conocido por el matemático árabe Alhazen,
alrededor del año 1000. Waring conocía el enunciado, y también Wilson, su
estudiante. Sin embargo, ninguno de los dos pudo demostrarlo. La primera
demostración es de 1771 y fue encontrada por Lagrange. Es posible de Leibniz
conociera este resultado, pero nunca lo publicó.

% Una demostración de Gauss bien fácil utiliza el teorema de Fermat. La saqué de Davenport, página 4X. 


\begin{exercise}
	Demuestre que la función 
	\[
		n\mapsto \left\lfloor\frac{n!\bmod (n+1)}{n}\right\rfloor (n-1)+2.
	\]
	genera todos los números primos. 
\end{exercise}

Estos dos ejercicios nos muestran que existen fórmulas para encontrar números
primos, pero son absolutamente ineficientes y no pueden usarse en aplicaciones
serias. En esta misma dirección, mencionaremos un resultado demostrado por
Mills en 1946:

\begin{theorem}[Mills]
	Existe un número real $A$ tal que $\lfloor A^{3^n}\rfloor$ es un número
	primo para todo $n\in\N$.	
\end{theorem}

\begin{proof}
	Para la demostración necesitamos utilizar el siguiente resultado demostrado
	por Ingham en 1937: Si $p_n$ denota al $n$-ésimo número primo, existe una
	constante $K$ tal que 
	\[
		p_{n+1}-p_n\leq Kp_n^{5/8}.
	\]

	\begin{claim}
		Si $N>K^8$, entonces existe un primo $p$ tal que $N^3<p<(N+1)^3-1$. 
	\end{claim}

	Para demostrar esta afirmación, sea $p_n$ el mayor número primo mayor que
	$N^3$. Como $N>K^8$, entonces $N^{1/8}>K$ y luego
	\[
		N^3<p_{n+1}<p_n+Kp_n^{5/8}<N^3+KN^{15/8}<N^3+N^2<(N+1)^3-1.
	\]

	Sea $P_0$ un número primo tal que $P_0>K^8$. La afirmación anterior nos
	permite construir una sucesión $P_0,P_1,P_2\dots$ de primos tales que
	\[
		P_n^3<P_{n+1}<(P_n+1)^3-1.
	\]
	Sean 
	\[
		u_n=P_n^{3^{-n}},\quad
		v_n=(P_n+1)^{3^{-n}}.
	\]
	Vamos a demostrar que la sucesión $u_1,u_2\dots$ converge. Para eso veremos que es monótona y acotada. 
	Primero observamos que, como 
	\begin{align*}
		&u_{n+1}=P_{n+1}^{3^{-n-1}}>P_n^{3^{-n}}=u_n,
	\end{align*}
	la sucesión $(u_n)_{n\in\N}$ tiene límite pues es monótona y acotada. 
	Sea $A=\lim_{n\to\infty}u_n$. Como además 
	\begin{align*}
		&v_n=(P_n+1)^{3^{-n}}>P_n^{3^{-n}}=u_n,\\
		&v_{n+1}=(P_{n+1}+1)^{3^{-n-1}}<(P_n+1)^{3^{-n}}=v_n,
	\end{align*}
	se tiene que
	$u_n<A<v_n$ para todo $n$. Luego $P_n<A^{3^n}<P_n+1$ para todo $n$ y el
	teorema queda demostrado al tomar parte entera pues $\lfloor
	A^{3^n}\rfloor=P_n$. 
\end{proof}

El resultado anterior no tiene aplicaciones prácticas. No se conoce el valor
real de la constante $A$ del teorema; de hecho, ni siquiera se sabe si $A$ es
un número racional. Sin embargo, si se asume la veracidad de la hipótesis de
Riemann, puede demostrarse que $A$ es aproximadamente igual a
\[
	1.3063778838630806904686144926\dots
\]
Tampoco se conocen los primos que produce la función de Mills. Si se asume la
veracidad de la hipótesis de Riemann puede demostrarse que los primeros primos
producidos por la función de Mills son
\[
    2,11,1361,2521008887, 16022236204009818131831320183\dots
\]
Estos números son los elementos de la sucesión \verb+A051254+.  

Es natural preguntarse si puede demostrarse algún resultado similar al de Mills
pero sin apelar al profundo teorema de Ingham. 

\begin{exercise}
	Si existen constantes $\beta<1$, $A$ y $c$ tales que para cualesquiera dos primos
	consecutivos $p_{n-1}<p_n$ tales que $A<p_{n-1}$ se tiene que
	$p_n-p_{n-1}<cp_{n-1}^\beta$, entonces para todo $\alpha>\frac{1}{1-\beta}$
	existe un número real $\theta$ tal que 
	\[
		\lfloor \theta^{\alpha^n}\rfloor 
	\]
	es primo para todo $n\in\N$.
\end{exercise}

En 1845 Bertrand conjeturó que dado $n\geq1$ siempre existe un número primo $p$
tal que $n<p<2n$.  Si bien Bertrand pudo comprobar que la conjetura era cierta
para todo $n\leq 3\times 10^6$, no pudo demostrarlo para todo $n$. La conjetura
fue demostrada completamente por Chebyshev en 1852. Ramanujan encontró una
demostración más sencilla que la encontrada por Chebyshev en 1919.  En 1932
Erd\"os dio una demostración incluso más sencilla y completamente elemental que
utiliza solamente propiedades de los números combinatorios. 

En 1951 Wright demostró que existe $\alpha\in\R$ tal que si $g_0=\alpha$ y
$g_{n+1}=2^{g_n}$, entonces $\lfloor g_n\rfloor$ es siempre un número primo. La
demostración de este resultado puede consultarse en~\cite{MR43805}; es similar
a la demostración que hicimos del teorema de Mills, aunque el teorema de Ingham
será reemplazado por el teorema de Bertrand--Chebyshev.


\section*{La conjetura de Goldbach}

\index{Constante!de Brun}
\index{Pentium FDIV bug}
\index{Nicely, Thomas}
\index{Teorema!de Brun}
La \textbf{conjetura de Goldbach} afirma que todo entero par mayor que dos
puede expresarse como suma de dos primos. Por ejemplo:
\[
	10=3+7,\quad
	18=5+13,\quad
	20=7+13,\quad
	90=43+47.
\]
Una versión de esta conjetura aparece en una carta del 7 de junio de 1742 que
Goldbach le escribió a Euler. Aparentemente esta conjetura era conocida desde
tiempo antes; en los trabajos póstumos de Descartes puede encontrarse una
afirmación similar a la conjetura de Goldbach: no está demostrado pero todo
número puede escribirse como suma de uno, dos o tres números primos.

A pesar de los esfuerzos de muchos matemáticos, la conjetura permanece abierta. 

Gracias a cálculos computacionales se sabe que la conjetura es verdadera para
números menores que $4\times 10^{17}$. En 1937 Vinogradov demostró que todo
número impar suficientemente grande es suma de tres números primos. En 1948
Rényi demostró que existe un entero $M$ tal que cada número impar $n$
suficientemente grande es suma de un primo y otro número que tiene a lo sumo
$M$ factores primos. Si alguien pudiera demostrar que $M=1$, se tendría
entonces la veracidad de la conjetura de Goldbach. En 1966 Jing-Run demostró
que $M\leq 2$. 

Recientemente pudo demostrarse una versión débil de la conjetura de Goldbach:
Todo número impar mayor que cinco puede expresarse como suma de tres números
primos.  Esta conjetura es la \textbf{conjetura débil de Goldbach}. Es fácil
demostrar que la veracidad de la conjetura de Goldbach implicaría la veracidad
de la versión débil de la conjetura.  La conjetura débil apareció publicada sin
demostración por primera vez en 1770 en el tratado \emph{Meditationes
algebraicae}, escrito por el matemático inglés Edward Waring. Fue demostrada
por el matemático peruano Harald Helfgott en 2013. 

\section*{La infinitud de los primos gemelos}

Dos números primos $p$ y $q$ son números \textbf{primos gemelos} si $|p-q|=2$.
Los primeros ejemplos son
\[
	(3, 5), (5, 7), (11, 13), (17, 19), (29, 31), (41, 43),\dots
\]
La sucesión de primos gemelos es \verb+A077800+.  
En 2009 se demostró que los números 
\[
	65516468355\times 2^{333333}+1,
	\quad
	65516468355\times 2^{333333}-1,
\]
son primos gemelos; cada uno de estos números tiene 100355 dígitos.  Hasta
ahora no se descubrieron primos gemelos mayores. Este par de primos se encontró
dentro del proyecto de cálculo distribuido PrimeGrid, desarrollado por Rytis
Slatkevi\v{c}ius.  Se tardó más de dos años en encontrar este par de primos y
en esta tarea colaboraron 18661 personas. Según el reporte oficial, una única
computadora de escritorio hubiera tardado cerca de dos siglos en encontrar
estos números.

\begin{exercise}
	Utilice el teorema de Wilson y 
	demuestre que los enteros $n$ y $n+2$ son primos gemelos si y sólo si
	\[
		4( (n-1)!+1)\equiv -n\bmod{n(n+1)}.
	\]
\end{exercise}

Desde hace mucho tiempo se conjetura que existen infinitos primos gemelos. 

En 1919 Brun demostró que la serie
\[
	\sum\left(\frac1p+\frac1{p+2}\right)=\left(\frac13+\frac15\right)+\left(\frac15+\frac17\right)+\left(\frac1{11}+\frac1{13}\right)+\cdots,
\]
donde la suma se toma sobre todos los primos $p$ tales que $p+2$ es también un
número primo, es convergente. En 1994, mediante el cálculo de los primos
gemelos menores que $10^{14}$ el matemático estadounidense Thomas Nicely
determinó que la serie converge a 
\[
1,9021605777\dots,
\]
número que hoy se conoce
como la \textbf{constante de Brun}. Fue durante este cálculo que Nicely
descubrió un problema en los procesadores Pentium, hoy conocido como el
``Pentium FDIV bug''. La prestigiosa revista Science publicó un artículo en
1995 que describe la importancia de los problemas de teoría de números para
descubrir problemas como el que encontró Nicely.  Después del incidente Intel,
acabó cooperando con Nicely para testear los nuevos microprocesadores.

En 2013 Zhang demostró que existe un entero $N$ tal que existen infinitos pares
de primos cuya diferencia es $N$; el entero $N$ es aproximadamente igual a
$70\times10^6$.  Este trabajo tuvo gran repercusión, ya que no se conocían
resultados de este tipo desde tiempos de Euclides.  Después de la publicación
del trabajo de Zhang, Terence Tao propuso trabajar en conjunto dentro un
proyecto Polymath para intentar optimizar la cota de setenta millones
encontrada por Zhang. En 2014 un grupo de matemáticos logró reducir la cota a
246. Tao y Maynard, independientemente, lograron reducir estas cotas
notablemente si ciertas conjeturas de la teoría de números fueran verdaderas. 

\subsection*{Is massively collaborative mathematics possible?}

En 2009 el matemático inglés Tim Gowers anunció en su blog que llevaría acabo
un experimento que hasta ese momento no existía en matemática. Inspirado en
muchos de los proyectos abiertos que existen en internet, sugirió que se
trabajara en conjunto para intentar resolver algún problema muy difícil de la
matemática. El blog de Gowers es: \verb+https://gowers.wordpress.com/+ 

Gowers eligió un problema que fuera acorde para ser atacado con esta idea
colaborativa. El experimento fue un éxito: pocas semanas después del anuncio
del experimento, la comunidad matemática supo que el problema que se había
elegido estaba esencialmente resuelto. Proyectos como este son conocidos como
\textbf{proyectos Polymath}. Un proyecto Polymat es entonces un cierto tipo de
colaboración entre matemáticos con la objetivo de resolver algún problema
difícil. 

En general, los trabajos realizados bajo esta modalidad se firman con el
seudónimo D. H. J. Polymath. Sin embargo, el cuarto proyecto Polymath fue
firmado por Tao, Croot y Helfgott, porque los editores de la revista
\emph{Mathematics of Computation} pidieron que en la versión final del trabajo
figuraran los nombres reales de los autores. 



