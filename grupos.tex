\chapter{Teoría de grupos}

¿Qué son los grupos? ¿Cómo y dónde aparecen? 

\index{Teorema!de Fermat}
\index{Teorema!del binomio}
El pequeño teorema de Fermat afirma que si $p$ es un número primo y $a\in\Z$ 
entonces $a^p\equiv a\bmod p$. Nótese que si $a$ 
es
coprimo con $p$, entonces $a^{p-1}\equiv 1\bmod p$. No sabemos bien cuál fue la
demostración que encontró Fermat, pero Weil sugiere que fue la siguiente.
Primero se demuestra el resultado el en caso en que $a=2$ de la siguiente
forma: el teorema del binomio nos permite escribir
\[
	2^p=(1+1)^p=\sum_{k=0}^p\binom{p}{k}=1+1+\binom{p}{1}+\binom{p}{2}+\cdots+\binom{p}{p-1}
\]
y luego solamente debemos observar que, como los números binomiales
$\binom{p}{k}$ son todos divisibles por $p$ si $k\in\{1,\dots,p-1\}$, entonces
\[
	2^p-2=\binom{p}{1}+\cdots+\binom{p}{p-1}
\]
es también divisible por $p$. 
Ahora debemos utilizar el resultado ya probado en el caso $a=2$
y demostrar el caso $a=3$ ya que gracias al teorema del binomio podemos
escribir 
\[
	3^p=(2+1)^p=3+pm
\]
para algún $m$. Más generalmente, si sabemos que el resultado es válido para un
cierto $a$, entonces, nuevamente gracias al teorema del binomio, podemos
escribir 
\begin{align*}
	&(a+1)^p = (a+1)+pm,
\end{align*}
para algún entero $m$, y demostrar que el resultado es también válido para
$a+1$. 

Para ver que el primo $p$ siempre divide a los binomiales $\binom{p}{k}$ con $k\in\{1,\dots,p-1\}$ 
simplemente hay que utilizar la fórmula
\[
	\binom{n}{k}=\frac{n!}{k!(n-k)!}.
\]
Fermat no conocía esta fórmula, pero sí sabía que vale la fórmula
\[
	n\binom{n+m-1}{m-1}=m\binom{n+m-1}{m}
\]
y que esta fórmula alcanza para demostrar lo que se necesita.

\begin{exercise}
	\index{Teorema!multinomial}
	Enuncie el teorema multinomial. Una demostración
	probabilística puede verse en \cite{MR3453544}.
\end{exercise}

Otra forma de demostrar el pequeño teorema de Fermat es la siguiente:  
Un argumento similar al anterior pero donde el teorema del
binomio es reemplazado por el teorema multinomial, nos permitirá escribir 
\[
	(1+1+\cdots+1)^p=1^p+1^p+\cdots+1^p+pm
\]
para algún entero $m$. De aquí el teorema de Fermat se seguirá
inmediatamente. 

% esta prueba parece que es la primera que apareció y la encontró Leibniz (eso dice Davenport). 
% Ivori en 1806 dio una prueba distinta, que es la que permite extender el teorema al teorema
% de Euler. Ver Davenport pag 46

\index{Teorema!de Euler}
En 1758 Euler generalizó el resultado de Fermat de la siguiente forma: si $a\in\Z$ y
$n\in\N$, entonces $a^{\varphi(n)}\equiv 1\bmod n$, donde 
\[
	\varphi(n)=\#\{k:1\leq k\leq n-1\text{ y además }(k,n)=1\}
\]
es la función de Euler. Por ejemplo, $\varphi(5)=4$ y $\varphi(8)=4$.

\begin{exercise}
	Demuestre el teorema de Euler.
\end{exercise}

Los elementos del conjunto $\{1,2,\dots,n-1\}$ que son coprimos con $n$ forman una estructura
algebraica de cardinal $\varphi(n)$ que es cerrada por la multiplicación módulo $n$. Veamos un ejemplo concreto en
el caso $n=5$. La tabla de multiplicación viene dada por 
\begin{table}
\begin{center}
	\begin{tabular}{c|cccc}
	  & 1 & 2 & 3 & 4 \\
	  \hline
	1 & 1 & 2 & 3 & 4 \\
	2 & 2 & 4 & 1 & 3 \\
	3 & 3 & 1 & 4 & 2 \\
	4 & 4 & 3 & 2 & 1
\end{tabular}
\end{center}
\end{table}

Esta tabla tiene propiedades muy interesantes: a) es cerrada por
multiplicación, b) la multiplicación es asociativa, c) existe un elemento
neutro, y por último d) cada elemento admite un inverso multiplicativo:
\[
	1^{-1}=1,\quad
	2^{-1}=3,\quad
	3^{-1}=2,\quad
	4^{-1}=4.
\]

En lenguaje moderno, simplemente diremos que el conjunto de enteros módulo $5$
forma un grupo con la multiplicación.

\begin{exercise}
	Construya la tabla de multiplicación de enteros impares módulo 8.
\end{exercise}

La teoría de grupos no fue concebida originalmente como una generalización de
la aritmética módulo $n$ sino como herramienta fundamental para resolver el
problema de encontrar soluciones de ecuaciones. El problema era intentar
resolver la ecuación
\[
	a_5X^5+a_4X^4+\cdots+a_1X+a_0=0
\]
mediante fórmulas que involucraran radicales. 
%Por ejemplo: ¿es posible
%encontrar una solución para la ecuación\dots

Lagrange entendió que era importante estudiar el grupo generado por las
permutaciones de las raíces. Así surgieron los primeros 
grupos de permutaciones. 
El trabajo de Lagrange motivó los trabajos de Ruffini, Abel y Galois. 
Euler y Gauss sin saberlo habían considerado grupos conmutativos 
al estudiar propiedades de los enteros módulo $n$. 

La palabra
\emph{grupo} fue concebida por Galois. Para él, un \emph{grupo} era un
conjunto de permutaciones (o biyecciones) cerrado por composición. 

Expliquemos qué es un \emph{grupo de permutaciones}. Hagamos un caso sencillo, el de las funciones 
$\{1,2,3\}\to\{1,2,3\}$ que son biyectivas. El conjunto $\Sym_3$ de funciones
biyectivas $\{1,2,3\}\to\{1,2,3\}$ forma un grupo con la composición de funciones. Podemos escribir
a los elementos de $\Sym_3$ como
\[
	\Sym_3=\{\id,(12),(13),(23),(123),(132)\}
\]

El símbolo $(12)$ representa a la función tal que $1\mapsto 2$, $2\mapsto 1$ y $3\mapsto 3$. Similarmente, $(123)$ es la función tal que $1\mapsto 2$, $2\mapsto 3$ y $3\mapsto 1$. 

% Escribir como matrices de $3\times 3$. 
% Por ejemplo
% $(123)$ puede representarse por la matriz
% \[
% 	P=\begin{pmatrix}
% 		0 & 0 & 1\\
% 		1 & 0 & 0\\
% 		0 & 1 & 0
% 	\end{pmatrix}
% \]
% Luego $P(x,y,z)=\dots$. 

% El signo de una permutación puede verse como el determinante de esta matriz de permutación. 

Un \emph{subgrupo} de un grupo de permutaciones es subconjunto de permutaciones 
cerrado por multiplicación y 
que contiene a la identidad. Como ejemplo, vamos a calcular los subgrupos de $\Sym_3$.

\begin{example}
\label{exa:S3}
    Sabemos que 
    \[
	\Sym_3=\{\id,(12),(13),(23),(123),(132)\}
    \]
    Si $S$ es un subgrupo, entonces $\id\in S$. Dejamos como ejercicio
    verificar que los únicos subgrupos de $\Sym_3$ son
    \[
    \{\id\},\{\id,(12)\},\{\id,(13)\},\{\id,(23)\},\{\id,(123),(132)\},\Sym_3.
    \]
\end{example}

En el ejemplo anterior muestra un fenómeno interesante. Tenemos un grupo con seis
elementos y si $S$ es un subgrupo entonces $|S|$ divide a seis. Esto no es un accidente:  
el teorema de Lagrange afirma que si $H$ es un subgrupo de un grupo finito $G$,
entonces $|H|$ divide a $|G|$. 

Los primeros estudios sistemáticos sobre grupos de permutaciones se deben a Cauchy, que publicó varios trabajos entre 1815 y 1844 donde aparecen muchos resultados 
que hoy vemos en cursos de teoría de grupos. Por ejemplo, Cauchy calculó 
los subgrupos de $\Sym_3$ que vimos en el ejemplo \ref{exa:S3}. 

Los grupos abstractos fueron concebidos por Cayley en 1854, que parece se
inspiró en los trabajos abstractos de Boole. Un grupo abstracto es un conjunto
$G$ con una operación $G\times G\to G$, $(x,y)\mapsto xy$, que satisface
ciertas propiedades (asociatividad, existencia del elemento neutro 
y existencia de inverso). Si bien la idea de Cayley 
resultó fundamental, no tuvo impacto en aquel momento. Cayley volvió a trabajar
con grupos abstractos varios años más tarde, en 1878, esta vez con mucho éxito. 
Otra definición abstracta de grupo 
también apareció en en 1858, en las clases de Dedekind sobre teoría de Galois. Para conectar
la definición abstracta de grupo con los grupos de permutaciones concretos con los que 
se trabajaba, 
Cayley demostró que si $G$ es un grupo finito, entonces
$G$ es esencialmente un subgrupo de algún $\Sym_n$. 

En 1882 Von Dyck estudió grupos abstractos (no necesariamente finitos) y muchas de sus propiedades. En 1883 publicó trabajos donde estudió aplicaciones de la teoría de
grupos en otras áreas de la matemática: teoría de números, simetrías y geometría, grupos de permutaciones. Pueden entenderse estos trabajos como el primer anuncio
acerca de la importancia que iban a tener los grupos en el desarrollo de la matemática.  

Resultados sobre grupos concretos son en realidad resultados generales (el teorema de Lagrange es un claro ejemplo de este fenómeno). Hay otro ejemplo que es digno de mencionar. En 1872 Sylow
publicó una serie de teoremas sobre grupos de permutaciones, hoy los conocemos como
los \emph{teoremas de Sylow}. En 1887 Frobenius observó que, como todo grupo abstracto es un grupo de permutaciones, los teoremas de Sylow también valen para grupos abstractos. Sin embargo, fue más lejos. Publicó una demostración de los teoremas de Sylow para grupos abstractos, aunque en principio no fuera necesario. La abstracción juega acá un papel importante, esas ideas resultaron fundamentales 
para el desarrollo de la teoría de grupos. 

%No podemos hacer un estudio exhaustivo de la historia de la teoría de los
%grupos finitos, pero sí debemos mencionar algunos resultados muy importantes
%dentro del desarrollo de la matemática del siglo XX.  Mencionamos que la noción
%de resolubilidad es aquella que nos permite estudiar soluciones por radicales
%de ecuaciones. De hecho, sabemos que en general no podremos encontrar una
%fórmula para resolver la ecuación de grado $5$ por radicales pues el grupo
%$\Sym_5$ no es resoluble.  En \dots Burnside demostró que todo grupo de orden
%$p^a q^b$ es resoluble y conjeturó que todo grupo de orden impar es resoluble.
%Esto fue demostrado en \dots por Feit y Thompson. La demostración es muy
%difícil y ocupa un volumen completo del Pacific Journal of Mathematics.  Este
%teorema es uno de los pasos fundamentales hacia una clasificación de los grupos
%simples finitos.  Recientemente este teorema fue verificado formalmente
%por\dots en Coq\dots

\section*{La clasificación de grupos simples finitos}

Sin duda uno de los resultados más profundos de la matemática moderna es la
clasificación de los grupos simples finitos. Repasaremos un poco de la historia
básica de este notable teorema. El lector interesado en obtener información más
detallada puede consultar los artículos de Solomon~\cite{MR1824893,MR3854074}. 

La noción de subgrupo normal fue introducida por Galois en 1832, noción cuya
importancia fue reconocida por Jordan en su tratado sobre grupos de 1870. Este
libro posee una modesta lista con infinitos grupos simples. En un trabajo
publicado en 1892, H\"older remarcó que sería de gran interés disponer de la
clasificación de grupos simples, ya que estos grupos funcionan como pilares básicos de la teoría de grupos. 
Eso hizo que poco tiempo después Cole determinara todos los grupos simples de tamaño $\leq600$, y en particular
demostrara que $\SL_3(2)$, el grupo de matrices de $2\times 2$ con determinante
igual a uno y coeficientes en el cuerpo de dos elementos, es un grupo simple. En 1900 
Miller y Ling, casi sin herramientas, apenas utilizando los teoremas de Sylow y 
el famoso principio del palomar, 
extendieron estos trabajos y lograron construir todos los grupos
simples de orden $\leq2001$. 

Los grupos simples más fáciles de describir son los cíclicos con un número
primo de elementos. Por ejemplo
\[
	\Z/2=\{0,1\},\quad
	\Z/3=\{0,1,2\},\quad
	\Z/5=\{0,1,2,3,4\}\dots
\]
Están también los grupos simples alternados (no abelianos). El menor de estos grupos suele
denotarse $\Alt_5$, tiene orden 60 y fue descubierto por Galois en sus trabajos
sobre la resolubilidad por radicales. Tenemos además grupos simples 
finitos análogos a grupos geométricos que aparecen en la teoría de Lie. Algunos
de estos grupos eran ya conocidos por Galois y por Jordan. 

Gran parte de los análogos finitos de grupos de Lie fueron descubiertos en el
siglo XX, primero por Dickson en algunos casos particulares y luego por
Chevalley, Steinberg, Suzuki y Ree en forma más sistemática. 

Hay además cinco grupos ``esporádicos'' descubiertos por Mathieu en 1861 y 1873 que resultan ser los grupos de
las simetrías de ciertas configuraciones finitas que ahora nos resultaría un poco
complicado explicar. Los grupos de Mathieu no pertenecen a la familia de análogos finitos de los grupos que
aparecen en geometría y teoría de Lie. 

Tenemos entonces,
digámoslo vagamente, una lista de grupos simples finitos: a) los grupos
cíclicos de orden primo, b) los grupos alternados de grado $\geq 5$, c) los
grupos de tipo Lie, y por último d) los grupos de Mathieu. 

En 1955 Brauer y Fowler demostraron un teorema que permitió a los investigadores
pensar en un posible plan de acción para clasificar grupos simples. El método 
se basaría en clasificar grupos simples sabiendo de antemano algo sobre sus 
involuciones.  

En 1963 Feit y Thompson demostraron que todo grupo simple no abeliano tiene orden par, 
un resultado que había sido conjeturado por Burnside en 1906. La demostración ocupa
un volumen completo de la revista \emph{Pacific Journal of Mathematics} de casi
trescientas páginas. 

A partir de estos resultados y de las técnicas desarrolladas
por Feit y Thompson, y gracias a un trabajo bastante más experimental 
que el que se vio en otras etapas de la matemática, 
fue posible construir nuevos grupos simples, grupos que en
general hoy llevan el nombre de sus descubridores: Janko, Highman, Sims,
MacLaughlin, Suzuki, Conway, Held, Fischer, Lyons, O'Nan, Rudavalis. En muchos
casos, para demostrar la existencia de tales grupos fue necesario recurrir a
complicados cálculos computacionales. En algún momento los especialistas en
teoría de grupos encontraron evidencia de la existencia de un nuevo grupo
simple, esta vez monstruosamente grande, de orden 
\[
	2^{46}\times 3^{20}\times 5^9\times 7^6\times 11^2\times 13^3\times 17\times 19\times 23\times 29\times 31\times 41\times 47\times 59\times 71. 
\]
Es fácil entender por qué a este grupo se lo conoce como ``el monstruo'' o el 
``gigante amistoso''. Este grupo tiene orden 
\[
808\,017\,424\,794\,512\,875\,886\,459\,904\,961\,710\,757\,005\,754\,368\,000\,000\,000,
\]
es decir, el grupo tiene aproximadamente $8\times 10^{53}$ elementos.

A diferencia de lo que había pasado con los grupos mencionados en el párrafo
anterior, semejante cantidad de elementos hacía imposible siquiera pensar en
utilizar una computadora para demostrar la existencia. En 1982 Griess demostró
la existencia del monstruo al observar que podía verse como el grupo de
simetrías de un cierto espacio de dimensión 196883. 

El enunciado del teorema de la clasificación de grupos simples estaba ahí y
solo era necesario agregar 26 excepciones a las familias que mencionamos
anteriormente. Entre estas excepciones está el monstruo. En 1972 Gorenstein
propuso un programa para demostrar esta clasificación. 

Para algunos, la clasificación estaba completa en 1974. No para toda la
comunidad matemática, sin embargo. En 1976, por ejemplo, Brauer mencionó que
muchos matemáticos creen que tener completa la clasificación depende únicamente
del tiempo y pone énfasis en
que quizá se necesiten unos cinco años en completar los detalles que falten,
aunque luego sean necesarios muchos años más para poder ordenar y escribir
prolija y precisamente cada uno de los pasos necesarios de aquella demostración. 

Gracias a los trabajos de varios matemáticos tales como Aschbacher, Mason y
Griess, donde casos difíciles eran tratados exitosamente, Gorenstein declaró en
1981 que la clasificación de los grupos simples estaba completa. Los trabajos
de Aschbacher y Griess fueron sometidos a revisión y luego publicados, pero no
así el de Mason, que ocupaba unas ochocientas páginas.  En 1989 Aschbacher
observó que el trabajo de Mason estaba incompleto y que quedaban allí varios casos
sin resolverse. En 1992 Aschbacher anunció que había logrado completar con
éxito los detalles faltantes en el trabajo de Mason, aunque esto tampoco fue
publicado. En 1996 Aschbacher y Smith comenzaron a trabajar en la demostración 
de estos resultados; esto finalmente fue publicado en dos volúmenes en 2004. Gracias
de estos dos volúmenes, 
la comunidad matemática aceptó finalmente la clasificación de los
grupos simples. 

Hoy en día, la comunidad matemática considera que el 
teorema de clasificación de los grupos
simples es un hecho demostrado, aunque quizá no haya alguien que conozca todos
los pasos de la demostración. Como ya mencionamos, la demostración 
de la clasificación de grupos simples finitos involucra el trabajo de cientos de
matemáticos a lo largo de más de cien años y más de diez mil páginas de
revistas científicas. 

% En 1900 Burnside demostró que si $G$ es un grupo no abeliano y simple de orden
% impar, entonces el orden de $G$ es producto de al menos siete primos. 

% En 1893 Frobenius demostró que un grupo simple de orden libre de cuadrados
% resulta ser cíclico de orden primo. Dos años después Burnside extendió el
% resultado de Frobenius.

% En 1896 Dedekind le escribió una carta a Frobenius donde le proponía el
% problema de factorizar el determinante de una cierta matriz cuyos coeficientes
% dependían de un grupo simple. Un año después Frobenius encontró no solamente la
% solución al problema de Dedekind sino que lo hizo gracias al desarrollo de la
% teoría de caracteres. 

% En 1900 Burnside utilizó las ideas de Frobenius y pudo demostrar varios
% resultados que sugieren la no existencia de grupos simples no abelianos con una
% cantidad impar de elementos, conjetura que escribió formalmente en 1911. En
% 1904 demostró que todo grupo de orden $p^aq^b$ es resoluble. 

% En 1900 Dickson encontró nuevas familias de grupos simples y lo hizo inspirado en
% las teorías de Killing y Cartan sobre grupos de Lie. 

% Curiosamente Dickson declaró en 1920 que la teoría de grupos finitos había muerto. 

% La teoría de grupos finitos avanzó mucho durante el siglo XX gracias a los
% trabajos de muchos matemáticos. Por la relevancia que estos trabajos tienen en
% la clasificación de los grupos simples mencionaremos a Schur, Hall, Zassenhaus,
% Fitting.

% Después de la segunda guerra mundial algunos algebristas comenzaron a mostrar gran interés por el problema de la clasificación de grupos simples. Zassenhaus, por ejemplo, 
% lo consideró el problema más importante. Zassenhaus creía posible atacar esta problema utilizando los métodos geométricos de Killing y Cartan pertenecientes a la teoría de Lie. La idea
% era buena: de esa forma Chevalley construyó nuevos grupos simples. Steinberg y Ree también construyeron familias de grupos simples basados en la teoría de Lie. 

% Se habían construido varias familias de grupos simples, pero 
% se estaba lejos de una clasificación. Ni siquiera existía una estrategia posible para obtener la clasificación. 

% En su tesis de doctorado Thompson demostró una conjetura de Frobenius.

% Los resultados sobre la estructura de grupos finitos ocpaban más de mil quinientas páginas de publicaciones científicas. En 1971 Gorenstein propuso un programa para la clasificación de los grupos simples. 
% Ese programa fue refinado en 1972 y consistía en atacar 16 pasos. La comunidad no creía que la clasificación fuera posible.

% El teorema de Feit--Thompson fue publicado en 1963 y afirma que todo grupo de orden impar es resoluble. Si bien es un enunciado muy fácil de reproducir, la demostración es muy difícil y ocupa casi trescientas páginas. 
% El primer teorema donde se clasifica una familia infinita de grupos simples 
% es de Burnside y fue publicado en 1899. El último teorema necesario en la demostración rigurosa de la clasificación de grupos simples fue publicado en 2004 
% y fue demostrado por Aschbacher y Smith; son dos volúmenes que ocupan más de mil páginas. 


\section*{Formalización}

El teorema de clasificación de los grupos simples finitos y las dificultad
evidente de chequear la demostración es otro punto que sugiere nuevas presentaciones
del rigor en la matemática. 

En la matemática actual las computadores resultan una herramienta fundamental. Una razón 
simple de reconocer es la potencia de cálculo que tienen los ordenadores actuales. 

Existe una corriente que trabaja incansablemente en el desarrollo de software  
que permita verificar demostraciones matemáticas. Algunos ejemplos son 
\href{https://coq.inria.fr/}{Coq}, \href{https://hol-theorem-prover.org/}{HOL}, 
\href{https://isabelle.in.tum.de/}{Isabelle} 
y \href{https://leanprover.github.io/}{Lean}. 

Hay cierta controversia con las demostraciones hechas con computadoras. Las objeciones
son variadas. Una, por ejemplo, es la 
falta de belleza matemática en demostraciones formales hechas por computadora. Otra
objeción está basada en que una demostración formal por computadora requiere muchísimos pasos
y es, por lo tanto, muy larga e imposible de ser verificada o leída por humanos. 

Por otro lado, una verificación formal por computadora de una demostración complicada
podría ser de gran ayuda, en particular en casos como el de la clasificación de grupos
simples. En este sentido, se hicieron muchos avances. Por ejemplo, 
en se verificó formalmente el teorema de Feit--Thompson, ese que ocupa un volumen
completo de trescientas páginas. Muy posiblemente, en unos años 
se anunciará la verificación formal de la clasificación de grupos simples. El tiempo 
dirá si las verificaciones formales de teoremas pasarán a ser parte de una nueva concepción
del rigor en la matemática. 

\section*{La abstracción en la matemática}

Las digresiones que hicimos sobre teoría de grupos ilustran la importancia de la abstracción en la matemática. Veamos otro ejemplo. La igualdad $(-1)(-1)=1$ se conoce
desde la antigüedad. Sin embargo, en cierto momento la gente empezó a cuestionarse
ese tipo de identidades. En 1830 Peacock tuvo una idea muy avanzada: consideró dos tipos de manipulación algebraica y una forma de conectar esos distintos tipo de manipulación. La idea de Peacock es la siguiente: Tenemos a) el álgebra aritmética (la que todos conocemos, la que vale para números positivos), b) el álgebra simbólica y c) el \emph{principio de permanencia de formas equivalentes}, que establece que las leyes del álgebra simbólica son las del álgebra aritmética.  Veamos cómo funciona esa idea en un ejemplo concreto. La fórmula
\[
a-(b-c)=a-b+c
\]
es una ley aritmética, vale si $b>c$ y $a>b-c$. De acá se deduce que
$(-a)(-b)=ab$ sin restricciones. Para eso, 
primero observamos que 
\[
(a-b)(c-d)=ac+bd-ad-bc
\]
es una ley aritmética, vale si $a>b$ y $c>d$. Esa ley aritmética
se transforma en una ley del álgebra simbólica gracias al 
\emph{principio de permanencia de formas equivalentes}. Por eso, 
ahora tenemos la validez de la identidad 
\[
(a-b)(c-d)=ac+bd-ad-bc
\]
sin restricción. Evaluamos entonces esta ley simbólica en $a=0$ y $c=0$ 
para obtener la identidad
\[
(-b)(-d)=bd.
\]

Es importante remarcar que en los trabajos de Peacock, las reglas aritméticas
no estaban explícitamente establecidas: eso vino mucho después, con los axiomas
que definen a los anillos y a los cuerpos.  

Hoy vemos en cursos de álgebra definiciones axiomáticas de anillos
y cuerpos. La primera definición abstracta de anillo 
es de 1914, se debe a Fraenkel y es básicamente la misma que usamos
ahora. 

Todos sabemos la importancia que tiene el álgebra lineal 
dentro de la matemática moderna. Muchos resultados del álgebra
lineal ya se conocían en 1880, aunque en forma desordenada y algo caótica. 
El problema: no se conocían los espacios vectoriales. Fue Peano
que los introdujo en 1888, un trabajo que en su momento 
fue ignorado. 

Hoy en día los cursos de álgebra lineal comienzan con la resolución de sistemas lineales. 
Alrededor del año 200 a. C. en China se resolvían sistemas lineales de $3\times 3$ 
usando únicamente números y no incógnitas, idea precursora de la teoría de matrices. El estudio
moderno de los sistemas lineales comenzó mucho después, con Leibniz en 1693. Curiosamente,
en aquel momento, el estudio estaba basado en los determinantes. Sí, en la historia de la matemática primero aparecen  
los determinantes y tiempo después las matrices. En 1750 Cramer publicó fórmulas
para resolver sistemas lineales mediante el uso de determinantes. Se escribieron varios tratados sobre determinantes. Los primeros fueron escritos por Maclaurin y Vandermonde en 1772. En 1815 Cauchy  
estableció las bases de la teoría de determinantes más o menos como la conocemos hoy. Por ejemplo, demostró que 
\[
\det(AB)=(\det A)(\det B).
\]
La definición abstracta o axiomática de determinante apareció alrededor de 1860 y 
se debe a Weierstrass 
y Kronecker. En aquellos tiempos los determinantes estaban muy de moda. En el siglo XX 
dejaron de considerarse esenciales, ya que no eran necesarios para demostrar
resultados del álgebra lineal. 

Las matrices, como dijimos, concretamente, aparecieron después. Vimos que el germen 
de la idea aparece en el año 200 a. C. También la idea aparece implícitamente en los trabajos de Gauss. De hecho, en 1811 Gauss presentó tu método de eliminación, 
lo hizo para calcular la órbita de un asteroide. Sin embargo, el método
tampoco usaba matrices. Entre 1850 y 1858 Cayley estudió sistemáticamente las
matrices rectangulares. La terminología se debe a Sylvester (1850). Cayley 
demostró en 1858 el resultado que hoy conocemos como teorema de Cayley--Hamilton, lo hizo 
para matrices de $2\times 2$ y $3\times 3$. Hamilton lo hizo para $n=4$.

