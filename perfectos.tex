\chapter{Números perfectos}


La sucesión de números perfectos es \verb+A000396+.  

En el libro IX Euclides
demostró que si $p$ es un número primo tal que $2^o-1$ es primo, entonces
$n=2^{p-1}(2^p-1)$ es un número perfecto.  

En \dots Euler demostró el
recíproco: cada número perfecto par $n$ es de la forma $n=2^{p-1}(2^p-1)$ donde
$p$ y $2^{p}-1$ son números primos. 

No se sabe si existen finitos números perfectos y no se conocen números perfectos impares. 
En 2012, Ochem y Rao demostraron que no existen números perfectos impares de
orden $<10^{1500}$. 
