\chapter{Ecuaciones algebraicas}

En este capítulo nos concentraremos en algunos aspectos 
sobre la teoría de ecuaciones algebraicas. Empezaremos con
las ecuaciones lineales. En la antigua matemática china nos encontraremos con
un método para resolver sistemas ecuaciones lineales tipo 
\begin{equation}
\label{eq:sistema_lineal}
\begin{aligned}
	&a_{11}x_1+a_{12}x_2+\cdots+a_{1n}x_n=b_1,\\
	&a_{21}x_1+a_{22}x_2+\cdots+a_{2n}x_n=b_2,\\
	&\phantom{a_{11}x_1}\vdots\\
	&a_{n1}x_1+a_{n2}x_2+\cdots+a_{nn}x_n=b_n.
\end{aligned}
\end{equation}

Sorprendentemente, el método que se describe en los textos que se conservan 
de aquella época nos muestra un método de resolución 
esencialmente igual al que hoy conocemos como el \emph{método de
eliminación de Gauss}. Recordemos que este método 
nos permite reescribir el
sistema~\eqref{eq:sistema_lineal} como
\begin{equation*}
\begin{aligned}
	&a'_{11}x_1+a'_{12}x_2+\cdots+a'_{1n}x_n=b'_1,\\
	&\phantom{a'_{21}x_1+}a'_{22}x_2+\cdots+a'_{2n}x_n=b'_2,\\
	&\phantom{a'_{11}x_1+a'_{22}x_2+\cdots+a'_{2n}x_n}\vdots\\
	&\phantom{a'_{n1}x_1+a_{n2}x_2+\cdots+}a'_{nn}x_n=b'_n,
\end{aligned}
\end{equation*}
para poder despejar $x_n$ de la $n$-ésima ecuación, después despejar 
$x_{n-1}$ de la ecuación $(n-1)$-ésima, y así sucesivamente hasta 
despejar $x_1$ de la primera ecuación. 

En el siglo XII se descubrió que este método podía adaptarse para
resolver algunos sistemas polinomiales en dos o más variables. Veamos
un ejemplo concreto en el siguiente ejercicio:  

\begin{exercise}
	Resuelva 
	el sistema polinomial
	\begin{align*}
		 x^2+xy+y^2&=1\\
		 4x^2+3xy+2y^2&=3.
	\end{align*}
\end{exercise}

El problema presentado en el ejercicio anterior requiere
saber resolver ecuaciones cuadráticas. 

\section*{La ecuación cuadrática}

Cerca del año 2000 a. C. 
los babilónicos sabían cómo resolver 
sistema de ecuaciones de la forma 
\begin{align*}
	x+y&=p,\\
	xy&=q.
\end{align*}
De hecho, podemos reescribir al sistema 
como $x^2+q=px$. Las soluciones son entonces 
\[
	x=\frac{p}{2}+\sqrt{(p/2)^2-q},\quad
	y=\frac{p}{2}-\sqrt{(p/2)^2-q},
\]
siempre que ambos sean números positivos\footnote{Recordemos que los
babilónicos no admitían números negativos.}.  

Veamos un ejemplo concreto que
nos muestre cómo funciona aquel método concebido en tiempos de los babilónicos. Es interesante remarcar
que el método tal como lo presentaban los babilónicos era apenas ilustrado con
números, pues en aquellos tiempos no existía el álgebra, y no había indicaciones que 
explicaran por qué aquel procedimiento funcionaba correctamente.


\begin{example}
	Queremos resolver el sistema 
	\begin{align*}
		x+y&=13/2,\\
		xy&=15/2.
	\end{align*}
	El método para resolver este sistema es el siguiente. Primero formamos el número $(x+y)/2=13/4$ y 
	después calculamos el número 
	\[
		\left(\frac{x+y}{2}\right)^2=169/16.
	\]
	Como entonces
	\[
		\left(\frac{x+y}{2}\right)^2-xy=49/16,
	\]
	calculamos
%	\[
%		\sqrt{\left(\frac{x+y}{2}\right)^2-xy}. 
%	\]
%	Como sabemos que 
	\[
		\frac{x-y}{2}=\sqrt{\left(\frac{x+y}{2}\right)^2-xy}=7/4.
	\]
	Esto permite resolver la ecuación ya que 
	\begin{align*}
		x&=\frac{x+y}{2}+\frac{x-y}{2}, && %=\frac{13}{4}+\frac{7}{4}=5,\\
		y=\frac{x+y}{2}-\frac{x-y}{2}.%=\frac{13}{4}-\frac{7}{4}=\frac{3}{2}.
	\end{align*}
	En efecto, un simple cálculo con fracciones nos muestra que la solución 
	es 
	\[
		(x,y)=(5,3/2).
	\]
	
	Este ejemplo fue tomado del libro~\cite{MR1094813} de Boyer sobre historia
	de la matemática y está basado en el contenido de una tabla de arcilla 
	babilónica que se conserva en la universidad de Yale.  
	
	Debemos mencionar que los babilónicos
	escribían sus números en base sesenta, así que el sistema de ecuaciones que
	resolvimos bien podríamos haberlo presentado como 
	\begin{align*}
	x+y&=6|30,\\
	xy&=7|30,
\end{align*}
donde el símbolo $6|30$ denota al número $6+30\cdot 60^{-1}=13/2$ y el símbolo
$7|30$ al número $7+30\cdot 60^{-1}=15/2$. Obviamente esta no es la notación
que usaban los babilónicos.
\end{example}

Brahmagupta dio un método para resolver la ecuación cuadrática. Como no
disponía del álgebra, se vio obligado a explicar 
aquel método con palabras. El
método muestra que 
\[
x=\frac{\sqrt{4ac+b^2}-b}{2s}
\]
es la solución de la ecuación cuadrática 
\[
	ax^2+bx=c.
\]
Brahmagupta dio además otra fórmula para las raíces:
\[
	x=\frac{\sqrt{ac+(b/2)^2}-b/2}{a}.
\]
Hoy en día, nuestra habilidad para manipular expresiones algebraicas nos
permite demostrar muy fácilmente que las dos expresiones encontradas por
Brahmagupta son equivalentes.

Heath observó que la proposición 28 del libro VI de Euclides también propone un
método para resolver la ecuación cuadrática. En los libros de Euclides, 
ese método es demostrado
rigurosamente.

%\begin{example}
%	% metodo de aljarismi
%	$x^2+10x=39$. 
%\end{example}

La palabra ``álgebra'' viene del árabe. La introdujo el matemático Al-Juarismi
\footnote{La palabra ``algoritmo" proviene de una versión latinizada del nombre 
Al-Juarismi.} en el año 830 en un libro sobre la teoría de ecuaciones. 
En este libro mostró
cómo resolver ecuaciones de grado dos en una variable, algo que era conocido ya
por los babilonios.  Si bien Brahmagupta llegó mucho más lejos en
sus estudios algebraicos (en la notación y en la introducción de los
números negaritos, por ejemplo), el trabajo de Al-Juarismi tuvo mucha
repercusión. De hecho, Al-Juarismi es considerado uno de los 
fundadores del álgebra. 

\section*{Las ecuaciones de grado superior}

La solución de la ecuación cúbica, siglo XVI, fue quizá el avance más
importante que vio la matemática en occidente después de los resultados
obtenidos por la matemática griega. Sobre cómo fue que se llegó a resolver la
ecuación cúbica sabemos poco, y gran parte de lo que sabemos viene de la voz de
Cardano. Debemos recordar que durante la primera mitad del siglo XVI muchos matemáticos
mantenían sus descubrimientos en secreto para resaltar sobre sus adversarios en
los torneos y disputas donde se planteaban problemas científicos.

Aparentemente Scipione del Ferro, profesor en Bolonia, fue el primero en
resolver la ecuación cúbica de la forma $x^3+px=q$, según Tartaglia esto
sucedió en 1506 y según Cardano, en 1515. No sabemos qué método utilizó del
Ferro. 

A principio de siglo aparecieron varios problemas matemáticos donde la solución
requería resolver una ecuación cúbica. En este contexto fue que Tartaglia,
según su propia versión, encontró una regla para resolver la cúbica en 1534,
independientemente del trabajo de del Ferro. En 1535 se produjo un desafío
entre Fior, un discípulo de del Ferro, y Tartaglia.  Según sabemos, Tartaglia
resolvió en unas dos horas los treinta problemas planteados por Fior, mientras
que Fior no pudo resolver ninguno de los problemas planteados por Tartaglia.
Cardano, profesor en Milán, supo entonces acerca de Tartaglia y su notable
reputación y, en particular, intentó aprender el método con el que Tartaglia
podía resolver la ecuación cúbica. Tartaglia se resistió durante un tiempo,
hasta que en 1539 cedió ante los pedidos de Cardano: le comunicó su forma de
resolver la cúbica y le hizo jurar que no publicaría el método para resolver la
cúbica antes de que Tartaglia lo hiciera en un libro en el que estaba
trabajando desde hacía algún tiempo. En 1545 Cardano traicionó su juramento y
publicó una edición de \emph{Ars Magna} que incluye aquellos secretos que
Tartaglia le había contado.  Esto desata una gran pelea entre ambos que incluso
involucra a otros matemáticos de la época. Cardano en su libro  
agradece cierta inspiración obtenida a partir de los trabajos de Tartaglia y la
colaboración de Ferrari, que tiempo atrás había sido su estudiante. 

% nombrar a Tartagia y a Ferrari, quizá alguien más
Como podemos imaginarnos, para evitar números negativos, Cardano también se vio
obligado a estudiar varios casos distintos de la ecuación cúbica. Veamos uno en
particular para entender cómo funciona el método. Nos interesa resolver la
ecuación cúbica
\[
x^3+ax^2+bx+c=0.
\]
Vamos a realizar la sustitución $x=y-a/3$ y vamos a elegir $a$ para transformar
nuestra ecuación cúbica en una ecuación de la forma
\[
	y^3=py+q.
\]

Si ahora $y=u+v$, entonces
\[
	y^3=(u+v)^3=(u^3+v^3)+3uv(u+v).
\]

Si $y^3=py+q$, entonces 
\begin{align*}
	&3uv = p,\\
	&u^3+v^3=q.
\end{align*}
Después de eliminar la variable $v$ nos quedamos con la ecuación
\[
	u^3+\left(\frac{p}{3u}\right)^3=q,
\]
que es una ecuación cuadrática en $u^3$, con raíces
\[
	\frac{q}{2}\pm\sqrt{(q/2)^2-(p/3)^3}.
\]

%\begin{example}
%	Usamos la fórmula de Cardano para resolver $y^3=2$. 
%\end{example}

\begin{exercise}
	Resuelva la ecuación $y^3=6y+6$.	
\end{exercise}

Ferrari, un estudiante de Cardano, fue el que logró resolver la
ecuación de grado cuatro de la forma
\[
	x^4+ax^3+bx^2+cx+d=0
\]
mediante el uso de expresiones similares a las que Cardano encontró para
resolver la ecuación cúbica. Son expresiones que involucran raíces cuadradas y
cúbicas de funciones racionales en los coeficientes. 

El descubrimiento de Ferrari hizo que la comunidad matemática intentara
resolver la ecuación de grado cinco mediante el uso de fórmulas similares, aunque
prácticamente nada interesante fue posible obtenerse. 

Quizá uno de los mejores resultados fue el que obtuvo Bring en 1786 al lograr 
reducir la ecuación general de grado cinco a una expresión de la forma
\[
	x^5-x-A=0.
\]
Como el resultado de Bring fue publicado en una revista bastante desconocida,
los matemáticos tardaron unos cincuenta años en encontrar aquel resultado.
Quizá hoy reconozcamos esto como un golpe de suerte ya que, de otra forma, el
resultado de Bring posiblemente hubiera sido interpretado como una fuerte
indicación de que la ecuación de grado cinco también iba a poder resolverse
mediante el uso de expresiones que involucraran raíces de funciones racionales
en los coeficientes.

En 1799 Ruffini demostró, aunque su demostración no era correcta, que no era
posible resolver la ecuación general de grado cinco mediante el uso de fórmulas
que involucraran raíces de funciones racionales de los coeficientes. Sí, la
demostración presentada por Ruffini tenía algunos problemas y es por eso que el
crédito de aquel sorprendente resultado se lo quedó Abel en 1826.  En 1831
Galois llevó estas ideas muy lejos y logró demostrar que, en general, no puede
resolverse mediante expresiones que involucren las operaciones usuales y 
radicales. En el camino, Galois concibió una importantísima rama
de la matemática: la teoría de grupos.

Es importante destacar que no significa que la ecuación general de grado cinco
no pueda resolverse. Lo que Ruffini, Abel y Galois demostraron es que nunca
vamos a poder encontrar expresiones por radicales para la solución, es decir
que no existen fórmulas para la solución en términos de raíces de funciones
racionales de los coeficientes. ¡Hay otras formas de resolver esta ecuación!

En 1858 Hermite resolvió la ecuación general de grado cinco mediante el uso de
funciones trascendentes, similar a lo que Viéta había hecho con la ecuación
cúbica y las funciones trigonométricas. 

Es conveniente hacer aquí una pequeña digresión y contar que Decartes hizo
algunos aportes a la teoría de ecuaciones polinomiales de una variable. De
hecho, a él debemos la notación 
\[
	x^3,x^4,x^5,\dots
\]
para representar las potencias de la variable $x$. Es curioso que Decartes
utilizara esta notación solamente a partir de las potencias cúbicas y siguiera
utilizando $xx$ para representar al cuadrado de la variable $x$. 

