\chapter{Algunos temas para el proyecto final}

\pagestyle{plain}
\fancyhf{}
\fancyhead[LE,RO]{Historia de la matemática}
\fancyhead[RE,LO]{Proyecto final}
\fancyfoot[CE,CO]{\leftmark}
\fancyfoot[LE,RO]{\thepage}

%\addcontentsline{toc}{chapter}{Algunos temas para el proyecto final}

En este capítulo mencionaremos algunos temas
que pueden utilizarse para el proyecto final. 

\subsection*{Plimpton 322}

El objetivo es presentar el trabajo \cite{MR3716328} sobre la tabla Plimpton 322 y
la trigonometría sexagesimal. 

\subsection*{Eliminación Gaussiana}

Se pretende estudiar el trabajo \cite{MR2775854} sobre aspectos históricos del famoso 
método que nos permite 
resolver sistemas lineales.

\subsection*{Aproximaciones de $\pi$}

El cálculo del número $\pi$ es un tema que ocupa un lugar preponderante en la
historia de la matemática. Hay muchos matemáticos y muchas técnicas distintas
que quizá convendría exponer, pero tal tarea nos llevaría mucho tiempo y solo
podríamos hacerlo a expensas de sacrificar la presentación de otros resultados
también importantes en el desarrollo de la matemática. Para más información
sobre el cálculo de $\pi$ referimos al libro~\cite{MR0449960}. 

\subsection*{Euler y las series infinitas}

Para obtener información sobre el magistral manejo que Euler tenía con las series
infinitas referimos al artículo~\cite{MR2338363}.

\subsection*{La fórmula de Bhaskhara}

En el capítulo tres se mencionó una sorprendente aproximación 
para la función seno descubierta por Bhaskhara. 
Una demostración moderna de la fórmula puede
consultarse en los artículos~\cite{MR1108101,MR2793182}.