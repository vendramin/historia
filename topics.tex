\chapter{Algunos temas para el proyecto final}

\pagestyle{plain}
\fancyhf{}
\fancyhead[LE,RO]{Historia de la matemática}
\fancyhead[RE,LO]{Proyecto final}
\fancyfoot[CE,CO]{\leftmark}
\fancyfoot[LE,RO]{\thepage}

%\addcontentsline{toc}{chapter}{Algunos temas para el proyecto final}

En este apéndice mencionaremos algunos temas extra 
que pueden utilizarse para complementar algunos aspectos que se
vieron durante el curso y también para la preparación
del proyecto final.  

Lamentablemente, algunas de las referencias
que se mencionan a continuación 
están únicamente en inglés. Sin embargo, presentaremos además
varios posibles temas con bibliografía en castellano, en gran parte
con referencias a artículos a la revista española \emph{La Gaceta de la Real
Sociedad Matemática Española (RMSE)}. 

\subsubsection*{Plimpton 322}

En el primer capitulo, al hablar sobre ternas pitagóricas, se hizo 
referencia a una tabla de arcilla (la tabla Plimpton 322) que contiene soluciones
de la ecuación $x^2+y^2=z^2$. Distintas teorías intentan explicar cómo
fue que pudieron obtenerse los datos que aparecen en la tabla. 

Hay mucha controversia alrededor de la tabla Plimpton 322. En particular, 
hay muchas otras referencias que quizá convenga estudiar, véanse por ejemplo 
\cite{MR1917126,MR631810,MR3074274,MR1849797,MR1903149,MR3955330}.

El trabajo
\cite{MR3716328} 
de Mansfield y Wildberger esboza una explicación basada en una especie 
de trigonometría sexagesimal. 

\subsubsection*{Eliminación Gaussiana}

Para conocer detalles históricos sobre 
el método de Gauss que usamos para resolver sistemas lineales
referimos al artículo \cite{MR2775854}. 

%\subsubsection*{El método de Newton}

\subsubsection*{Aproximaciones de $\pi$}

El cálculo del número $\pi$ es un tema que ocupa un lugar preponderante en la
historia de la matemática. Hay muchos matemáticos y muchas técnicas distintas
que quizá convendría exponer, pero tal tarea nos llevaría mucho tiempo y solo
podríamos hacerlo a expensas de sacrificar la presentación de otros resultados
también importantes en el desarrollo de la matemática. Para más información
sobre el cálculo de $\pi$ referimos al libro~\cite{MR0449960}. 

El artículo 
\cite{zbMATH05356362} de 
Guillera Goyanes está escrito castellano y 
detalla algunos aspectos históricos sobre fórmulas y métodos que
permiten aproximar el valor de $\pi$. 

\subsubsection*{Euler y las series infinitas}

En el segundo capítulo vimos varios resultados que involucran series infinitas. En particular 
vimos algunas identidades sorprendentes --descubiertas por Euler-- 
donde aparece el número $\pi$. Para obtener información 
sobre el magistral manejo que Euler tenía con las series
infinitas referimos al artículo~\cite{MR2338363}.

\subsubsection*{La fórmula de Bhaskhara}

En el capítulo tres se mencionó una sorprendente aproximación 
para la función seno descubierta por Bhaskhara. 
Una demostración moderna de la fórmula de Bhaskhara puede
consultarse, por ejemplo, en los artículos~\cite{MR1108101} y \cite{MR2793182}.

\subsubsection*{Aproximación de Pad\'e}

Vimos en el capítulo tres que la fórmula de Bhaskhara 
para aproximar la función seno es similar a la fórmula que
se obtiene mediante el método de aproximación de Pad\'e. Para
más información histórica sobre distintos m\'etodos 
de aproximación referimos al libro \cite{MR1083352} 
de Claude Brezinski. 

\subsubsection*{Números poligonales}

Vimos en el cuarto capítulo el teorema de Cauchy 
sobre la descomposición de enteros como suma de números poligonales (todo número
es suma de tres cubos, cuatro cuadrados, cinco pentagonales). Una demostración
elemental (es decir, una demostración que no usa conceptos avanzados) 
sobre el teorema de Cauchy puede consultarse en \cite{MR866422}. Sin embargo,
conviene destacar que esta demostración, a pesar de ser elemental, no es
para nada sencilla. ¡Es más difícil que la demostración 
del teorema de Lagrange \ref{thm:Lagrange} de los cuatro cuadrados!

El artículo \cite{zbMATH06696438}, escrito por 
Antonio P{\'e}rez Sanz, está escrito en castellano y contiene información
histórica sobre los números poligonales. 

\subsubsection*{El teorema de Fermat}

El teorema de Fermat también ocupa un lugar importante en 
la matemática. No muchas veces uno puede encontrarse
con un libro bien escrito y atrapante (y con traducción al castellano) 
como \emph{El enigma de Fermat}, escrito por Simon Singh. 

Para
el que quiera conocer más sobre la historia del teorema
de Fermat, desde sus orígenes hasta Wiles, 
recomendamos el artículo \cite{corry_fermat}, que está escrito en castellano.  

Para obtener información sobre
los trabajos de Sophie Germain sobre el teorema de Fermat referimos a los
artículos~\cite{MR2415091,MR2735899}.

\subsubsection*{Curvas topológicas}

En el capítulo sobre los elementos de Euclides hicimos algunos
comentarios sobre cónicas y curvas algebraicas. En ese capítulo 
hicimos referencia al libro \cite{MR2975988}, que contiene
varias páginas con mucha información histórica importante. 

Si alguien 
quisiera conocer más sobre curvas planas, principalmente con énfasis 
en aspectos topológicos, 
puede consultar el artículo \cite{tf_curvas} de Juan Tarrés Feixenet, 
que además está escrito en castellano. 

\subsubsection*{Computación científica}

A lo largo del texto se hicieron muchas referencias que involucran cálculos
con computadora. Aquel que quiera profundizar sobre 
el uso de computadoras en matemática puede consultar
el artículo \cite{pd_ENIAC} escrito por Manuel Perera Domínguez. 

\subsubsection*{El problema isoperimétrico}

Muchos temas geométricos quedaron afuera de nuestra presentación. Uno de esos
temas es el \emph{problema isoperimétrico}. Si tenemos dos figuras planas
con el mismo perímetro ¿cuál es la que encierra mayor área? Para 
saber más sobre este problema 
recomendamos el artículo \cite{hp_isoperimetric} 
escrito en castellano por Pedro Jos{\'e} Herrero Pi{\~n}eyro. 

\subsubsection*{Sobre los fundamentos de la geometría}

El tema sobre la existencia de distintas geometrías es inmenso y podríamos
hacer un curso solamente sobre el quinto postulado y las geometrías no euclidianas. 
Para consultar un 
artículo sobre Gauss, sus mediciones para ``demostrar" que la tierra
no es plana y los fundamentos de la geometría
referimos a \cite{scholz_triangulos}. El artículo fue escrito originalmente 
en alemán, la traducción al castellano
es de José Ferreirós.

\subsubsection*{La ecuación de Pell}

El matemático inglés Pell nada tiene que ver con la ecuación que lleva su nombre.
Aparentemente, la confusión se le atribuye a Euler, 
que creyó que un cierto método que permite
resolver la ecuación~\eqref{eq:Pell} era de Pell, 
cuando en realidad había sido descubierto
por el matemático inglés Brouncker.  La historia de la ecuación de Pell es muy
rica e interesante. El lector interesado en más detalles puede consultar el
libro de Weil~\cite{MR734177} o el segundo volumen del tratado de
Dickson~\cite{MR0245500}.

\subsubsection*{El problema del ganado de Arquímedes}

Para obtener información sobre este interesante problema y la historia
que circula alrededor referimos a
los artículos~\cite{MR1513794,MR1875156,MR1238181,MR1614879,MR1344311}.

\subsubsection*{La matemática aplicada en Argentina}

Para los interesados en la matemática aplicada, referimos al artículo
\cite{MR2504537} 
de Pablo Jacovkis (con una errata en \cite{MR2500178}), publicado
en la \emph{Revista de la Unión Matemática Argentina}. 
El trabajo, escrito en Inglés, 
trata sobre la historia de la matemática aplicada en Argentina.  

\subsubsection*{Rey Pastor en la matemática argentina}

Una de las figuras más importantes en el desarrollo de la matemática
en Argentina en el siglo XX es Julio Rey Pastor. Para conocer
detalles sobre cómo fue que Rey Pastor influyó tanto en la matemática
de nuestro país referimos al artículo 
\cite{THAA4111} de Carlos Borches. 

En el artículo \cite{zbMATH05139177}, además, podremos conocer
más detalles sobre la historia 
de Rey Pastor
y su influencia en el desarrollo de la matemática en España antes de su viaje a 
Argentina. 

Para conocer un poco más sobre la interacción entre la matemática
dentre Argentina y España, referimos al artículo \cite{zbMATH06696384}. 

\subsubsection*{Euler y la teoría de números}

El artículo \cite{zbMATH05763410} de 
Revent{\'o}s Tarrida está escrito en castellano y nos detalla
las contribuciones de Euler en la teoría de números. 

%\subsubsection*{La matemática en España y Argentina}
%zbMATH06696384

% Método de Newton
% zbMATH05791387

% Arquímedes
% zbMATH05139173

% infinito
% zbMATH06696356

\subsubsection*{Luis Santaló en la matemática argentina}

El matemático español Luis Santaló jugó un papel fundamental
en el desarrollo de la matemática argentina. Referimos a 
los artículos 
\cite{zbMATH06696371,zbMATH06696401} para más información.

% hilbert y la fisica
% zbMATH06696376

% hypatia
% zbMATH06696423

% 
\subsubsection*{Emmy Noether y su influencia en la matemática}

Es más o menos claro que Emmy Noether es considerada la madre del álgebra abstracta. 
Noether hizo además contribuciones importantes en otras áreas de la matemática 
y en física matemática. Los artículos 
\cite{zbMATH06696362} y \cite{zbMATH06696361} sirven para 
ilustrar la enorme influencia que Noether tuvo dentro del 
desarrollo de la matemática
del siglo XX. 

\subsubsection*{El concepto de función}

El artículo \cite{zbMATH02353879} es la traducción al castellano 
realizada por J. M. Almira y D. Arcoya de un 
articulo de Luzin sobre la evolución del 
concepto de función durante los siglos XVIII, XIX y principios del XX.

\subsubsection*{Leibniz, Newton y el cálculo infinitesimal}

Ya mencionamos que las ideas fundamentales del cálculo infinitesimal 
aparecieron en los trabajos de Arquímedes. También mencionamos que
Newton y Leibniz protagonizaron una de las peleas más importantes
en la ciencia, ya que ambos se disputaban la invención del cálculo 
infinitesimal. Para conocer más detalles sobre estas historias
referimos, por ejemplo, al libro 
\cite{MR0124178} de C. Boyer. 

Para conocer más detalles sobre la disputa entre Newton y Leibniz 
puede consultarse también el artículo \cite{MR4224032}. 
