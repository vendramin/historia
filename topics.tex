\chapter{Algunos temas para el proyecto final}

\pagestyle{plain}
\fancyhf{}
\fancyhead[LE,RO]{Historia de la matemática}
\fancyhead[RE,LO]{Proyecto final}
\fancyfoot[CE,CO]{\leftmark}
\fancyfoot[LE,RO]{\thepage}

%\addcontentsline{toc}{chapter}{Algunos temas para el proyecto final}

En este apéndice mencionaremos algunos temas extra 
que pueden utilizarse para complementar algunos aspectos que se
vieron durante el curso y también para la preparación
del proyecto final. 

\subsubsection*{Plimpton 322}

En el primer capitulo, al hablar sobre ternas pitagóricas, se hizo 
referencia a una tabla de arcilla (la tabla Plimpton 322) que contiene soluciones
de la ecuación $x^2+y^2=z^2$. Distintas teorías intentan explicar cómo
fue que pudieron obtenerse los datos que aparecen en la tabla. 

Hay mucha controversia alrededor de la tabla Plimpton 322. En particular, 
hay muchas otras referencias que quizá convenga estudiar, véanse por ejemplo 
\cite{MR1917126,MR631810,MR3074274,MR1849797,MR1903149,MR3955330}.

El trabajo
\cite{MR3716328} 
de Mansfield y Wildberger esboza una explicación basada en una especie 
de trigonometría sexagesimal. 

\subsubsection*{Eliminación Gaussiana}

Para conocer detalles históricos sobre 
el método de Gauss que usamos para resolver sistemas lineales
referimos al artículo \cite{MR2775854}. 

\subsubsection*{Aproximaciones de $\pi$}

El cálculo del número $\pi$ es un tema que ocupa un lugar preponderante en la
historia de la matemática. Hay muchos matemáticos y muchas técnicas distintas
que quizá convendría exponer, pero tal tarea nos llevaría mucho tiempo y solo
podríamos hacerlo a expensas de sacrificar la presentación de otros resultados
también importantes en el desarrollo de la matemática. Para más información
sobre el cálculo de $\pi$ referimos al libro~\cite{MR0449960}. 

\subsubsection*{Euler y las series infinitas}

En el segundo capítulo vimos varios resultados que involucran series infinitas. En particular 
vimos algunas identidades sorprendentes --descubiertas por Euler-- 
donde aparece el número $\pi$. Para obtener información 
sobre el magistral manejo que Euler tenía con las series
infinitas referimos al artículo~\cite{MR2338363}.

\subsubsection*{La fórmula de Bhaskhara}

En el capítulo tres se mencionó una sorprendente aproximación 
para la función seno descubierta por Bhaskhara. 
Una demostración moderna de la fórmula de Bhaskhara puede
consultarse, por ejemplo, en los artículos~\cite{MR1108101} y \cite{MR2793182}.

\subsubsection*{Aproximación de Pad\'e}

Vimos en el capítulo tres que la fórmula de Bhaskhara 
para aproximar la función seno es similar a la fórmula que
se obtiene mediante el método de aproximación de Pad\'e. Para
más información histórica sobre distintos m\'etodos 
de aproximación referimos al libro \cite{MR1083352} 
de Claude Brezinski. 