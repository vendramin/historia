\chapter{La ecuación de Pell}

\index{Ecuación!de Pell}
En este capítulo veremos un poco de historia sobre una ecuación diofántica muy
famosa, la ecuación de Pell:
\begin{equation}
	\label{eq:Pell}
	x^2-Ny^2=1.
\end{equation}
Nos interesan las soluciones enteras, es decir soluciones $(x,y)\in\Z\times\Z$.
Hay soluciones triviales, son $(x,y)=(1,0)$ y $(x,y)=(-1,0)$. Una solución
$(x,y)$ de la ecuación~\eqref{eq:Pell} se dirá positiva si $x>0$ e $y>0$. 

\begin{exercise}
	Demuestre que si $N$ no es un cuadrado, la ecuación~\eqref{eq:Pell} solo
	tiene soluciones triviales.
\end{exercise}

Curiosamente, el matemático inglés Pell nada tiene que ver con esta ecuación.
Este error se le atribuye a Euler, que creyó que un cierto método que permite
resolver~\eqref{eq:Pell} era de Pell cuando en realidad había sido descubierto
por el matemático inglés Brouncker.  La historia de la ecuación de Pell es muy
rica e interesante. Aquí veremos solamente algunos puntos importantes y algunos
resultados básicos. El lector interesado en más detalles puede consultar el
libro de Weil~\cite{MR734177} o el segundo volumen del tratado de
Dickson~\cite{MR0245500}.

El método para resolver~\eqref{eq:Pell} de Brouncker es similar 
al que ya conocían los matemáticos indios del siglo X; Euler reformuló el
método en términos de fracciones continuas.  Se cree que Fermat tenía una
demostración de que la ecuación de Pell siempre tiene solución en los enteros
positivos; de hecho, en una carta desafió a otros matemáticos a que intentaran
al menos resolver las ecuaciones de Pell con $N=61$ y $N=109$.  En 1768
Lagrange fue el primero en demostrar que la ecuación de Pell tiene soluciones
no triviales siempre que $N$ no sea un cuadrado perfecto. 

\begin{example}
	Los números que son simultáneamente cuadrados y triangulares se
	corresponden con soluciones de la ecuación $x^2-2y^2=1$. En efecto, si $n$
	es triangular y es además un cuadrado, podemos escribir 
	\[
		\frac{k(k+1)}{2}=l^2
	\]
	para $k,l\in\N$, y esta última fórmula es equivalente a
	$(2k+1)^2-2(2l)^2=1$.
\end{example}

\begin{exercise}
Demuestre que de la misma forma puede relacionarse la ecuación de Pell
$x^2-6y^2=1$ con números que son simultáneamente triangulares y pentagonales.
\end{exercise}

Fermat sabía que la ecuación $x^2-61y^2=1$ es particularmente difícil de
resolver. La solución $(x,y)$ con el menor $y$ positivo es
\[
(x,y)=(1766319049, 226153980).
\]

Veamos cómo fue que los matemáticos indios pudieron resolver esta ecuación.  En
el año 628 Brahmagupta resolvió la ecuación de Pell para muchos $N$  pero no
para todos. En el siglo IX Jayadeva mejoró el método de Brahmagupta y concibió
el ``método cíclico''. Este método fue perfeccionado en el siglo XII por
Bhaskara II. 

El ``método cíclico'' funciona de la siguiente manera. 
Diremos que $(a,b,N)$ es una \textbf{terna de Pell}
si $a^2-Nb^2=1$. La identidad de Brahmagupta, 
\begin{equation}
	\label{eq:brahmagupta}
	(x_1^2-Ny_1^2)(x_2^2-Ny_2^2)=(x_1x_2+Ny_1y_2)^2-N(x_1y_2+x_2y_1)^2,
\end{equation}
entonces, nos permite combinar las ternas $(x_1,y_1,k_1)$ y $(x_2,y_2,k_2)$ en
una nueva terna $(x_1x_2+Ny_1y_2,x_1y_2+x_2y_1,k_1k_2)$ para todo $N$. En
particular, si $(a,b,k)$ es una terna de Pell, podemos combinarla con cualquier
$(m,1,m^2-N)$ para obtener la ternas de Pell de la forma
$(am+Nb,a+bm,k(m^2-N))$.

\begin{exercise}
	Demuestre la identidad~\ref{eq:brahmagupta} de Brahmagupta.
\end{exercise}

Sea $(a,b)$ una solución de $x^2-Ny^2=k$, donde $a$ y $b$ son coprimos y $k$ es
algún número entero cualquiera. Elegiremos este $k$ de forma que sea fácil
conseguir $a$ y $b$. Lo que dijimos antes implica que
\begin{equation}
	\label{eq:Pell_inductive}
	\left(\frac{am+Nb}{k}\right)^2-N\left(\frac{a+bm}{k}\right)^2=\frac{m^2-N}{k}
\end{equation}
para todo $m\in\N$. 
Si elegimos $m$ para que $\frac{a+bm}{k}$ sea un entero y que minimice la
expresión $m^2-N$, tenemos una nueva solución a otra ecuación de Pell. Podemos
entonces repetir este procedimiento tantas veces como sea necesario hasta
obtener una ecuación de Pell de la forma $a^2-Nb^2=1$. ¡Resolvimos entonces la ecuación original!

\begin{exercise}
	Encuentre soluciones para $x^2-61y^2=1$ con el método cíclico. 
\end{exercise}

%Veamos cómo este método nos permite resolver eficientemente la ecuación
%$x^2-61y^2=1$.  Comencemos con la ecuación $x^2-61y^2=3$. A simple vista puede
%verse que $(8,1)$ es una solución no trivial, y entonces tenemos la terna de
%Pell $(8,1,3)$. Si componemos esta terna con alguna terna de Pell de la forma
%$(m,1,m^2-61)$, obtenemos la terna de Pell $(8m+61,8+m,3(m^2-61))$. Al dividir,
%tenemos la terna de Pell
%\[
%	\left(\frac{8m+61}{3},\frac{8+m}{3},\frac{m^2-61}{3}\right).
%\]
%Tenemos que elegir $m$ para que $\frac{m^2-61}{3}\in\Z$ y que minimice
%$m^2-61$, y basta con tomar $m=7$. Así es que tenemos entonces la terna de Pell
%$(39,5,-4)$, que al combinar con $(m,1,m^2-61)$ nos permite construir la terna 
%\[
%	\left(\frac{}{},\frac{}{},\frac{}{}\right),
%\]
%que es la terna~\eqref{eq:Pell_inductive} con $a=39$, $b=5$, $k=-4$, $m=7$ y $N=61$. 
Otro método que permite resolver la ecuación de Pell es el método de Lagrange, 
que fue reformulado por Euler en términos de fracciones continuas. 
Si queremos resolver $x^2-61y^2=1$ con el método de las fracciones continuas, necesitamos 
calcular la representación del número irracional $\sqrt{61}$ como fracción
continua: 
\[
	\sqrt{61}=7+\cfrac{1}{1+\cfrac{1}{4+\cfrac{1}{3+\cdots}}}
\]

\begin{exercise}
	Calcule la fracción continua que representa a $\sqrt{61}$. ¿Cuál es su período?
	Podemos ver esta representación en la sucesión \verb+A010145+.
\end{exercise}

Veamos cómo funciona el método por fracciones continuas en el caso
\begin{equation}
	\label{eq:Pell14}
	x^2-14y^2=1.
\end{equation}
Primero calculamos la fracción continua que representa a $\sqrt{14}$, que será
periódica con período igual a cuatro:
\[
	\sqrt{14}=3+\cfrac{1}{1+\cfrac{1}{2+\cfrac{1}{1+\cfrac{1}{3+\sqrt{14}}}}}.
\]
El método explica cómo proceder en caso de que el período sea par o impar. En nuestro caso, debemos
trucar esta aproximación al final del primer período. Obtenemos así 
una buena aproximación racional para $\sqrt{14}$, 
\[
	\sqrt{61}\sim 3+\cfrac{1}{1+\cfrac{1}{2+\cfrac{1}{1}}}=\cfrac{15}{4}.
\]
Esta fracción nos da además la solución $(x_1,y_1)=(15,4)$ de la
ecuación~\eqref{eq:Pell14}.  Podemos conseguir otras soluciones al calcular
\[
	x_n+y_n\sqrt{14}=(x_1+y_1\sqrt{14})^n.
\]

Hoy en día, para resolver la ecuación de Pell $x^2-Ny^2=1$, se considera
el cuerpo $\Q(\sqrt{N})$, es decir el conjunto
de números complejos de la forma
\[
	a+b\sqrt{N}
\]
donde $a$ y $b$ son números racionales. Cada elemento $\alpha=a+b\sqrt{N}$
tiene una norma dada por $N(\alpha)=a^2-Nb^2$ y esta norma es una función
multiplicativa, es decir
\[
	N(\alpha\beta)=N(\alpha)N(\beta)
\]
para todo $\alpha,\beta\in\Q(\sqrt{N})$. Encontrar entonces soluciones a la
ecuación de Pell equivale a encontrar los elementos $\alpha=a+b\sqrt{N}$ con
$a,b\in\Z$ y tal que $N(\alpha)=1$. 

%La identidad de Brahmugupta en el caso $N=-1$ nos da una fórmula para escribir
%el producto de dos sumas de cuadrados como una suma de cuadrados. De hecho,
%esta fórmula tiene puede expresarse simplemente así
%\[
%	|z_1z_2|^2=|z_1|^2|z_2|^2,\quad z_1,z_2\in\C.
%\]

\section*{Inducción matemática}

\index{Inducción!matemática}
\index{Inducción!demostrativa}
El método de Bhaskara nos sugiere cierta conexión con la inducción matemática
que comunmente utilizamos en algunas demostraciones.  Y ya que mencionamos la
inducción matemática, este parece ser un buen momento para hacer algunos
comentarios sobre las demostraciones por inducción. 

Nos basaremos en el
artículo~\cite{MR1519060}. 

Las demostraciones por inducción aparecen
independiente en trabajos de Pascal, Fermat y Maurolycus, aunque no siempre
presente de la forma en la que hoy la conocemos. Por ejemplo, en los trabajos
de Fermat aparece bajo una técnica que hoy conocemos como ``método del descenso''. 
%la técnica de Fermat consiste básicamente
%en asumir que si una cierta propiedad es válida para $n$, entonces será válida para $n-n_1$ y luego para $n-n_1-n_2$
Un proceso similar al descenso de Fermat había sido usado por Campanus en su
demostración de la irracionalidad del número $\frac{1+\sqrt{5}}{2}$ en su
edición de los elementos de Euclides del año 1260. Ninguno de los matemáticos
que utilzaron técnicas de demostración similares a la inducción consideraron
necesario asignarle un nombre especial. Wallis es uno de los primeros
matemáticos en utilizar frases como ``proceder por inducción'', aunque usaba
una inducción incompleta. Por ejemplo, en su libro de 1656, para demostrar que
\[
	\frac{1+4+9+\cdots+n^2}{(n+1)n^2}>\frac13
\]
calcula los primeros seis casos y afirma que esos casos son suficiententes para
entender qué pasa para otros valores de $n$. Jakob Bernoulli intenta en 1686
mejorar la técnica de Wallis y considera necesario agregar el argumento que
permite pasar del caso $n$ al caso $n+1$. Bernoulli no consideró necesario
asignarle un nombre especial a este procedimiento. En un libro póstumo
publicado en 1713 demuestra el teorema del binomio por inducción y lo hace con
las mejoras que introdujo anteriormente. Durante muchos de los años siguientes
los matemáticos utilizaron las técnicas de inducción de Wallis y de Bernoulli
indistintamente. A principios del siglo XVII varios diccionarios hablan de la
inducción en matemática y mencionan el teorema del binomio como un ejemplo de
aplicación.  En un diccionario matemático publicado en 1814 Barlow da una
definición formal del proceso de inducción:
\begin{quote}
	Induction is a term used by mathematicians to denote those of any law, or
	form, is inferred cases in which the generality from observing it to have
	obtained in several cases. Such inductions, however, are very deceptive,
	and ought to be admitted with the greatest cautio
\end{quote}
En 1830, el matemático inglés Peacock considera la inducción tal como fue
concebida por Bernoulli y la llama ``inducción demostrativa''. A partir de ese
momento la comunidad matemática poco a poco comienza a aceptar la noción de
inducción dada Bernoulli y olvida la inducción incompleta de Wallis.  En 1838,
De Morgan publicó un artículo sobre la inducción y la denominó ``inducción
matemática''. Muchos autores de libros de texto popularizaron estos nombres.
El famoso tratado sobre álgebra de Chrystal, por ejemplo, utiliza ``inducción
matemática''. Con el tiempo, esta terminología ganó terreno y poco a poco la
comunidad olvidó el nombre sugerido por Peacock. 


%\section*{El teorema de unidades de Dirichlet}
%
%En 1837 Dirichlet publicó una fórmula explícita para algunos ecuaciones de Pell. Por ejemplo, 
%para $N=13$, 
%\[
%	\eta^2=x_1+y_1\sqrt{13},
%\]
%donde 
%\[
%	\eta=\frac{\sin (2\pi/13) \sin(5\pi/13)\sin(6\pi/13)}{\sin(\pi/13)\sin (3\pi/13)\sin(4\pi/13)}\in\Q(\sqrt{13}).
%\]
%Resolver la ecuación de Pell en enteros positivos puede verse como un caso
%particular del teorema de las unidades de Dirichlet, demostrado en 1846. El teorema
%en el caso particular que mencionamos describe el grupo de unidades del anillo 
%de enteros algebraicos $\Z[\sqrt{N}]$. 
%
%En 1863 Kronecker publicó una expresión en términos de funciones elípticas. \framebox{?}
%
%
%
%
