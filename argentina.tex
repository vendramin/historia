\chapter{Un artículo de Luis Santaló}

\pagestyle{plain}
\fancyhf{}
\fancyhead[LE,RO]{Historia de la matemática}
\fancyhead[RE,LO]{Un artículo de Santaló}
\fancyfoot[CE,CO]{\leftmark}
\fancyfoot[LE,RO]{\thepage}

En 1961 Luis Santaló escribió 
``La matemática en la Argentina'', un artículo 
publicado en la \emph{Revista de la
Universidad de Buenos Aires}, V Época, Año VI, Núm. 2, 1961. 
Dado que es un artículo difícil de encontrar y contiene valiosa información
sobre el desarrollo de la matemática en la Argentina, lo transcribimos a
continuación. 

\subsection*{La matemática en la Argentina}
\subsubsection*{por L. A. Santaló}

En los pueblos nuevos, tal como ocurrió en los que ahora son viejos cuando lo
fueron, la matemática se desarrolla primero como técnica. Únicamente más tarde,
en una segunda etapa, cuyo momento de aparición depende de un conjunto de
circunstancias diversas, la matemática se convierte en ciencia.

Al decir que la matemática se desarrolla como técnica, entendemos que lo hace
como instrumento para satisfacer ciertas necesidades inmediatas. En tiempos de
paz, se necesita construir caminos, acondicionar puertos y levantar edificios y
se enseñan y estudian las matemáticas que se precisan para ello. En tiempos de
guerra, se necesita fortificar plazas estratégicas o calcular trayectorias
balísticas y se cultivan las matemáticas útiles para estos fines. Sea en paz o
en guerra, si se quiere desarrollar la navegación marítima, es preciso iniciar
los estudios matemáticos correspondientes. 

Durante este período los llamados matemáticos son únicamente ``técnicos'' o
``conocedores'' de la matemática. Han aprendido su simbolismo y la manera de
aplicar a cada problema la fórmula correspondiente, pero todavía no toman parte
activa en la creación matemática. Las bibliotecas se componen de manuales,
tablas o textos, únicos libros utilizados; no existen, por no haber sentido la
necesidad de ello, las revistas en que aparecen los últimos progresos y las
nuevas investigaciones.

De repente, casi siempre por obra de una sola persona, ayudada por un ambiente
ya devenido propicio, empieza la matemática como ciencia. Se inicia la
investigación. Empiezan los cursos de matemática pura. Aparecen en las
bibliotecas las colecciones de revistas especializadas. Se inicia la
publicación de revistas propias, respondiendo al afán de dar a conocer los
resultados nuevos obtenidos. Los matemáticos pasan a ser científicos que toman
parte activa en la construcción del edificio matemático. 

No vamos a ocuparnos en esta exposición, de la historia de la matemática en la
Argentina durante el período pre-científico. El lector interesado puede verla
en las monografías de C. C. Dassen, \emph{Las matemáticas en la Argentina}
(Buenos Aires, 1924), N. Besio Moreno, \emph{Sinopsis histórica de la Facultad
de Ciencias Exactas, Físicas y Naturales de Buenos Aires de la enseñanza de la
matemática y de la física en la Argentina} (Buenos Aires, 1915), y en el
estudio más amplio de J. Babini, \emph{Historia de la ciencia argentina}
(Buenos Aires, 1949). Observaremos únicamente que la visión de los creadores de
los sucesivos centros de enseñanza de la matemática y ciencias afines
(Belgrano, Rivadavia, Sarmiento, Gutierrez) superaba el medio ambiente,
quedando siempre sus intenciones de ilustración superior limitadas a los
preámbulos de los decretos, ya que la realidad no permitía a dichos centros más
que una vida efímera y un nivel justo para las necesidades mínimas del momento.
Sin embargo, es gracias a los sucesivos esfuerzos de estas iniciativas que el
ambiente, en un principio ausente, fue creándose lentamente. 

\subsubsection*{El período 1917--1940}

La matemática como ciencia empieza en la Argentina en 1917, con la llegada al
país del profesor español Julio Rey Pastor, contratado por la universidad de
Buenos Aires. Según gráficas palabras de Babini (\emph{loc. cit.}, pág. 147):

\begin{quote}
	En la Argentina, convertir la matemática de una doncella de la ingeniería
	en una escuela de artesanía, con un ambiente de maestros y discípulos, ha
	sido la obra de estas últimas décadas que se inició con el arribo en 1917
	del eminente maestro español Julio Rey Pastor.  
\end{quote}

Ayudado en parte por el ambiente, ya en elevada etapa de desarrollo como lo
prueba el hecho mismo de su contratación y el entusiasta grupo de alumnos que
siguieron sus primeros cursos, pero debido en otra parte a su proverbial
energía, espíritu de lucha para afrontar las dificultades en vez de orillarlas
o acomodarse ante ellas y a su entusiasmo contagioso por la investigación, la
obra de Rey Pastor fue rápida, extensa y perdurable. Prácticamente toda la
matemática argentina --como la española-- durante más de veinte años llevó el
sello, más o menos profundo, de Rey Pastor. Los temas estudiados fueron los
que él inició y las ramas que dejó de lado, a pesar del amplio campo que
siempre abarcó, quedaron prácticamente desconocidas\footnote{Un análisis
	detallado de la obra de Rey Pastor se encuentra en los trabajos de E.
	Terradas, \emph{Julio Rey Pastor, hombre e investigador}, y F. Toranzos,
	\emph{Rey Pastor y la enseñanza de la matemática en la Argentina}, en las
Publicaciones del Instituto de Matemáticas de la Universidad Nacional del
Litoral, vol. 5, 1945, dedicado a Rey Pastor al cumplirse 25 años de su
actuación en la Argentina.}.

A su llegada, Rey Pastor organiza y dirige el instituto de Matemáticas de la
Facultad de Ciencias Exactas de la Universidad de Buenos Aires. Como hechos
fundamentales, inspirados por Rey Pastor, que pueden servir de referencia para
seguir la evolución de la matemática argentina citaremos: 
\begin{enumerate}
	\item[(a)] Creación del Doctorado en Ciencias Exactas; con ello, si bien las
		materias básicas siguen comunes con los estudios de ingeniería,
		empiezan a dictarse cursos superiores, la matemática moderna se
		introduce en el país y salen los primeros egresados como especialistas
		en matemática.
	\item[(b)] En 1928, con el apoyo de las autoridades de la Facultad
		(decanato de Enrique Butty), se forma una importante biblioteca
		matemática; llegan por primera vez al país, para el Instituto de
		Matemáticas, colecciones fundamentales de revistas, indispensables para
		cualquier trabajo de investigación (\emph{Journal de Crelle},
		\emph{Journal de Liouivlle}, \emph{Mathematische Annalen}, \emph{Annals
		of Mathematics}, \emph{Transactions of the American Mathematical
		Society}, \emph{Jahrbuch \"uber die Dortschritte der Mathematik}, y
		muchas otras). 
	\item [(c)] Entre 1929 y 1933 aparece el \emph{Boletín del Seminario
		Matemático Argentino} como publicación de la Facultad de Ciencias donde
		hay que buscar los primeros trabajos de investigación en un nivel
		superior. 
	\item [(d)] En 1936 se crea la Unión Matemática Argentina, entidad que
		junto con su \emph{Revista} ha seguido hasta nuestros días y de la cual
		hablaremos más adelante.
\end{enumerate}

Los cursos de Rey Pastor, sus seminarios y su siembra de temas de investigación
fructifican rápidamente y en la década 1920--30 empiezan a aparecer las
primeras contribuciones originales argentinas en el campo de la matemática, las
cuales siguen con intensidad creciente entre 1930 y 1940. A ello contribuye
también, desde su cátedra en la Facultad de Ciencias Físico--matemáticas de La
Plata el italiano Hugo Broggi, principalmente en los temas que constituyen su
especialidad: teoría de probabilidades y matemática actuarial. 

El deseo de publicar los resultados obtenidos motiva la creación de revistas
diversas. Aparte del \emph{Boletín del Seminario Matemático Argentino} ya
mencionado, comienzan las \emph{Contribuciones} de la Facultad de Ciencias
Físico--matemáticas de La Plata y aparece en 1924, por un breve período, la
\emph{Revista de Matemáticas} como órgano de una primera Sociedad Matemática
Argentina, también de corta duración. Dentro de un marco mucho más amplio,
algunos trabajos de matemáticas se publican en los \emph{Anales de la Sociedad
Científica Argentina} y también algunas monografías aisladas aparecen como
\emph{Publicaciones del Círculo Matemático del Instituto Nacional del
Profesorado Secundario} de Buenos Aires.

Repasando los volúmenes de esas publicaciones se encuentran los trabajos y
nombres de los primeros ``pioneers'' de la investigación matemática argentina:
J. Babini, J. Blaquier, E. de Cesare, A. Durañona y Vedia, A. González
Domínguez, F. La Menza, A. Sagastume Berra, E. Samatan, F. Toranzos, A.
Valeiras, J. C. Vignaux. Los temas tratados son en general de análisis
infinitesimal o algebraico: series divergentes, criterios de sumación,
relaciones entre series integrales, generalizaciones de una a más variables,
aparte de algunas contribuciones en el campo de la geometría. Predomina, sin
que ello tenga forma exclusiva, la tendencia a la generalización y al cálculo
algorítmico. 

En 1928 la matemática argentina sale de los límites del país y se hace presente
en el Congreso Internacional de Matemáticos de Bolonia, con trabajos de Babini,
Blaquier y La Menza (publicados en las Actas del Congreso).

\subsubsection*{El período de 1940 a nuestros días}

La labor tesonera y entusiasta de todos los matemáticos mencionados, la mayoría
de los cuales pasa a desempeñar en plena juventud cátedras universitarias,
rinde profundamente sus frutos, iniciándose una segunda etapa, la de 1940--60
en la cual aparecen nuevos valores, se amplía el campo de las especialidades
cultivadas, los trabajos salen del ámbito nacional para ser publicados en
revistas extranjeras de primera clase y son citados en las publicaciones de
otros matemáticos foráneos. 

El Instituto de Matemáticas de Buenos Aires, manteniéndose el foco central,
deja de ser el exclusivo en investigación matemática. Se crean nuevos centros
en el interior, no sin dificultades que a veces retrasan o anulan su
desarrollo, pero que otras son superadas, afirmándose con el tiempo con mayor
vitalidad. Se llega así al estado actual, cuyo panorama vamos a describir
pasando revista de los principales centros que pueden considerarse en
actividad. 

\emph{El Departamento de Matemáticas de Buenos Aires}. La obra de Rey Pastor se
afianza y prolonga gracias a la obra de su discípulo y continuador Alberto
González Domínguez. El Seminario del Instituto de Matemáticas es el verdadero
semillero por donde se van divulgando nuevas teorías: análisis general y
teorías de la integral (Rey Pastor), funciones analíticas y teoría matemática
de circuitos (González Domínguez). Se investiga sobre estos temas. Se forman
alumnos.

Entre 1945--48 frecuenta estos seminarios un alumno de ingeniería, cuya
vocación y aptitud por la matemática se muestra evidente y es alentada por Rey
Pastor y González Domínguez. Es el mendocino Alberto P. Calderón, que en 1949
se va a Chicago con una beca, para progresar rápidamente hasta conquistar el
cargo de profesor titular que actualmente ocupa en la misma universidad. Se
cumplen las esperanzas que expresara Dassen 1923, en el informe antes citado;
de que ``a su hora, aparezcan las lumbreras llamadas a dar lustre y
originalidad a la ciencia matemática argentina''. La obra de Calderón es ya
imperecedera. El mundo hispano--parlante tiene ya, por primera vez, un
representantes en las altas esferas de la ciencia matemática.

En 1950, al aparecer la teoría de distribuciones de L. Schwartz, ella encuentra
en González Domínguez un entusiasta propulsor. Desde entonces el Instituto de
Matemáticas de Buenos Aires es un foco de ``distribucionistas'', tanto en su
puro aspecto matemático, como en sus aplicaciones a la física matemática y en
especial a la electrodinámica cuántica. Hay que mencionar, a este respecto, los
trabajos de R. Scarfiello, alumno y colaborador de González Domínguez.

En 1955, al iniciarse la nueva época para la Universidad, el Instituto de
Matemáticas desaparece para integrarse con el Departamento de Matemáticas. Las
nuevas autoridades, el interventor J. Babibi primero y el decano R. García
después, ofrecen amplia ayuda al departamento consiguiendo fondos para poner al
día su biblioteca, creando cargos con dedicación exclusiva o semi-exclusiva
para su personal docente de todas las categorías; se incorporan al Departamento
nuevos profesores. 

Todo ello significó un rápido incremento de los estudios matemáticos: se
extendieron los campos, aumentó el número de alumnos, mejoró el ambiente para
el trabajo original. La incorporación de Mischa Cotlar se deja sentir
rápidamente. Si bien ya había pertenecido al Instituto anteriormente en calidad
de investigador y había influido con sus primeras tendencias hacia teorías
abstractas (teoría de la medida, teoría ergódica), desde 1957 se dedica al
campo de las integrales singulares que encuentra ambiente propicio por la
tradición analista de la escuela y permite rápidamente la formación de jóvenes
de seguro porvenir (Panzone, Merlo).

Otras direcciones son también cultivadas en el departamento. La historia y la
filosofía de la ciencia, por J. Babini; las probabilidades y sus ramificaciones
actuales en la investigación operativa por R. Carranza, la topología algebraica
por Gutiérrez Burzaco; la lógica matemática por G. Klimovsky; las ecuaciones
diferenciales no lineales por E. O. Roxin; la geometría diferencial por L.
Santaló; el análisis y topología general por O. Varsavsky (hasta 1959). En 1958
empieza la publicación en rotaprint de la serie ``Cursos y Seminarios de
Matemática'', gracias al esfuerzo de Cora Sadosky, profesora de álgebra, los
cuales rápidamente adquieren un alto prestigio, llegando de todas partes
pedidos de canje o adquisición.

Teniendo en cuenta la importancia del cálculo numérico y de la matemática
aplicada en muchos aspectos de la ciencia actual, el Departamento prestó
especial atención a su cultivo y desarrollo. Contó para ello con Manuel
Sadosky, hábil especialista y excelente organizador. Se fue formando un equipo
de calculistas y creando ambiente para los estudios teórico-prácticos del
cálculo numérico. En 1960 se funda la Sociedad Argentina de Cálculo. En mayo de
1961 se inaugura una computadora electrónica, marca Mercury de la casa
Ferranti, comprada gracias a un subsidio del Consejo Nacional de
Investigaciones Científicas y Técnicas, para que funcione en el nuevo edificio
de la Facultad de Ciencias Exactas y depende del Instituto del Cálculo de la
misma. El Instituto contará con P. M. Zadunaisky, especialista reconocido,
contratado al efecto.

Una computadora electrónica tiene una capacidad de trabajo extraordinaria. Por
ello se ha puesto a disposición de las distintas Universidades e Institutos del
país para que presenten sus problemas. Se han organizado, con la idea de
repetirlos periódicamente, cursillos rápidos para enseñar los métodos de
preparar los problemas para la máquina. Así, cada institución interesada podrá
tener un equipo propio de especialistas dedicados a preparar los problemas
(trazado de ``programas''), que luego la computadora resolverá en pocos
minutos.

\emph{La Unión Matemática Argentina.} Aunque no constituye un centro de
investigación matemática, todo balance de la matemática argentina de los
últimos decenios debe mencionar el papel importante desempeñado por la Unión
Matemática Argentina (U. M. A.), asociación que agrupa prácticamente a todos
los matemáticos argentinos. Se fundó en 1937 y tiene en su haber: 
\begin{enumerate}
	\item[(a)] La publicación de la \emph{Revista de la Unión Matemática Argentina},
		que más tarde (1945) fue también el órgano de la Asociación Física
		Argentina (A. F. A) y actualmente (desde 1951) es la Revista conjunta
		de ambas instituciones.  Actualmente está próxima la terminación del
		volumen XIX. La \emph{Revista} no tuvo nunca subvención oficial, hasta
		1959 en que empezó a recibir, y desde entonces anualmente, una
		subvención del Consejo Nacional de Investigaciones Científicas y
		Técnicas. 
	\item[(b)] Patrocinó varias Jornadas Matemáticas en distintos lugares del país
		(Buenos Aires, La Plata, San Luis, Rosario, Córdoba, Bahía Blanca) y
		reuniones de matemáticos, a algunas de las cuales concurrieron
		matemáticos calificados de países vecinos. 
	\item[(c)] En 1960, en ocasión del sesquicentenario de la Revolución de Mayo, la
		U. M. A. fue encargada por la Comisión Nacional al efecto de organizar
		las Sesiones Matemáticas Argentinas, que tuvieron lugar en el mes de
		septiembre en Buenos Aires y La Plata.
	\item[(d)] Es miembro del patronato de la \emph{Mathematical Reviews}.
	\item[(e)] Tiene un convenio de reciprocidad por la cual sus miembros pueden
		serlo de la American Mathematical Society de manera automática, con
		sólo el abono de la cuota, reducida a la mitad de la ordinaria. 
	\item[(f)] Está adherida a la Unión Internacional de Matemáticos.
\end{enumerate}

\emph{El Departamento de Matemáticas de La Plata}. Depende de la Facultad de
Ciencias Físico-matemáticas de la Universidad de La Plata. Aunque en esta
Facultad tienen peso predominante los estudios de ingeniería, la existencia de
un doctorado en ciencias matemáticas favoreció los estudios de matemática pura.
Aparte del campo del análisis (Durañona, Trejo), mecánica celeste (R. Cesco) y
geometría (de Cesare), una característica notable de la escuela platense fueron
los estudios de álgebra moderna llevados a cabo por Sagastume. Durante varios
años, Sagastume fue el único que se dedicó en la Argentina al álgebra moderna,
publicando trabajos y dirigiendo tesis en dicho campo.

Exponentes de la joven escuela platense han sido Ricabarra, Zarantonello y
Turrin, este último fallecido prematuramente. Ricabarra, después de trabajar
varios años con Cotlar, se dedica al problema de Suslin, publicando en 1959 la
importante contribución al mismo, en forma de libro titulado \emph{Conjuntos
ordenados y ramificados}. Actualmente orienta sus seminarios en La Plata hacia
la topología diferencial. Zarantonello se perfeccionó en los Estados Unidos con
G. Birkhoff, publicando con el mismo el libro \emph{Jets, wakes and cavities}
(1957), de gran importancia para la fundamentación rigurosa de muchos puntos de
la mecánica de fluidos. El clima obligó a Zarantonello a abandonar las
humedades del litoral bonaerense, trasladándose a Mendoza donde se encuentra
actualmente, esforzándose para crear un ambiente de investigación en su
especialidad.

En el presente, el Departamento de Matemáticas de La Plata, además de
Ricabarra, cuenta con G. Fernández para la geometría diferencial, C. Trejo para
probabilidades y análisis superior, J. Bosch para la topología algebraica y L.
Bruschi, brillante discípulo de Monteiro, Zarantonello y Cotlar, que ha probado
ya su capacidad para la investigación original. Como órgano de publicación, el
Departamento cuenta con la \emph{Revista} de la Facultad de Ciencias
Físico-matemáticas, en la cual aparecen trabajos de matemáticas, si bien no de
manera exclusiva, pues es común a todos los departamentos de la facultad.

\emph{La matemática en Tucumán}. La Universidad de Tucumán ha mostrado desde
1940 especial interés por la matemática. En esta fecha se contrató al italiano
A. Terracini, que junto con el físico F. Cernuschi funda la \emph{Revista de
Matemáticas y Física teórica} cuyo volumen XIII acaba de aparecer (1961).
Terracini regresó a Italia en 1947, pero el departamento de matemática quedó en
las ya expertas manos de F. E. Herrera, quien desde entonces lo ha mantenido en
estado activo y, sobre todo en su especialidad (análisis matemático, series de
Fourier), en un nivel elevado.

Herrera ha procurado activar las distintas especialidades, aconsejando a la
Universidad la contratación de profesores extranjeros y así, desde 1947, pasan
por Tucumán los matemáticos alemanes A. Lammel, W. Dahmk\"ohler, L. Koschmieder
y el japonés M. Itoh. Lamentablemente, a veces el ambiente y otras el cambio de
rumbo de las autoridades universitarias, hicieron que todos ellos permanecieran
en Tucumán por poco tiempo (dos, tres o cuatro años). Sin embargo, su paso no
fue, ni mucho menos, inútil. En el departamento de matemáticas se nota en
múltiples detalles la influencia del mismo. Hay biblioteca y ambiente para la
investigación. Algunos jóvenes, en especial R. Luccioni, han probado ya su
capacidad y vocación para la misma.

\emph{El Instituto de Matemáticas de Rosario}. En 1939 se crea en Rosario el
Instituto de Matemáticas, dependiente de la Facultad de Ciencias Matemáticas de
la Universidad Nacional del Litoral. Ya con anterioridad, el decano Cortés Pla,
secundado por un grupo de entusiastas profesores (C. Dieulefait, F. Gaspar, J.
Olguin, S. Rubinstein) había conseguido dotarlo de una excelente biblioteca. Se
contrata a Beppo Levi como director y a L. Santaló como investigador principal.
La labor desarrollada por el Instituto puede verse en sus dos series de
publicaciones: las \emph{Publicaciones del Instituto de Matemáticas} desde 1939
hasta 1946 y las \emph{Mathematicae Notae} iniciadas en 1941 y que continúan en
la actualidad.

No puede decirse que el Instituto haya logrado formar una escuela de
investigaciones, tal vez debido a estar unida a una Facultad de ingeniería, sin
un doctorado u otra carrera análoga más específicamente matemática, pero no hay
duda de que a su alrededor se ha formado el grupo de bien capacitados
profesores (E. Gaspar, E. O. Ferrari, J. Olguin) que hoy tienen a su cargo las
matemáticas de la Facultad. En sus seminarios se formó P. E. Zadunaisky,
especialista en cálculo numérico, hoy en Harvard, pero ya contratado por el
Instituto del Cálculo de Buenos Aires, como ya dijimos anteriormente. 

Gracias al apoyo de las autoridades de la Facultad y al canje con sus
publicaciones, la biblioteca del Instituto es de las mejores del país. Si bien
en el presente, con la jubilación de Beppo Levio, la actividad del Instituto
aparece disminuida, las bases están creadas y el ímpetu es latente para que
aparezca, en cualquier momento, un nuevo y más definitivo florecimiento.

\emph{Las matemáticas en la zona de Cuyo}. La Universidad Nacional de Cuyo
inicia los estudios de matemática llevando a los españoles M. Balanzat a San
Luis (1940), E. Corominas a Mendoza (1940) y P. Pi Calleja a San Juan (1941).
En San Juan, probablemente por formar parte de una Facultad de Ingeniería, es
donde tienen más arraigo. La obra iniciada por Pi Calleja y continuada, desde
1950, por el portugués A. Monteiro encuentra brillantes adeptos en A. Diego y
L. Bruschi, actualmente en Bahía Blanca el primero y en La Plata la segunda y,
principalmente, en O. R. Villamayor, que después de perfeccionarse durante dos
años en los Estados Unidos en álgebra homológica, en cuyo campo produce
importantes contribuciones, dirige actualmente el Instituto de Matemáticas de
Bahía Blanca. 

En 1954 se crea en Mendoza el Instituto de Matemáticas dependiente del D. I. C.
(Departamento de Investigaciones Científicas de la Universidad de Cuyo),
consiguiendo reunir por breve tiempo un grupo importante de matemáticos:
Cotlar, Monteiro, Zarantonello, Villamayor, Ricabarra, Voelker, Klimovsky,
Varsavsky, Bosch. Se inicia la publicación de la \emph{Revista Matemática
Cuyana}. Lamentablemente el Instituto tiene poca duración. Por causas diversas,
en 1957, sus componentes se dispersan y queda únicamente Zarantonello. La
Universidad decide entonces concentrar sus estudios de matemática superior en
San Luis, sede de la Facultad de Ciencias. Allí se traslada la biblioteca, ya
cuantiosa, y se contrata a los alemanes Voelker y Dahmk\"oler en cuya obra,
recién empezada, hay que cifrar halagüeñas esperanzas. 

\emph{El Departamento de Matemáticas de Bahía Blanca}. Desde el momento de su
fundación (1956) la Universidad Nacional del Sur demostró interés por los
estudios de matemática pura. Con buen acierto contrató a Monteiro, cuya
habilidad insuperable para formar escuela estaba bien probada por su actuación
anterior en Lisboa, Río de Janeiro y San Juan. Efectivamente, contraponiendo a
la falta de ambiente y a la carencia de biblioteca, su capacidad, entusiasmo y
tenacidad sin igual, consigue en menos de un lustro formar una escuela original
que promete desarrollarse con bríos al soplo de las brisas de las más nuevas
tendencias de la matemática moderna. Acompañan a Monteiro varios colaboradores:
Villamayor, siempre dedicado al álgebra moderna y en especial al álgebra
homológica; el portugués Ruy Gomes, analista distinguido, y A. Diego, joven
alumno de Monteiro dedicado a la lógica matemática. Por otra parte, es
frecuente que el Instituto cuente con profesores visitantes extranjeros,
gracias a la ayuda financiera de la universidad y muchas veces del Consejo
Nacional de Investigaciones Científicas y Técnicas. 

El afianzamiento de una escuela de investigadores no es cosa fácil. La
tradición y los años de vida son factores importantes para ello. Esperemos que
los vaivenes, tan frecuentes en nuestras universidades, no sean en la de Bahía
Blanca lo suficientemente acentuados como para destruir esta obra, ya que es un
ejemplo de cómo la capacidad, dedicación y vocación pueden salvar etapas y
formar, en pocos años, un centro comparable o superior a otros con mucha más
vieja tradición.

\emph{El Centro Regional de Matemáticas de Buenos Aires}. En 1959 se creó el
Centro Regional de Matemáticas para América Latina, mediante un convenio entre
la UNESCO y la Universidad de Buenos Aires. La UNESCO se compromete a enviar al
Centro diez becarios de distintos países latinoamericanos, cada dos años. La
Universidad de Buenos Aires debe proporcionarles alojamiento y asistencia
médica. Por otra parte, la Universidad de Buenos Aires permite a sus profesores
con dedicación exclusiva que dediquen al Centro la atención necesaria. Esto
hace que el Centro funcione en estrecha unión con la Facultad de Ciencias
Exactas y Naturales. Los cursos y seminarios son comunes a ambas instituciones.
La dirección del Centro está a cargo de A. González Domínguez, que a su vez es
el jefe del Departamento de Matemáticas de la Facultad. Profesores extranjeros,
enviados por la UNESCO en concepto de ayuda técnica, dictan anualmente uno o
dos cursos cuatrimestrales en el Centro.  Con ayuda del Consejo Nacional de
Investigaciones Científicas y Técnicas, el Centro y la Facultad invitan todos
los años a A. Calderón para que desarrolle un seminario entre los meses de
septiembre a diciembre. 

Los becarios del centro llegan en general con preparación muy diferente. Esto
hace que a unos se les aconseje seguir cursos regulares antes de dirigirlos
hacia temas más específicos, mientras que otros pueden polarizar de inmediato
su atención hacia temas concretos de investigación. Este año termina el primer
grupo de becarios que han procedido de Ecuador, Paraguay, Perú, Uruguay y
Venezuela. Los resultados obtenidos, en cada caso dentro de su nivel inicial,
han sido satisfactorios.

Hemos pasado revista de los centros matemáticos de la Argentina en los cuales
la investigación, sea en forma activa o como anhelo latente, es la principal
preocupación. No hemos detallado la producción conseguida. Ello sería fácil
resumiendo las críticas publicadas en el \emph{Mathematical Reviews} o en el
\emph{Zentralblatt f\"ur Mathematik} de los trabajos publicados por los
matemáticos mencionados y otros que podemos haber olvidado. Si lo mismo se
hiciera para los años anteriores a 1940, sustituyendo dichas revistas por la
más antigua y ya desaparecida \emph{Jahrbuch \"uber die Fortschritte der
Mathematik}, se vería de manera objetiva confirmada nuestra afirmación, por
otra parte evidente: la producción matemática original empieza en 1920, sigue
lentamente hasta 1940, para aumentar luego con mayor pendiente hasta nuestros
días. Si a la cantidad de los trabajos se le agrega el peso de la calidad, lo
que puede hacerse también de manera objetiva por el tenor de los comentarios en
las revistas citadas o por la repercusión que los trabajos han tenido para
otras publicaciones, el brusco cambio de pendiente que se opera entre 1940 y
1950 es todavía más notorio.

Estamos, pues, en punto ascendente, de máxima pendiente respecto del pasado.
Responsabilidad de los matemáticos es hacer que este impulso ascendente, que
inició Rey Pastor, siga sin alto ni decadencia. Las principales dificultades,
que como las de toda empresa fueron las iniciales, ya están vencidas. Pero no
hay que descuidarse; el ritmo del progreso debe ser cada día mayor si no se
quiere retroceder respecto de los centros que en el mundo van a la vanguardia
del mismo. La multiplicidad de éstos, y la protección con que cuentan en muchos
países, imprimen al progreso general un ritmo febril. Cuidemos, sin embargo,
que ello no alimente un suicida complejo de inferioridad. Y repitamos que las
grandes cosas son siempre la unión de otras más pequeñas: no esperemos, con las
manos cruzadas, la llegada de una inspiración genial que revolucione de golpe
toda la ciencia. Quienes adoptan esta posición, la más \emph{cómoda} por otra
parte, es muy probable que no hagan nunca nada. Alentemos, por el contrario,
las pequeñas producciones originales, puesto que ellas pueden ser el germen de
otras mayores y, en todo caso, nunca lo pequeño ha impedido que el capaz y
predestinado llegue a los descubrimientos geniales.

\bigskip
{\small{\noindent Departamento de Matemática\\
Facultad de Ciencias Exactas y Naturales, U. B. A.}
