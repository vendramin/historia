\chapter{El teorema de Fermat}

Si bien Fermat hizo muchas contribuciones a la matemática, este capítulo se refiere
a un resultado conjeturado por Fermat en 1637 y demostrado por Wiles en 1995. El teorema afirma
que no existen soluciones enteras no triviales 
de la ecuación
\[
	x^n+y^n=z^n
\]
para $n\geq3$. 

El resultado fue escrito por Fermat en los márgenes de una copia de un libro de
Diofanto y acompañado de una frase que decía que no había espacio suficiente en
esos márgenes para escribir la maravillosa demostración que él había
encontrado.  Durante mucho tiempo muchos matemáticos de gran nombre intentaron
en vano demostrar que no existen soluciones no triviales. La lista de nombres
es larga: Euler, Legendre, Gauss, Abel, Cauchy, Dirichlet, Lame, Kummer,
Frobenius, Furtwangler, Dickson\dots 

Fermat lo demostró en el caso $n=4$ y en 1770 Euler en el caso $n=3$, aunque
con algunos pequeños problemas que fueron resueltos más tarde por otros
matemáticos. Desde hace mucho tiempo que sabe que alcanza con demostrar el
teorema para los $n$ que son números primos.  Legendre en 1823 lo demostró para
$n=5$ y Lame en 1839 para $l=7$. 

Error de Lame\dots
Comentario de Cauchy\dots
Sophie Germain\dots

En 1850 Kummer lo demostró para los primos regulares. 
Los primeros primos regulares impares son
\[
    3, 5, 7, 11, 13, 17, 19, 23, 29, 31, 41, 43, 47, 53, 61, 71, 73\dots
\]
y aparecen en la sucesión \verb+A007703+. No se sabe si hay infinitos primos
regulares. En 1965 Siegel conjeturó que hay infinitos primos regulares y que representan
aproximadamente el 60\% del total de los primos. En 1915 Jensen 
demostró que 
hay infinitos primos irregulares.





