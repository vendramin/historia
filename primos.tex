\chapter{Números primos}

Como bien señala Weil en~\cite{MR734177}, el amor del hombre por los números es
quizá aún más antiguo que la teoría de números.  

\index{Número!primo}
Los griegos consideraban a los
números primos como números que no admiten representaciones rectangulares.  Los
primeros números primos son
\[
    2, 3, 5, 7, 11, 13, 17, 19, 23, 29, 31, 37, 41, 43, 47, 53, 59, 61, 67, 71, 73, 79\dots
\]
La sucesión de números primos es \verb+A000040+. 

El teorema fundamental de la aritmética afirma que todo número puede escribirse
como producto de números primos y esa escritura es esencialmente única. Este
resultado no aparece explícitamente en los libros de Euclides, aunque hay
muchos resultados del libro VII que son casi equivalentes. Tampoco aparece en
\emph{Essai sur la Théorie des Nombres}, famoso tratado escrito por Legendre de
1798 sobre teoría de números. La primera aparición precisa junto con una
demostración rigurosa figura en el famoso \emph{Disquisitiones arithmeticae} de
Gauss de 1801. 

Euclides demostró en la proposición 20 del libro IX que existen infinitos
números primos. 

\begin{quote}
	Hay más números primos que cualquier cantidad propuesta de números primos.
\end{quote}
La demostración es bien conocida y revela que la existencia de una sucesión de
enteros positivos coprimos dos a dos, implicará la infinitud de los números
primos. En efecto, si $a_1,a_2,\dots$ es una sucesión de enteros coprimos dos a
dos y para cada $j$ se toma un primo $p_j$ que divide al elemento $a_j$,
entonces los $p_j$ serán todos distintos.

Recordemos que los números de Fermat son los enteros de la forma
\[
	F_n=s^{2^n}+1.
\]
En un una carta que Goldbach le escribió a Euler en julio de 1730, se demuestra
que los números de Fermat son coprimos dos a dos. Este resultado da entonces
una demostración alternativa de la infinitud de los números primos.

\begin{theorem}(Goldbach)
	\index{Teorema!de Goldbach}
	Los numeros de Fermat son coprimos dos a dos.
\end{theorem}

\begin{proof}
	Se demuestra fácilmente por inducción que $F_n-2=F_0F_1\cdots F_{n-1}$ vale
	para todo $n\in\N$. Si $m<n$, entonces $F_m$ divide a $F_n-2$. Si $p$ es un
	número primo tal que $p\mid F_n$ y $p\mid F_m$, entonces $p\mid F_n-2$ y
	luego $p\mid 2$, una contradicción pues $F_m$ es un número impar. 
\end{proof}

Desde 1997 todos los primos de Mersenne descubiertos se hicieron dentro del
proyecto GIMPS. Este proyecto de cálculo distribuido fue creado por George
Woltman en 1996 y tiene como objetivo buscar números primos de Mersenne. GIMPS
es uno de los primeros proyectos de cálculo distribuido y es notablemente
exitoso: de los 50 primos de Mersenne conocidos, los últimos 17 encontrados
fueron dentro de este proyecto.  El mayor primo de Mersenne conocido hasta el
momento es 
\[
	2^{82589933}-1
\]
y tiene 24862048 dígitos; fue descubierto en diciembre de 2018. Este número es
además el mayor número primo que se conoce. 

Hoy en día los números primos son importantes no solo en matemática pura sino
en la criptografía. Muchos de los algoritmos de encriptación utilizados se
basan en el uso de números primos. 

Existen muchas demostraciones distintas de la infinitud de los números primos. 
En\dots Euler demostró que existen infinitos primos al considerar la divergencia de la serie
\[
	\sum\frac{1}{p},
\]
donde la suma se toma sobre el conjunto de números primos. 

Desde tiempos de Gauss $\pi(x)$ es la cantidad de primos $\leq x$. Gauss
conjeturó que
\[
	\pi(x)\sim \frac{x}{\log x}.
\]
Durante muchos años los matemáticos intentaron en vano demostrar aquella
conjetura. En 1859, Riemann atacó el problema con métodos analíticos y encontró
una sorprenden conexión entre los números primos y la función de variable
compleja 
\[
\zeta(s)=\sum_{n\geq1}1/n^{s}. 
\]
Esta función hoy se conoce como la \textbf{función zeta de Riemann} y es de
fundamental importancia en la matemática moderna. De hecho, uno de los
problemas más importantes de la matemática, que se conoce como la hipótesis de
Riemann, es una conjetura sobre la distribución de los ceros de la función
$\zeta(s)$. 

El teorema del número primo fue demostrado independiente por de la Valle
Poussin y por Hadamard en 1896; en ambos casos, la demostración se basa en el
estudio de propiedades analíticias de la función de variable compleja
$\zeta(s)=\sum_{n\geq1}1/n^{s}$. En 1949, Selberg e Erd\"os encontraron,
independientemente, una demostración elemental del teorema del número primo.
Esta demostración elemental generó mucho revuelo en la comunidad matemática y
muchas discusiones entre Erd\"os y Selberg, ya que ambos, de alguna forma,
consideraban merecer el crédito de haber encontrado una demostración elemental
del teorema del número primo.  Cabe aclarar que cuando nos referimos a una
demostración elemental, nos referimos a una demostración que solamente utiliza
técnicas básicas, aunque en general sean demostraciones muy difíciles de
entender. Es importante recordar lo siguiente: demostración elemental no quiere
significa sencilla y fácil de entender. 

Existen fórmulas para encontrar números primos pero son absolutamente
ineficientes y no pueden usarse en aplicaciones serias. Mencionaremos un resultado
demostrado por Mills en 1946:

\begin{theorem}[Mills]
	Existe un número real $A$ tal que $\lfloor A^{3^n}\rfloor$ es un número
	primo para todo $n\in\N$.	
\end{theorem}

\begin{proof}
	Para la demostración necesitamos utilizar el siguiente resultado demostrado
	por Ingham en 1937: Si $p_n$ denota al $n$-ésimo número primo, existe una
	constante $K$ tal que 
	\[
		p_{n+1}-p_n\leq Kp_n^{5/8}.
	\]

	\begin{claim}
		Si $N>K^8$, entonces existe un primo $p$ tal que $N^3<p<(N+1)^3-1$. 
	\end{claim}

	Para demostrar esta afirmación, sea $p_n$ el mayor número primo mayor que
	$N^3$. Como $N>K^8$, entonces $N^{1/8}>K$ y luego
	\[
		N^3<p_{n+1}<p_n+Kp_n^{5/8}<N^3+KN^{15/8}<N^3+N^2<(N+1)^3-1.
	\]

	Sea $P_0$ un número primo tal que $P_0>K^8$. La afirmación anterior nos
	permite construir una sucesión $P_0,P_1,P_2\dots$ de primos tales que
	\[
		P_n^3<P_{n+1}<(P_n+1)^3-1.
	\]
	Sean 
	\[
		u_n=P_n^{3^{-n}},\quad
		v_n=(P_n+1)^{3^{-n}}.
	\]
	Vamos a demostrar que la sucesión $u_1,u_2\dots$ converge. Para eso veremos que es monótona y acotada. 
	Primero observamos que, como 
	\begin{align*}
		&u_{n+1}=P_{n+1}^{3^{-n-1}}>P_n^{3^{-n}}=u_n,
	\end{align*}
	la sucesión $(u_n)_{n\in\N}$ tiene límite pues es monótona y acotada. 
	Sea $A=\lim_{n\to\infty}u_n$. Como además 
	\begin{align*}
		&v_n=(P_n+1)^{3^{-n}}>P_n^{3^{-n}}=u_n,\\
		&v_{n+1}=(P_{n+1}+1)^{3^{-n-1}}<(P_n+1)^{3^{-n}}=v_n,
	\end{align*}
	se tiene que
	$u_n<A<v_n$ para todo $n$. Luego $P_n<A^{3^n}<P_n+1$ para todo $n$ y el
	teorema queda demostrado al tomar parte entera pues $\lfloor
	A^{3^n}\rfloor=P_n$. 
\end{proof}

El resultado anterior no tiene aplicaciones prácticas. No se conoce el valor
real de la constante $A$ del teorema; de hecho, ni siquiera se sabe si $A$ es
un número racional. Sin embargo, si se asume la veracidad de la hipótesis de
Riemann, puede demostrarse que $A$ es aproximadamente igual a
\[
	1.3063778838630806904686144926\dots
\]
Tampoco se conocen los primos que produce la función de Mills. Si se asume la veracidad de la hipótesis de Riemann
puede demostrarse que los primeros primos producidos por la función de Mills son
\[
    2,11,1361,2521008887, 16022236204009818131831320183\dots
\]
Estos números son los elementos de la sucesión \verb+A051254+.  

Es natural preguntarse si puede demostrarse algún resultado similar al de Mills
pero sin apelar al profundo teorema de Ingham. 

En 1845 Bertrand conjeturó que
dado $n\geq1$ siempre existe un número primo $p$ tal que $n<p<2n$.  Si bien
Bertrand pudo comprobar que la conjetura era cierta para todo $n\leq 3\times
10^6$, no pudo demostrarlo para todo $n$. La conjetura fue demostrada
completamente por Chebyshev 1n 1852. Ramanujan encontró una demostración más
sencilla que la encontrada por Chebyshev en 1919.  En 1932 Erd\"os dio una
demostración incluso más sencilla y completamente elemental que utiliza
solamente propiedades de los números combinatorios. 

En 1951 Wright demostró que existe $\alpha\in\R$ tal que si $g_0=\alpha$ y $g_{n+1}=2^{g_n}$, entonces
$\lfloor g_n\rfloor$ es siempre un número primo. La demostración de este resultado puede consultarse en~\cite{MR43805}; 
es similar a la demostración 
que hicimos del teorema de Mills, aunque el teorema de Ingham será reemplazado por el teorema de Bertrand--Chebyshev.


\section*{La conjetura de Goldbach}

\index{Constante!de Brun}
\index{Pentium FDIV bug}
\index{Nicely, Thomas}
\index{Teorema!de Brun}
La \textbf{conjetura de Goldbach} afirma que todo entero par mayor que dos
puede expresarse como suma de dos primos. Por ejemplo:
\[
	10=3+7,\quad
	18=5+13,\quad
	20=7+13,\quad
	90=43+47.
\]
Una versión de esta conjetura aparece en una carta del 7 de junio de 1742 que
Goldbach le escribió a Euler. Aparentemente esta conjetura era conocida desde
tiempo antes; en los trabajos póstumos de Descartes puede encontrarse una
afirmación similar a la conjetura de Goldbach: no está demostrado pero todo
número puede escribirse como suma de uno, dos o tres números primos.

A pesar de los esfuerzos de muchos matemáticos, la conjetura permanece abierta. 

Gracias a cálculos computacionales se sabe que la conjetura es verdadera para
números menores que $4\times 10^{17}$. En 1937 Vinogradov demostró que todo
número impar suficientemente grande es suma de tres números primos. En 1948
Rényi demostró que existe un entero $M$ tal que cada número impar $n$
suficientemente grande es suma de un primo y otro número que tiene a lo sumo
$M$ factores primos. Si alguien pudiera demostrar que $M=1$, se tendría
entonces la veracidad de la conjetura de Goldbach. En 1966 Jing-Run demostró
que $M\leq 2$. 

Recientemente pudo demostrarse una versión débil de la conjetura de Goldbach:
Todo número impar mayor que cinco puede expresarse como suma de tres números
primos.  Esta conjetura es la \textbf{conjetura débil de Goldbach}. Es fácil
demostrar que la veracidad de la conjetura de Goldbach implicaría la veracidad
de la versión débil de la conjetura.  La conjetura débil apareció publicada sin
demostración por primera vez en 1770 en el tratado \emph{Meditationes
algebraicae}, escrito por el matemático inglés Edward Waring. Fue demostrada
por el matemático peruano Harald Helfgott en 2013. 

\section*{La infinitud de los primos gemelos}

Dos números primos $p$ y $q$ son números \textbf{primos gemelos} si $|p-q|=2$.
Los primeros ejemplos son
\[
	(3, 5), (5, 7), (11, 13), (17, 19), (29, 31), (41, 43),\dots
\]
La sucesión de primos gemelos es \verb+A077800+.  
En 2009 se demostró que los números 
\[
	65516468355\times 2^{333333}+1,
	\quad
	65516468355\times 2^{333333}-1,
\]
son primos gemelos; cada uno de estos números tiene 100355 dígitos.  Hasta
ahora no se descubrieron primos gemelos mayores. Este par de primos se encontró
dentro del proyecto de cálculo distribuido PrimeGrid, desarrollado por Rytis
Slatkevi\v{c}ius.  Se tardó más de dos años en encontrar este par de primos y
en esta tarea colaboraron 18661 personas. Según el reporte oficial, una única
computadora de escritorio hubiera tardado cerca de dos siglos en encontrar
estos números.

Desde hace mucho tiempo se conjetura que existen infinitos primos gemelos. 

En 1919 Brun demostró que la serie
\[
	\sum\left(\frac1p+\frac1{p+2}\right)=\left(\frac13+\frac15\right)+\left(\frac15+\frac17\right)+\left(\frac1{11}+\frac1{13}\right)+\cdots,
\]
donde la suma se toma sobre todos los primos $p$ tales que $p+2$ es también un
número primo, es convergente. En 1994, mediante el cálculo de los primos
gemelos menores que $10^{14}$ el matemático estadounidense Thomas Nicely
determinó que la serie converge a 
\[
1,9021605777\dots,
\]
número que hoy se conoce
como la \textbf{constante de Brun}. Fue durante este cálculo que Nicely
descubrió un problema en los procesadores Pentium, hoy conocido como el
``Pentium FDIV bug''. La prestigiosa revista Science publicó un artículo en
1995 que describe la importancia de los problemas de teoría de números para
descubrir problemas como el que encontró Nicely.  Después del incidente Intel,
acabó cooperando con Nicely para testear los nuevos microprocesadores.

En 2013 Zhang demostró que existe un entero $N$ tal que existen infinitos pares
de primos cuya diferencia es $N$; el entero $N$ es aproximadamente igual a
$70\times10^6$.  Este trabajo tuvo gran repercusión, ya que no se conocían
resultados de este tipo desde tiempos de Euclides.  Después de la publicación
del trabajo de Zhang, 
%Terence Tao propuso trabajar en conjunto dentro un
%proyecto Polymath para intentar optimizar la cota de setenta millones
%encontrada por Zhang. 
en 2014, un grupo de matemáticos logró reducir la cota a
246. 
%Tao y Maynard, independientemente, lograron reducir estas cotas
%notablemente si ciertas conjeturas de la teoría de números fueran verdaderas. 

¿Qué es un proyecto Polymath? Es un proyecto de colaboración entre matemáticos
con la objetivo de resolver algún problema difícil. La idea nació en 2009 en el
blog del matemático inglés Timothy Gowers, donde se preguntó si era posible
trabajar masiva y colaborativamente en matemática.

En general, los trabajos realizados bajo esta modalidad se firman con el
seudónimo D. H. J. Polymath. Sin embargo, el cuatro proyecto Polymath fue
firmado por Tao, Croot y Helfgott, porque los editories de la revista
\emph{Mathematics of Computation} pidieron que en la versión final del trabajo
figuraran los nombres reales de los autores. 



